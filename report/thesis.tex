\documentclass[a4paper]{report}

\usepackage{fontspec}
\usepackage{polyglossia}
\setdefaultlanguage[spelling=modern]{russian}
\setotherlanguage{english}
\setmainfont[Ligatures=TeX]{FreeSerif}
\newfontfamily\cyrillicfontsf{FreeSerif}
\newfontfamily\cyrillicfontit{FreeSerif}
\newfontfamily\cyrillicfonttt{FreeSerif}

\usepackage[backend=biber,bibencoding=utf8,sorting=none,sortcites=true,bibstyle=sty/gost71,maxnames=99,citestyle=numeric-comp,babel=other]{biblatex}
%\usepackage[utf8]{inputenc}
\usepackage{textcomp}
\usepackage{csquotes}
\usepackage{caption}
\usepackage{subcaption}
%\usepackage[pdftex]{graphicx}
\usepackage{amsmath}
\usepackage{amssymb}
\usepackage{amsthm}
\usepackage{sty/dbl12}
\usepackage{sty/rac}
\usepackage{epigraph}
\usepackage{agda}

\usepackage{tikz}
\usetikzlibrary{arrows,positioning,shapes}
\tikzset{context/.style={circle, draw=black, minimum size=1cm}}
\tikzset{patch/.style={fill=white}}
\tikzset{gitbranch/.style={rectangle, rounded corners, draw=black}}
\tikzset{gitcommit/.style={rectangle, rounded corners, draw=black,
    minimum width = 1.5cm, rectangle split, rectangle split parts = 2}}
\tikzset{gitbranchconnect/.style={->,dashed}}


\colorlet{patchform}{blue!40}
\usepackage{tikz}
\usetikzlibrary{shapes.multipart}
\usetikzlibrary{chains}
\usetikzlibrary{fit}
\usetikzlibrary{backgrounds}
\usetikzlibrary{arrows}
\tikzset{vecpatch/.style={
    draw=black, 
    minimum width=0.5cm, 
    minimum height=1cm,
    inner sep=-1cm,
    outer sep=-1cm,
    anchor=center}
}
\tikzset{vecpatch-tiny/.style={vecpatch, minimum height=0.5cm}}
\tikzset{vecpatch-f/.style={vecpatch}}
\tikzset{vecpatch-t/.style={vecpatch,fill=yellow}}
\tikzset{vecpatch-d/.style={vecpatch-tiny,anchor=base}}
\tikzset{vecpatch-op/.style={font=\large}}
\tikzset{vecpatch-form/.style={vecpatch-tiny,fill=patchform}}

\tikzset{treev/.style={draw=black,circle,outer sep=0,minimum
    height=8mm,very thick}}
\tikzset{treearr/.style={draw, -latex, very thick}}

\newcommand{\vecfe}[0]{\node[vecpatch-tiny] {};}
\newcommand{\vecff}[0]{\node[vecpatch-form] {};}
\newcommand{\vecf}[0]{\node[vecpatch-f] {};}
\newcommand{\vecd}[1]
{\node[vecpatch-d, draw=none] {$#1$}; 
 \node[vecpatch-d, text opacity=0] {b};}
\newcommand{\vect}[2]{\node[vecpatch-t]
{\begin{tabular}{l}$#1$\\$#2$\\\end{tabular}};}



\theoremstyle{plain}
\newtheorem{theorem}{Теорема}

\theoremstyle{definition}
\newtheorem{definition}[theorem]{Определение}
\newtheorem{notation}[theorem]{Обозначение}

% Redefine margins and other page formatting

\setlength{\oddsidemargin}{0.5in}

\binoppenalty=10000
\relpenalty=10000

\addbibresource{thesis.bib}

\defbibenvironment{bibliography} {\list
  {\printfield[labelnumberwidth]{labelnumber}.}
  {\setlength{\labelwidth}{2\labelnumberwidth}%
    \setlength{\leftmargin}{\labelwidth}%
    \setlength{\labelsep}{\biblabelsep}%
    \addtolength{\leftmargin}{\labelsep}%
    \setlength{\itemsep}{\bibitemsep}%
    \setlength{\parsep}{\bibparsep}}%
  \renewcommand*{\makelabel}[1]{\hss##1}} {\endlist} {\item}


\begin{document}

\pagestyle{title}

\begin{center}
  Университет ИТМО \\
  \hrulefill
 
\vspace{2cm}

Факультет информационных технологий и программирования

Кафедра компьютерных технологий

\vspace{3cm}

{\Large Комаров Андрей Валерьевич}

\vspace{2cm}

\vbox{\LARGE\bfseries Аксиоматизация системы контроля версий в
  формализме зависимых типов }

\vspace{4cm}

{\Large Научный руководитель: ассистент кафедры ТП Я.~М.~Малаховски}

\vspace{6cm}

Санкт-Петербург\\ \today
\end{center}


\newpage
\pagestyle{plain}

\tableofcontents

% Chapters
\startthechapters
\startprefacepage

Разнообразные системы контроля версий повседневно используются почти
всеми разработчиками программного обеспечения для работы над проектом.
В них могут одновременно разрабатываться множество различных
возможностей программы. Являясь одним из основных инструментов в
разработке программного обеспечения, система контроля версий должна
предоставлять гарантии своей корректной работы. Способом достижения
этих гарантий является, например, реализация её на каком-либо языке
доказательного программирования. Однако, подавляющее большинство
систем контроля версий реализованы без какой-либо низлежащей теории.
SVN~\cite{svnbook}, git~\cite{git} и прочие повсеместно используемые
системы работают на основе алгоритмов и структур данных, для которых
нет формального доказательства корректности их работы.

% не забыть и про ad-hoc структуры данных
%% например, так?

% Этот параграф надо менее подробно и более обобщённо. Про Coq тут
% точно не надо.

Ранее предпринималось несколько попыток сделать систему контроля
версий, обладающую такими свойствами. Например, для Darcs~\cite{darcs}
была разработана соответствующая \emph{теория патчей}, однако,
доказанной реализации у неё нет. Также существует проект
Camp~\cite{camp}, в котором поставлена цель получить полностью
доказанную Darcs-подобную систему контроля версий, однако этот проект
пока далёк от завершения.

Целью настоящей работы является разработка нескольких <<каркасов>> для
класса систем контроля версий на основе одной общей идеи
<<логистического>> характера с их формализацией, формулировкой и
доказательством некоторых их свойст на языке доказательного
программирования Agda~\cite{agda}.

\chapter{Обзор предметной области}
\label{chapter1}

\section{Выводы по главе \ref{chapter1}}

\chapter{Процесс решения задачи}
\label{chapter2}




\section{Выводы по главе \ref{chapter2}}


%\chapter{Результаты} 
\label{chapter3}

\input{chapters/Ololo.tex}

\section{Выводы по главе \ref{chapter3}}

\startconclusionpage

В данной работе была сформулирована задача создания доказанной системы
контроля версий. К сожалению, эта задача слишком обширна и полностью
решить её в рамках этой работы не представляется возможным. Были
сделаны первые шаги к её решению. На языке Agda было написано две
основы для системы контроля версий. В одной из них ревизии являются
векторами какой-то длины над произвольным типом. Во второй~---
двоичными деревьями. Для этих двух видов ревизий были описаны патчи,
операции над патчами и свойства этих операций. 

Дальнейшим развитием этой работы видится следующее. Замечено, что два
рассмотренных случая получились очень похожими. Хотелось бы выделить
какой-то общий интерфейс, набор типов и операций, имея который, можно
сказать, что нечто является системой контроля версий. 

Другим вариантом дальнейшего развития видится реализация интерфейса
\emph{настоящего} приложения, с которым может работать пользователь.
Небольшой помехой для этого является то, что в текущей реализации для
выполнения операций над патчами требуется иметь доказательство того,
что эту операцию можно сделать. Эту проблему можно решить, если
написать функцию, по двум патчам возвращающую либо доказательство
того, что их можно объединить, либо доказательство того, что нельзя.
Для первого из двух реализованных вариантов такая функция уже
реализована. % однако, в продакшен^W отчёт за ненадобностью не попала

Также, возможны дальнейшие исследования того, какие ещё структуры
могут быть использованы для ревизий. Например, исследование
возможности формализации UNIX-патчей, или, что более, вероятно, патчей
Darcs.

\printbibliography

\end{document}
