\documentclass[a4paper]{report}

\usepackage{fontspec}
\usepackage{polyglossia}
\setdefaultlanguage[spelling=modern]{russian}
\setotherlanguage{english}
%\newcommand{\myfont}[0]{XITS}
\newcommand{\myfont}[0]{FreeSerif}
%\newcommand{\myfont}[0]{Comic Sans MS}
\setmainfont[Ligatures=TeX]{\myfont}
\newfontfamily\cyrillicfontsf{\myfont}
\newfontfamily\cyrillicfontit{\myfont}
\newfontfamily\cyrillicfonttt{\myfont}

\usepackage[backend=biber,bibencoding=utf8,sorting=none,sortcites=true,bibstyle=sty/gost71,maxnames=99,citestyle=numeric-comp,babel=other]{biblatex}
%\usepackage[utf8]{inputenc}
\usepackage{textcomp}
\usepackage{csquotes}
\usepackage{caption}
\usepackage{subcaption}
%\usepackage[pdftex]{graphicx}
\usepackage{amsmath}
\usepackage{amssymb}
\usepackage{amsthm}
\usepackage{sty/dbl12}
\usepackage{sty/rac}
\usepackage{epigraph}
\usepackage{agda}

\usepackage[toc,page]{appendix}
\renewcommand{\appendixtocname}{Приложения}
\renewcommand{\appendixpagename}{Приложения}
\renewcommand{\appendixname}{Приложение}

\usepackage{tikz}
\usetikzlibrary{arrows,positioning,shapes}
\tikzset{context/.style={circle, draw=black, minimum size=1cm}}
\tikzset{patch/.style={fill=white}}
\tikzset{gitbranch/.style={rectangle, rounded corners, draw=black}}
\tikzset{gitcommit/.style={rectangle, rounded corners, draw=black,
    minimum width = 1.5cm, rectangle split, rectangle split parts = 2}}
\tikzset{gitbranchconnect/.style={->,dashed}}


\colorlet{patchform}{blue!40}
\usepackage{tikz}
\usetikzlibrary{shapes.multipart}
\usetikzlibrary{chains}
\usetikzlibrary{fit}
\usetikzlibrary{backgrounds}
\usetikzlibrary{arrows}
\tikzset{vecpatch/.style={
    draw=black, 
    minimum width=0.5cm, 
    minimum height=1cm,
    inner sep=-1cm,
    outer sep=-1cm,
    anchor=center}
}
\tikzset{vecpatch-tiny/.style={vecpatch, minimum height=0.5cm}}
\tikzset{vecpatch-f/.style={vecpatch}}
\tikzset{vecpatch-t/.style={vecpatch,fill=yellow}}
\tikzset{vecpatch-d/.style={vecpatch-tiny,anchor=base}}
\tikzset{vecpatch-op/.style={font=\large}}
\tikzset{vecpatch-form/.style={vecpatch-tiny,fill=patchform}}

\tikzset{treev/.style={draw=black,circle,outer sep=0,minimum
    height=8mm,very thick}}
\tikzset{treearr/.style={draw, -latex, very thick}}

\newcommand{\vecfe}[0]{\node[vecpatch-tiny] {};}
\newcommand{\vecff}[0]{\node[vecpatch-form] {};}
\newcommand{\vecf}[0]{\node[vecpatch-f] {};}
\newcommand{\vecd}[1]
{\node[vecpatch-d, draw=none] {$#1$}; 
 \node[vecpatch-d, text opacity=0] {b};}
\newcommand{\vect}[2]{\node[vecpatch-t]
{\begin{tabular}{l}$#1$\\$#2$\\\end{tabular}};}



\theoremstyle{plain}
\newtheorem{theorem}{Теорема}

\theoremstyle{definition}
\newtheorem{definition}[theorem]{Определение}
\newtheorem{notation}[theorem]{Обозначение}

% Redefine margins and other page formatting

\setlength{\oddsidemargin}{0.5in}

\binoppenalty=10000
\relpenalty=10000

\addbibresource{thesis.bib}

\defbibenvironment{bibliography} {\list
  {\printfield[labelnumberwidth]{labelnumber}.}
  {\setlength{\labelwidth}{2\labelnumberwidth}%
    \setlength{\leftmargin}{\labelwidth}%
    \setlength{\labelsep}{\biblabelsep}%
    \addtolength{\leftmargin}{\labelsep}%
    \setlength{\itemsep}{\bibitemsep}%
    \setlength{\parsep}{\bibparsep}}%
  \renewcommand*{\makelabel}[1]{\hss##1}} {\endlist} {\item}


\begin{document}

\pagestyle{title}

\begin{center}
  Университет ИТМО \\
  \hrulefill
 
\vspace{2cm}

Факультет информационных технологий и программирования

Кафедра компьютерных технологий

\vspace{3cm}

{\Large Комаров Андрей Валерьевич}

\vspace{2cm}

\vbox{\LARGE\bfseries Аксиоматизация системы контроля версий в
  формализме зависимых типов }

\vspace{4cm}

% Общее правило: в таких местах пишут научные звания, а не должности у
% меня научных званий нет.
% По этому же правилу ты не "студент", а просто "ты" чуть выше.
{\Large Научный руководитель: Я.~М.~Малаховски}

\vspace{6cm}

Санкт-Петербург\\ \today
\end{center}


\newpage
\setcounter{page}{3}
\pagestyle{plain}

\tableofcontents

% Chapters
\startthechapters
\startprefacepage

Разнообразные системы контроля версий повседневно используются почти
всеми разработчиками программного обеспечения для работы над проектом.
В них могут одновременно разрабатываться множество различных
возможностей программы. Являясь одним из основных инструментов в
разработке программного обеспечения, система контроля версий должна
предоставлять гарантии своей корректной работы. Способом достижения
этих гарантий является, например, реализация её на каком-либо языке
доказательного программирования. Однако, подавляющее большинство
систем контроля версий реализованы без какой-либо низлежащей теории.
SVN~\cite{svnbook}, git~\cite{git} и прочие повсеместно используемые
системы работают на основе алгоритмов и структур данных, для которых
нет формального доказательства корректности их работы.

% не забыть и про ad-hoc структуры данных
%% например, так?

% Этот параграф надо менее подробно и более обобщённо. Про Coq тут
% точно не надо.

Ранее предпринималось несколько попыток сделать систему контроля
версий, обладающую такими свойствами. Например, для Darcs~\cite{darcs}
была разработана соответствующая \emph{теория патчей}, однако,
доказанной реализации у неё нет. Также существует проект
Camp~\cite{camp}, в котором поставлена цель получить полностью
доказанную Darcs-подобную систему контроля версий, однако этот проект
пока далёк от завершения.

Целью настоящей работы является разработка нескольких <<каркасов>> для
класса систем контроля версий на основе одной общей идеи
<<логистического>> характера с их формализацией, формулировкой и
доказательством некоторых их свойст на языке доказательного
программирования Agda~\cite{agda}.

\chapter{Обзор предметной области}
\label{chapter1}

\section{Выводы по главе \ref{chapter1}}

\chapter{Процесс решения задачи}
\label{chapter2}




\section{Выводы по главе \ref{chapter2}}


\startconclusionpage

В данной работе была сформулирована задача создания доказанной системы
контроля версий. К сожалению, эта задача слишком обширна и полностью
решить её в рамках этой работы не представляется возможным. Были
сделаны первые шаги к её решению. На языке Agda было написано две
основы для системы контроля версий. В одной из них ревизии являются
векторами какой-то длины над произвольным типом. Во второй~---
двоичными деревьями. Для этих двух видов ревизий были описаны патчи,
операции над патчами и свойства этих операций. 

Дальнейшим развитием этой работы видится следующее. Замечено, что два
рассмотренных случая получились очень похожими. Хотелось бы выделить
какой-то общий интерфейс, набор типов и операций, имея который, можно
сказать, что нечто является системой контроля версий. 

Другим вариантом дальнейшего развития видится реализация интерфейса
\emph{настоящего} приложения, с которым может работать пользователь.
Небольшой помехой для этого является то, что в текущей реализации для
выполнения операций над патчами требуется иметь доказательство того,
что эту операцию можно сделать. Эту проблему можно решить, если
написать функцию, по двум патчам возвращающую либо доказательство
того, что их можно объединить, либо доказательство того, что нельзя.
Для первого из двух реализованных вариантов такая функция уже
реализована. % однако, в продакшен^W отчёт за ненадобностью не попала

Также, возможны дальнейшие исследования того, какие ещё структуры
могут быть использованы для ревизий. Например, исследование
возможности формализации UNIX-патчей, или, что более, вероятно, патчей
Darcs.

\printbibliography

\begin{appendices}
  \renewcommand{\thechapter}{\arabic{chapter}}
 \renewcommand{\chaptername}{Приложение}
  \chapter{Исходный код программ}
\label{sec:source}

\begin{code}%
\>\AgdaKeyword{module} \AgdaModule{VCSreport} \AgdaKeyword{where}\<%
\\
%
\\
\>\AgdaKeyword{open} \AgdaKeyword{import} \AgdaModule{OXIj.BrutalDepTypes}\<%
\\
%
\\
\>\AgdaKeyword{module} \AgdaModule{With-⋀-and-⋙} \AgdaSymbol{(}\AgdaBound{A} \AgdaSymbol{:} \AgdaPrimitiveType{Set}\AgdaSymbol{)} \AgdaSymbol{(}\AgdaBound{eqA} \AgdaSymbol{:} \AgdaSymbol{(}\AgdaBound{a} \AgdaBound{b} \AgdaSymbol{:} \AgdaBound{A}\AgdaSymbol{)} \AgdaSymbol{→} \AgdaDatatype{Dec} \AgdaSymbol{(}\AgdaBound{a} \AgdaDatatype{≡} \AgdaBound{b}\AgdaSymbol{))} \AgdaKeyword{where}\<%
\\
%
\\
\>[0]\AgdaIndent{2}{}\<[2]%
\>[2]\AgdaKeyword{open} \AgdaModule{Data-Vec}\<%
\\
\>[0]\AgdaIndent{2}{}\<[2]%
\>[2]\AgdaKeyword{open} \AgdaModule{≡-Prop}\<%
\\
%
\\
\>[0]\AgdaIndent{2}{}\<[2]%
\>[2]\AgdaFunction{Form} \AgdaSymbol{:} \AgdaDatatype{ℕ} \AgdaSymbol{→} \AgdaPrimitiveType{Set}\<%
\\
\>[0]\AgdaIndent{2}{}\<[2]%
\>[2]\AgdaFunction{Form} \AgdaSymbol{=} \AgdaDatatype{Vec} \AgdaDatatype{Bool} \<[18]%
\>[18]\<%
\\
%
\\
\>[0]\AgdaIndent{2}{}\<[2]%
\>[2]\AgdaKeyword{data} \AgdaDatatype{\_∥\_} \AgdaSymbol{:} \AgdaSymbol{∀} \AgdaSymbol{\{}\AgdaBound{n}\AgdaSymbol{\}} \AgdaSymbol{→} \AgdaFunction{Form} \AgdaBound{n} \AgdaSymbol{→} \AgdaFunction{Form} \AgdaBound{n} \AgdaSymbol{→} \AgdaPrimitiveType{Set} \AgdaKeyword{where}\<%
\\
\>[2]\AgdaIndent{4}{}\<[4]%
\>[4]\AgdaInductiveConstructor{[]∥[]} \AgdaSymbol{:} \AgdaInductiveConstructor{[]} \AgdaDatatype{∥} \AgdaInductiveConstructor{[]}\<%
\\
\>[2]\AgdaIndent{4}{}\<[4]%
\>[4]\AgdaInductiveConstructor{⊥∥⊥} \AgdaSymbol{:} \AgdaSymbol{∀} \AgdaSymbol{\{}\AgdaBound{n}\AgdaSymbol{\}\{}\AgdaBound{f₁} \AgdaBound{f₂} \AgdaSymbol{:} \AgdaFunction{Form} \AgdaBound{n}\AgdaSymbol{\}} \AgdaSymbol{→} \AgdaBound{f₁} \AgdaDatatype{∥} \AgdaBound{f₂} \AgdaSymbol{→} \AgdaSymbol{(}\AgdaInductiveConstructor{false} \AgdaInductiveConstructor{∷} \AgdaBound{f₁}\AgdaSymbol{)} \AgdaDatatype{∥} \AgdaSymbol{(}\AgdaInductiveConstructor{false} \AgdaInductiveConstructor{∷} \AgdaBound{f₂}\AgdaSymbol{)}\<%
\\
\>[2]\AgdaIndent{4}{}\<[4]%
\>[4]\AgdaInductiveConstructor{⊤∥⊥} \AgdaSymbol{:} \AgdaSymbol{∀} \AgdaSymbol{\{}\AgdaBound{n}\AgdaSymbol{\}\{}\AgdaBound{f₁} \AgdaBound{f₂} \AgdaSymbol{:} \AgdaFunction{Form} \AgdaBound{n}\AgdaSymbol{\}} \AgdaSymbol{→} \AgdaBound{f₁} \AgdaDatatype{∥} \AgdaBound{f₂} \AgdaSymbol{→} \AgdaSymbol{(}\AgdaInductiveConstructor{true} \AgdaInductiveConstructor{∷} \AgdaBound{f₁}\AgdaSymbol{)} \AgdaDatatype{∥} \AgdaSymbol{(}\AgdaInductiveConstructor{false} \AgdaInductiveConstructor{∷} \AgdaBound{f₂}\AgdaSymbol{)}\<%
\\
\>[2]\AgdaIndent{4}{}\<[4]%
\>[4]\AgdaInductiveConstructor{⊥∥⊤} \AgdaSymbol{:} \AgdaSymbol{∀} \AgdaSymbol{\{}\AgdaBound{n}\AgdaSymbol{\}\{}\AgdaBound{f₁} \AgdaBound{f₂} \AgdaSymbol{:} \AgdaFunction{Form} \AgdaBound{n}\AgdaSymbol{\}} \AgdaSymbol{→} \AgdaBound{f₁} \AgdaDatatype{∥} \AgdaBound{f₂} \AgdaSymbol{→} \AgdaSymbol{(}\AgdaInductiveConstructor{false} \AgdaInductiveConstructor{∷} \AgdaBound{f₁}\AgdaSymbol{)} \AgdaDatatype{∥} \AgdaSymbol{(}\AgdaInductiveConstructor{true} \AgdaInductiveConstructor{∷} \AgdaBound{f₂}\AgdaSymbol{)}\<%
\\
%
\\
\>[0]\AgdaIndent{2}{}\<[2]%
\>[2]\AgdaKeyword{data} \AgdaDatatype{Patch} \AgdaSymbol{:} \AgdaSymbol{∀} \AgdaSymbol{\{}\AgdaBound{n}\AgdaSymbol{\}} \AgdaSymbol{→} \AgdaFunction{Form} \AgdaBound{n} \AgdaSymbol{→} \AgdaPrimitiveType{Set} \AgdaKeyword{where}\<%
\\
\>[2]\AgdaIndent{4}{}\<[4]%
\>[4]\AgdaInductiveConstructor{O} \AgdaSymbol{:} \AgdaDatatype{Patch} \AgdaInductiveConstructor{[]}\<%
\\
\>[2]\AgdaIndent{4}{}\<[4]%
\>[4]\AgdaInductiveConstructor{⊥∷} \AgdaSymbol{:} \AgdaSymbol{∀} \AgdaSymbol{\{}\AgdaBound{n}\AgdaSymbol{\}\{}\AgdaBound{f} \AgdaSymbol{:} \AgdaFunction{Form} \AgdaBound{n}\AgdaSymbol{\}} \AgdaSymbol{→} \AgdaDatatype{Patch} \AgdaBound{f} \AgdaSymbol{→} \AgdaDatatype{Patch} \AgdaSymbol{(}\AgdaInductiveConstructor{false} \AgdaInductiveConstructor{∷} \AgdaBound{f}\AgdaSymbol{)}\<%
\\
\>[2]\AgdaIndent{4}{}\<[4]%
\>[4]\AgdaInductiveConstructor{⟨\_⇒\_⟩∷\_} \AgdaSymbol{:} \AgdaSymbol{∀} \AgdaSymbol{\{}\AgdaBound{n}\AgdaSymbol{\}\{}\AgdaBound{f} \AgdaSymbol{:} \AgdaFunction{Form} \AgdaBound{n}\AgdaSymbol{\}}\<%
\\
\>[4]\AgdaIndent{6}{}\<[6]%
\>[6]\AgdaSymbol{→} \AgdaSymbol{(}\AgdaBound{from} \AgdaBound{to} \AgdaSymbol{:} \AgdaBound{A}\AgdaSymbol{)}\<%
\\
\>[4]\AgdaIndent{6}{}\<[6]%
\>[6]\AgdaSymbol{→} \AgdaDatatype{Patch} \AgdaBound{f} \AgdaSymbol{→} \AgdaDatatype{Patch} \AgdaSymbol{(}\AgdaInductiveConstructor{true} \AgdaInductiveConstructor{∷} \AgdaBound{f}\AgdaSymbol{)}\<%
\\
%
\\
\>[0]\AgdaIndent{2}{}\<[2]%
\>[2]\AgdaKeyword{data} \AgdaDatatype{\_⊂\_} \AgdaSymbol{:} \AgdaSymbol{∀} \AgdaSymbol{\{}\AgdaBound{n}\AgdaSymbol{\}} \AgdaSymbol{→} \AgdaFunction{Form} \AgdaBound{n} \AgdaSymbol{→} \AgdaFunction{Form} \AgdaBound{n} \AgdaSymbol{→} \AgdaPrimitiveType{Set} \AgdaKeyword{where}\<%
\\
\>[2]\AgdaIndent{4}{}\<[4]%
\>[4]\AgdaInductiveConstructor{[]⊂[]} \AgdaSymbol{:} \AgdaInductiveConstructor{[]} \AgdaDatatype{⊂} \AgdaInductiveConstructor{[]}\<%
\\
\>[2]\AgdaIndent{4}{}\<[4]%
\>[4]\AgdaInductiveConstructor{⊂⊤} \AgdaSymbol{:} \AgdaSymbol{∀} \AgdaSymbol{\{}\AgdaBound{n}\AgdaSymbol{\}\{}\AgdaBound{f₁} \AgdaBound{f₂} \AgdaSymbol{:} \AgdaFunction{Form} \AgdaBound{n}\AgdaSymbol{\}} \AgdaSymbol{→} \AgdaSymbol{(}\AgdaBound{b} \AgdaSymbol{:} \AgdaDatatype{Bool}\AgdaSymbol{)} \AgdaSymbol{→} \AgdaSymbol{(}\AgdaBound{b} \AgdaInductiveConstructor{∷} \AgdaBound{f₁}\AgdaSymbol{)} \AgdaDatatype{⊂} \AgdaSymbol{(}\AgdaInductiveConstructor{true} \AgdaInductiveConstructor{∷} \AgdaBound{f₂}\AgdaSymbol{)}\<%
\\
\>[2]\AgdaIndent{4}{}\<[4]%
\>[4]\AgdaInductiveConstructor{⊥⊂⊥} \AgdaSymbol{:} \AgdaSymbol{∀} \AgdaSymbol{\{}\AgdaBound{n}\AgdaSymbol{\}\{}\AgdaBound{f₁} \AgdaBound{f₂} \AgdaSymbol{:} \AgdaFunction{Form} \AgdaBound{n}\AgdaSymbol{\}} \AgdaSymbol{→} \AgdaSymbol{(}\AgdaInductiveConstructor{false} \AgdaInductiveConstructor{∷} \AgdaBound{f₁}\AgdaSymbol{)} \AgdaDatatype{⊂} \AgdaSymbol{(}\AgdaInductiveConstructor{false} \AgdaInductiveConstructor{∷} \AgdaBound{f₂}\AgdaSymbol{)}\<%
\\
%
\\
\>[0]\AgdaIndent{2}{}\<[2]%
\>[2]\AgdaKeyword{data} \AgdaDatatype{\_⊏\_} \AgdaSymbol{:} \AgdaSymbol{∀} \AgdaSymbol{\{}\AgdaBound{n}\AgdaSymbol{\}\{}\AgdaBound{f} \AgdaSymbol{:} \AgdaFunction{Form} \AgdaBound{n}\AgdaSymbol{\}} \AgdaSymbol{→} \AgdaDatatype{Patch} \AgdaBound{f} \AgdaSymbol{→} \AgdaDatatype{Vec} \AgdaBound{A} \AgdaBound{n} \AgdaSymbol{→} \AgdaPrimitiveType{Set} \AgdaKeyword{where}\<%
\\
\>[2]\AgdaIndent{4}{}\<[4]%
\>[4]\AgdaInductiveConstructor{[]⊏[]} \AgdaSymbol{:} \AgdaInductiveConstructor{O} \AgdaDatatype{⊏} \AgdaInductiveConstructor{[]}\<%
\\
\>[2]\AgdaIndent{4}{}\<[4]%
\>[4]\AgdaInductiveConstructor{⊥⊏} \AgdaSymbol{:} \AgdaSymbol{∀} \AgdaSymbol{\{}\AgdaBound{n}\AgdaSymbol{\}\{}\AgdaBound{f} \AgdaSymbol{:} \AgdaFunction{Form} \AgdaBound{n}\AgdaSymbol{\}\{}\AgdaBound{p} \AgdaSymbol{:} \AgdaDatatype{Patch} \AgdaBound{f}\AgdaSymbol{\}\{}\AgdaBound{v} \AgdaSymbol{:} \AgdaDatatype{Vec} \AgdaBound{A} \AgdaBound{n}\AgdaSymbol{\}}\<%
\\
\>[4]\AgdaIndent{6}{}\<[6]%
\>[6]\AgdaSymbol{→} \AgdaSymbol{(}\AgdaBound{a} \AgdaSymbol{:} \AgdaBound{A}\AgdaSymbol{)} \AgdaSymbol{→} \AgdaBound{p} \AgdaDatatype{⊏} \AgdaBound{v} \AgdaSymbol{→} \AgdaSymbol{(}\AgdaInductiveConstructor{⊥∷} \AgdaBound{p}\AgdaSymbol{)} \AgdaDatatype{⊏} \AgdaSymbol{(}\AgdaBound{a} \AgdaInductiveConstructor{∷} \AgdaBound{v}\AgdaSymbol{)}\<%
\\
\>[0]\AgdaIndent{4}{}\<[4]%
\>[4]\AgdaInductiveConstructor{⊤⊏} \AgdaSymbol{:} \AgdaSymbol{∀} \AgdaSymbol{\{}\AgdaBound{n}\AgdaSymbol{\}\{}\AgdaBound{f} \AgdaSymbol{:} \AgdaFunction{Form} \AgdaBound{n}\AgdaSymbol{\}\{}\AgdaBound{p} \AgdaSymbol{:} \AgdaDatatype{Patch} \AgdaBound{f}\AgdaSymbol{\}\{}\AgdaBound{v} \AgdaSymbol{:} \AgdaDatatype{Vec} \AgdaBound{A} \AgdaBound{n}\AgdaSymbol{\}}\<%
\\
\>[4]\AgdaIndent{6}{}\<[6]%
\>[6]\AgdaSymbol{→} \AgdaSymbol{(}\AgdaBound{a} \AgdaBound{b} \AgdaSymbol{:} \AgdaBound{A}\AgdaSymbol{)}\<%
\\
\>[4]\AgdaIndent{6}{}\<[6]%
\>[6]\AgdaSymbol{→} \AgdaBound{p} \AgdaDatatype{⊏} \AgdaBound{v} \AgdaSymbol{→} \AgdaSymbol{(}\AgdaInductiveConstructor{⟨} \AgdaBound{a} \AgdaInductiveConstructor{⇒} \AgdaBound{b} \AgdaInductiveConstructor{⟩∷} \AgdaBound{p}\AgdaSymbol{)} \AgdaDatatype{⊏} \AgdaSymbol{(}\AgdaBound{a} \AgdaInductiveConstructor{∷} \AgdaBound{v}\AgdaSymbol{)}\<%
\\
%
\\
\>[0]\AgdaIndent{2}{}\<[2]%
\>[2]\AgdaFunction{patch} \AgdaSymbol{:} \AgdaSymbol{∀} \AgdaSymbol{\{}\AgdaBound{n}\AgdaSymbol{\}\{}\AgdaBound{f} \AgdaSymbol{:} \AgdaFunction{Form} \AgdaBound{n}\AgdaSymbol{\}} \AgdaSymbol{→} \AgdaSymbol{(}\AgdaBound{p} \AgdaSymbol{:} \AgdaDatatype{Patch} \AgdaBound{f}\AgdaSymbol{)} \AgdaSymbol{→} \AgdaSymbol{(}\AgdaBound{x} \AgdaSymbol{:} \AgdaDatatype{Vec} \AgdaBound{A} \AgdaBound{n}\AgdaSymbol{)} \AgdaSymbol{→} \AgdaBound{p} \AgdaDatatype{⊏} \AgdaBound{x} \AgdaSymbol{→} \AgdaDatatype{Vec} \AgdaBound{A} \AgdaBound{n}\<%
\\
\>[0]\AgdaIndent{2}{}\<[2]%
\>[2]\AgdaFunction{patch} \AgdaInductiveConstructor{O} \AgdaInductiveConstructor{[]} \AgdaInductiveConstructor{[]⊏[]} \AgdaSymbol{=} \AgdaInductiveConstructor{[]}\<%
\\
\>[0]\AgdaIndent{2}{}\<[2]%
\>[2]\AgdaFunction{patch} \AgdaSymbol{(}\AgdaInductiveConstructor{⊥∷} \AgdaBound{p}\AgdaSymbol{)} \AgdaSymbol{(}\AgdaBound{x} \AgdaInductiveConstructor{∷} \AgdaBound{xs}\AgdaSymbol{)} \AgdaSymbol{(}\AgdaInductiveConstructor{⊥⊏} \AgdaSymbol{.}\AgdaBound{x} \AgdaBound{p-xs}\AgdaSymbol{)} \AgdaSymbol{=} \AgdaBound{x} \AgdaInductiveConstructor{∷} \AgdaFunction{patch} \AgdaBound{p} \AgdaBound{xs} \AgdaBound{p-xs}\<%
\\
\>[0]\AgdaIndent{2}{}\<[2]%
\>[2]\AgdaFunction{patch} \AgdaSymbol{(}\AgdaInductiveConstructor{⟨} \AgdaSymbol{.}\AgdaBound{f} \AgdaInductiveConstructor{⇒} \AgdaBound{t} \AgdaInductiveConstructor{⟩∷} \AgdaBound{p}\AgdaSymbol{)} \AgdaSymbol{(}\AgdaBound{f} \AgdaInductiveConstructor{∷} \AgdaBound{xs}\AgdaSymbol{)} \AgdaSymbol{(}\AgdaInductiveConstructor{⊤⊏} \AgdaSymbol{.}\AgdaBound{f} \AgdaSymbol{.}\AgdaBound{t} \AgdaBound{p-xs}\AgdaSymbol{)} \AgdaSymbol{=} \<[51]%
\>[51]\<%
\\
\>[2]\AgdaIndent{4}{}\<[4]%
\>[4]\AgdaBound{t} \AgdaInductiveConstructor{∷} \AgdaFunction{patch} \AgdaBound{p} \AgdaBound{xs} \AgdaBound{p-xs}\<%
\\
%
\\
\>[0]\AgdaIndent{2}{}\<[2]%
\>[2]\AgdaFunction{\_⟶\_} \AgdaSymbol{:} \AgdaSymbol{∀} \AgdaSymbol{\{}\AgdaBound{n}\AgdaSymbol{\}\{}\AgdaBound{f₁} \AgdaBound{f₂} \AgdaSymbol{:} \AgdaFunction{Form} \AgdaBound{n}\AgdaSymbol{\}}\<%
\\
\>[2]\AgdaIndent{4}{}\<[4]%
\>[4]\AgdaSymbol{→} \AgdaSymbol{(}\AgdaBound{p₁} \AgdaSymbol{:} \AgdaDatatype{Patch} \AgdaBound{f₁}\AgdaSymbol{)} \AgdaSymbol{→} \AgdaSymbol{(}\AgdaBound{p₂} \AgdaSymbol{:} \AgdaDatatype{Patch} \AgdaBound{f₂}\AgdaSymbol{)} \AgdaSymbol{→} \AgdaPrimitiveType{Set}\<%
\\
\>[0]\AgdaIndent{2}{}\<[2]%
\>[2]\AgdaFunction{\_⟶\_} \AgdaSymbol{\{}\AgdaBound{n}\AgdaSymbol{\}} \AgdaBound{p₁} \AgdaBound{p₂} \AgdaSymbol{=} \AgdaSymbol{∀} \AgdaSymbol{(}\AgdaBound{x} \AgdaSymbol{:} \AgdaDatatype{Vec} \AgdaBound{A} \AgdaBound{n}\AgdaSymbol{)} \<[34]%
\>[34]\<%
\\
\>[2]\AgdaIndent{4}{}\<[4]%
\>[4]\AgdaSymbol{→} \AgdaSymbol{(}\AgdaBound{p₁-x} \AgdaSymbol{:} \AgdaBound{p₁} \AgdaDatatype{⊏} \AgdaBound{x}\AgdaSymbol{)} \AgdaSymbol{→} \AgdaRecord{Σ} \AgdaSymbol{(}\AgdaBound{p₂} \AgdaDatatype{⊏} \AgdaBound{x}\AgdaSymbol{)} \AgdaSymbol{(λ} \AgdaBound{p₂-x} \AgdaSymbol{→} \<[45]%
\>[45]\<%
\\
\>[4]\AgdaIndent{6}{}\<[6]%
\>[6]\AgdaSymbol{(}\AgdaFunction{patch} \AgdaBound{p₁} \AgdaBound{x} \AgdaBound{p₁-x} \AgdaDatatype{≡} \AgdaFunction{patch} \AgdaBound{p₂} \AgdaBound{x} \AgdaBound{p₂-x}\AgdaSymbol{))}\<%
\\
%
\\
\>[0]\AgdaIndent{2}{}\<[2]%
\>[2]\AgdaFunction{\_⟷\_} \AgdaSymbol{:} \AgdaSymbol{∀} \AgdaSymbol{\{}\AgdaBound{n}\AgdaSymbol{\}\{}\AgdaBound{f₁} \AgdaBound{f₂} \AgdaSymbol{:} \AgdaFunction{Form} \AgdaBound{n}\AgdaSymbol{\}}\<%
\\
\>[2]\AgdaIndent{4}{}\<[4]%
\>[4]\AgdaSymbol{→} \AgdaSymbol{(}\AgdaBound{p₁} \AgdaSymbol{:} \AgdaDatatype{Patch} \AgdaBound{f₁}\AgdaSymbol{)} \AgdaSymbol{→} \AgdaSymbol{(}\AgdaBound{p₂} \AgdaSymbol{:} \AgdaDatatype{Patch} \AgdaBound{f₂}\AgdaSymbol{)} \AgdaSymbol{→} \AgdaPrimitiveType{Set}\<%
\\
\>[0]\AgdaIndent{2}{}\<[2]%
\>[2]\AgdaBound{p₁} \AgdaFunction{⟷} \AgdaBound{p₂} \AgdaSymbol{=} \AgdaSymbol{(}\AgdaBound{p₁} \AgdaFunction{⟶} \AgdaBound{p₂}\AgdaSymbol{)} \AgdaRecord{∧} \AgdaSymbol{(}\AgdaBound{p₂} \AgdaFunction{⟶} \AgdaBound{p₁}\AgdaSymbol{)}\<%
\\
%
\\
\>[0]\AgdaIndent{2}{}\<[2]%
\>[2]\AgdaKeyword{module} \AgdaModule{⟷-equiv} \AgdaKeyword{where}\<%
\\
\>[2]\AgdaIndent{4}{}\<[4]%
\>[4]\AgdaFunction{⟷-refl} \AgdaSymbol{:} \AgdaSymbol{∀} \AgdaSymbol{\{}\AgdaBound{n}\AgdaSymbol{\}\{}\AgdaBound{f} \AgdaSymbol{:} \AgdaFunction{Form} \AgdaBound{n}\AgdaSymbol{\}} \AgdaSymbol{→} \AgdaSymbol{(}\AgdaBound{p} \AgdaSymbol{:} \AgdaDatatype{Patch} \AgdaBound{f}\AgdaSymbol{)}\<%
\\
\>[4]\AgdaIndent{6}{}\<[6]%
\>[6]\AgdaSymbol{→} \AgdaBound{p} \AgdaFunction{⟷} \AgdaBound{p}\<%
\\
\>[0]\AgdaIndent{4}{}\<[4]%
\>[4]\AgdaFunction{⟷-refl} \AgdaBound{p} \AgdaSymbol{=} \AgdaSymbol{(λ} \AgdaBound{x} \AgdaBound{x₁} \AgdaSymbol{→} \AgdaBound{x₁} \AgdaInductiveConstructor{,} \AgdaInductiveConstructor{refl}\AgdaSymbol{)} \AgdaInductiveConstructor{,} \AgdaSymbol{(λ} \AgdaBound{x} \AgdaBound{x₁} \AgdaSymbol{→} \AgdaBound{x₁} \AgdaInductiveConstructor{,} \AgdaInductiveConstructor{refl}\AgdaSymbol{)}\<%
\\
%
\\
\>[0]\AgdaIndent{4}{}\<[4]%
\>[4]\AgdaFunction{⟶-trans} \AgdaSymbol{:} \AgdaSymbol{∀} \AgdaSymbol{\{}\AgdaBound{n}\AgdaSymbol{\}\{}\AgdaBound{f₁} \AgdaBound{f₂} \AgdaBound{f₃} \AgdaSymbol{:} \AgdaFunction{Form} \AgdaBound{n}\AgdaSymbol{\}}\<%
\\
\>[4]\AgdaIndent{6}{}\<[6]%
\>[6]\AgdaSymbol{→} \AgdaSymbol{\{}\AgdaBound{p₁} \AgdaSymbol{:} \AgdaDatatype{Patch} \AgdaBound{f₁}\AgdaSymbol{\}\{}\AgdaBound{p₂} \AgdaSymbol{:} \AgdaDatatype{Patch} \AgdaBound{f₂}\AgdaSymbol{\}\{}\AgdaBound{p₃} \AgdaSymbol{:} \AgdaDatatype{Patch} \AgdaBound{f₃}\AgdaSymbol{\}}\<%
\\
\>[4]\AgdaIndent{6}{}\<[6]%
\>[6]\AgdaSymbol{→} \AgdaSymbol{(}\AgdaBound{p₁} \AgdaFunction{⟶} \AgdaBound{p₂}\AgdaSymbol{)} \AgdaSymbol{→} \AgdaSymbol{(}\AgdaBound{p₂} \AgdaFunction{⟶} \AgdaBound{p₃}\AgdaSymbol{)} \AgdaSymbol{→} \AgdaSymbol{(}\AgdaBound{p₁} \AgdaFunction{⟶} \AgdaBound{p₃}\AgdaSymbol{)}\<%
\\
\>[0]\AgdaIndent{4}{}\<[4]%
\>[4]\AgdaFunction{⟶-trans} \AgdaSymbol{\{p₁} \AgdaSymbol{=} \AgdaBound{p₁}\AgdaSymbol{\}\{}\AgdaBound{p₂}\AgdaSymbol{\}\{}\AgdaBound{p₃}\AgdaSymbol{\}} \AgdaBound{p₁⟶p₂} \AgdaBound{p₂⟶p₃} \AgdaBound{x} \AgdaBound{p₁⊏x} \<[49]%
\>[49]\<%
\\
\>[4]\AgdaIndent{6}{}\<[6]%
\>[6]\AgdaKeyword{with} \AgdaFunction{patch} \AgdaBound{p₁} \AgdaBound{x} \AgdaBound{p₁⊏x} \AgdaSymbol{|} \AgdaBound{p₁⟶p₂} \AgdaBound{x} \AgdaBound{p₁⊏x}\<%
\\
\>[0]\AgdaIndent{4}{}\<[4]%
\>[4]\AgdaSymbol{...} \AgdaSymbol{|} \AgdaSymbol{.(}\AgdaFunction{patch} \AgdaBound{p₂} \AgdaBound{x} \AgdaBound{p₂⊏x}\AgdaSymbol{)} \AgdaSymbol{|} \AgdaBound{p₂⊏x} \AgdaInductiveConstructor{,} \AgdaInductiveConstructor{refl} \AgdaSymbol{=} \AgdaBound{p₂⟶p₃} \AgdaBound{x} \AgdaBound{p₂⊏x}\<%
\\
\>[0]\AgdaIndent{2}{}\<[2]%
\>[2]\<%
\\
\>[2]\AgdaIndent{4}{}\<[4]%
\>[4]\AgdaFunction{⟷-trans} \AgdaSymbol{:} \AgdaSymbol{∀} \AgdaSymbol{\{}\AgdaBound{n}\AgdaSymbol{\}\{}\AgdaBound{f₁} \AgdaBound{f₂} \AgdaBound{f₃} \AgdaSymbol{:} \AgdaFunction{Form} \AgdaBound{n}\AgdaSymbol{\}}\<%
\\
\>[4]\AgdaIndent{6}{}\<[6]%
\>[6]\AgdaSymbol{→} \AgdaSymbol{\{}\AgdaBound{p₁} \AgdaSymbol{:} \AgdaDatatype{Patch} \AgdaBound{f₁}\AgdaSymbol{\}\{}\AgdaBound{p₂} \AgdaSymbol{:} \AgdaDatatype{Patch} \AgdaBound{f₂}\AgdaSymbol{\}\{}\AgdaBound{p₃} \AgdaSymbol{:} \AgdaDatatype{Patch} \AgdaBound{f₃}\AgdaSymbol{\}}\<%
\\
\>[4]\AgdaIndent{6}{}\<[6]%
\>[6]\AgdaSymbol{→} \AgdaSymbol{(}\AgdaBound{p₁} \AgdaFunction{⟷} \AgdaBound{p₂}\AgdaSymbol{)} \AgdaSymbol{→} \AgdaSymbol{(}\AgdaBound{p₂} \AgdaFunction{⟷} \AgdaBound{p₃}\AgdaSymbol{)} \AgdaSymbol{→} \AgdaSymbol{(}\AgdaBound{p₁} \AgdaFunction{⟷} \AgdaBound{p₃}\AgdaSymbol{)}\<%
\\
\>[0]\AgdaIndent{4}{}\<[4]%
\>[4]\AgdaFunction{⟷-trans} \AgdaSymbol{(}\AgdaBound{p₁⟶p₂} \AgdaInductiveConstructor{,} \AgdaBound{p₂⟶p₁}\AgdaSymbol{)} \AgdaSymbol{(}\AgdaBound{p₂⟶p₃} \AgdaInductiveConstructor{,} \AgdaBound{p₃⟶p₂}\AgdaSymbol{)} \AgdaSymbol{=} \<[46]%
\>[46]\<%
\\
\>[4]\AgdaIndent{6}{}\<[6]%
\>[6]\AgdaSymbol{(}\AgdaFunction{⟶-trans} \AgdaBound{p₁⟶p₂} \AgdaBound{p₂⟶p₃}\AgdaSymbol{)} \AgdaInductiveConstructor{,} \AgdaSymbol{(}\AgdaFunction{⟶-trans} \AgdaBound{p₃⟶p₂} \AgdaBound{p₂⟶p₁}\AgdaSymbol{)}\<%
\\
\>[4]\AgdaIndent{6}{}\<[6]%
\>[6]\<%
\\
\>[0]\AgdaIndent{4}{}\<[4]%
\>[4]\AgdaFunction{⟷-symm} \AgdaSymbol{:} \AgdaSymbol{∀} \AgdaSymbol{\{}\AgdaBound{n}\AgdaSymbol{\}\{}\AgdaBound{f₁} \AgdaBound{f₂} \AgdaSymbol{:} \AgdaFunction{Form} \AgdaBound{n}\AgdaSymbol{\}}\<%
\\
\>[4]\AgdaIndent{6}{}\<[6]%
\>[6]\AgdaSymbol{→} \AgdaSymbol{\{}\AgdaBound{p₁} \AgdaSymbol{:} \AgdaDatatype{Patch} \AgdaBound{f₁}\AgdaSymbol{\}} \AgdaSymbol{\{}\AgdaBound{p₂} \AgdaSymbol{:} \AgdaDatatype{Patch} \AgdaBound{f₂}\AgdaSymbol{\}}\<%
\\
\>[4]\AgdaIndent{6}{}\<[6]%
\>[6]\AgdaSymbol{→} \AgdaSymbol{(}\AgdaBound{p₁} \AgdaFunction{⟷} \AgdaBound{p₂}\AgdaSymbol{)} \AgdaSymbol{→} \AgdaSymbol{(}\AgdaBound{p₂} \AgdaFunction{⟷} \AgdaBound{p₁}\AgdaSymbol{)}\<%
\\
\>[0]\AgdaIndent{4}{}\<[4]%
\>[4]\AgdaFunction{⟷-symm} \AgdaBound{p₁⟷p₂} \AgdaSymbol{=} \AgdaField{snd} \AgdaBound{p₁⟷p₂} \AgdaInductiveConstructor{,} \AgdaField{fst} \AgdaBound{p₁⟷p₂}\<%
\\
%
\\
\>[0]\AgdaIndent{2}{}\<[2]%
\>[2]\AgdaKeyword{open} \AgdaModule{⟷-equiv}\<%
\\
%
\\
\>[0]\AgdaIndent{2}{}\<[2]%
\>[2]\AgdaFunction{\_∧ₛ\_} \AgdaSymbol{:} \AgdaSymbol{∀} \AgdaSymbol{\{}\AgdaBound{n}\AgdaSymbol{\}} \AgdaSymbol{(}\AgdaBound{f₁} \AgdaBound{f₂} \AgdaSymbol{:} \AgdaFunction{Form} \AgdaBound{n}\AgdaSymbol{)} \AgdaSymbol{→} \AgdaBound{f₁} \AgdaDatatype{∥} \AgdaBound{f₂} \AgdaSymbol{→} \AgdaFunction{Form} \AgdaBound{n}\<%
\\
\>[0]\AgdaIndent{2}{}\<[2]%
\>[2]\AgdaFunction{\_∧ₛ\_} \AgdaInductiveConstructor{[]} \AgdaInductiveConstructor{[]} \AgdaInductiveConstructor{[]∥[]} \AgdaSymbol{=} \AgdaInductiveConstructor{[]}\<%
\\
\>[0]\AgdaIndent{2}{}\<[2]%
\>[2]\AgdaFunction{\_∧ₛ\_} \AgdaSymbol{(}\AgdaSymbol{.}\AgdaInductiveConstructor{false} \AgdaInductiveConstructor{∷} \AgdaBound{xs}\AgdaSymbol{)} \AgdaSymbol{(}\AgdaSymbol{.}\AgdaInductiveConstructor{false} \AgdaInductiveConstructor{∷} \AgdaBound{ys}\AgdaSymbol{)} \AgdaSymbol{(}\AgdaInductiveConstructor{⊥∥⊥} \AgdaBound{p}\AgdaSymbol{)} \AgdaSymbol{=} \AgdaInductiveConstructor{false} \AgdaInductiveConstructor{∷} \AgdaSymbol{(}\AgdaBound{xs} \AgdaFunction{∧ₛ} \AgdaBound{ys}\AgdaSymbol{)} \AgdaBound{p}\<%
\\
\>[0]\AgdaIndent{2}{}\<[2]%
\>[2]\AgdaFunction{\_∧ₛ\_} \AgdaSymbol{(}\AgdaSymbol{.}\AgdaInductiveConstructor{true} \AgdaInductiveConstructor{∷} \AgdaBound{xs}\AgdaSymbol{)} \AgdaSymbol{(}\AgdaSymbol{.}\AgdaInductiveConstructor{false} \AgdaInductiveConstructor{∷} \AgdaBound{ys}\AgdaSymbol{)} \AgdaSymbol{(}\AgdaInductiveConstructor{⊤∥⊥} \AgdaBound{p}\AgdaSymbol{)} \AgdaSymbol{=} \AgdaInductiveConstructor{true} \AgdaInductiveConstructor{∷} \AgdaSymbol{(}\AgdaBound{xs} \AgdaFunction{∧ₛ} \AgdaBound{ys}\AgdaSymbol{)} \AgdaBound{p}\<%
\\
\>[0]\AgdaIndent{2}{}\<[2]%
\>[2]\AgdaFunction{\_∧ₛ\_} \AgdaSymbol{(}\AgdaSymbol{.}\AgdaInductiveConstructor{false} \AgdaInductiveConstructor{∷} \AgdaBound{xs}\AgdaSymbol{)} \AgdaSymbol{(}\AgdaSymbol{.}\AgdaInductiveConstructor{true} \AgdaInductiveConstructor{∷} \AgdaBound{ys}\AgdaSymbol{)} \AgdaSymbol{(}\AgdaInductiveConstructor{⊥∥⊤} \AgdaBound{p}\AgdaSymbol{)} \AgdaSymbol{=} \AgdaInductiveConstructor{true} \AgdaInductiveConstructor{∷} \AgdaSymbol{(}\AgdaBound{xs} \AgdaFunction{∧ₛ} \AgdaBound{ys}\AgdaSymbol{)} \AgdaBound{p}\<%
\\
%
\\
\>[0]\AgdaIndent{2}{}\<[2]%
\>[2]\AgdaFunction{unite} \AgdaSymbol{:} \AgdaSymbol{∀} \AgdaSymbol{\{}\AgdaBound{n}\AgdaSymbol{\}} \AgdaSymbol{\{}\AgdaBound{f₁} \AgdaBound{f₂} \AgdaSymbol{:} \AgdaFunction{Form} \AgdaBound{n}\AgdaSymbol{\}} \AgdaSymbol{→} \AgdaBound{f₁} \AgdaDatatype{∥} \AgdaBound{f₂} \AgdaSymbol{→} \AgdaFunction{Form} \AgdaBound{n}\<%
\\
\>[0]\AgdaIndent{2}{}\<[2]%
\>[2]\AgdaFunction{unite} \AgdaSymbol{\{}\AgdaBound{n}\AgdaSymbol{\}} \AgdaSymbol{\{}\AgdaBound{f₁}\AgdaSymbol{\}} \AgdaSymbol{\{}\AgdaBound{f₂}\AgdaSymbol{\}} \AgdaBound{p} \AgdaSymbol{=} \AgdaSymbol{(}\AgdaBound{f₁} \AgdaFunction{∧ₛ} \AgdaBound{f₂}\AgdaSymbol{)} \AgdaBound{p}\<%
\\
%
\\
\>[0]\AgdaIndent{2}{}\<[2]%
\>[2]\AgdaFunction{\_∧ₚ\_} \AgdaSymbol{:} \AgdaSymbol{∀} \AgdaSymbol{\{}\AgdaBound{n}\AgdaSymbol{\}} \AgdaSymbol{\{}\AgdaBound{f₁} \AgdaBound{f₂} \AgdaSymbol{:} \AgdaFunction{Form} \AgdaBound{n}\AgdaSymbol{\}} \AgdaSymbol{(}\AgdaBound{p₁} \AgdaSymbol{:} \AgdaDatatype{Patch} \AgdaBound{f₁}\AgdaSymbol{)} \AgdaSymbol{(}\AgdaBound{p₂} \AgdaSymbol{:} \AgdaDatatype{Patch} \AgdaBound{f₂}\AgdaSymbol{)}\<%
\\
\>[2]\AgdaIndent{4}{}\<[4]%
\>[4]\AgdaSymbol{→} \AgdaSymbol{(}\AgdaBound{f₁∥f₂} \AgdaSymbol{:} \AgdaBound{f₁} \AgdaDatatype{∥} \AgdaBound{f₂}\AgdaSymbol{)} \AgdaSymbol{→} \AgdaDatatype{Patch} \AgdaSymbol{(}\AgdaFunction{unite} \AgdaBound{f₁∥f₂}\AgdaSymbol{)} \<[49]%
\>[49]\<%
\\
\>[0]\AgdaIndent{2}{}\<[2]%
\>[2]\AgdaFunction{\_∧ₚ\_} \AgdaInductiveConstructor{O} \AgdaInductiveConstructor{O} \AgdaInductiveConstructor{[]∥[]} \AgdaSymbol{=} \AgdaInductiveConstructor{O}\<%
\\
\>[0]\AgdaIndent{2}{}\<[2]%
\>[2]\AgdaFunction{\_∧ₚ\_} \AgdaSymbol{(}\AgdaInductiveConstructor{⊥∷} \AgdaBound{p₁}\AgdaSymbol{)} \AgdaSymbol{(}\AgdaInductiveConstructor{⊥∷} \AgdaBound{p₂}\AgdaSymbol{)} \AgdaSymbol{(}\AgdaInductiveConstructor{⊥∥⊥} \AgdaBound{f₁∥f₂}\AgdaSymbol{)} \AgdaSymbol{=} \AgdaInductiveConstructor{⊥∷} \AgdaSymbol{((}\AgdaBound{p₁} \AgdaFunction{∧ₚ} \AgdaBound{p₂}\AgdaSymbol{)} \AgdaBound{f₁∥f₂}\AgdaSymbol{)}\<%
\\
\>[0]\AgdaIndent{2}{}\<[2]%
\>[2]\AgdaFunction{\_∧ₚ\_} \AgdaSymbol{(}\AgdaInductiveConstructor{⊥∷} \AgdaBound{p₁}\AgdaSymbol{)} \AgdaSymbol{(}\AgdaInductiveConstructor{⟨} \AgdaBound{from} \AgdaInductiveConstructor{⇒} \AgdaBound{to} \AgdaInductiveConstructor{⟩∷} \AgdaBound{p₂}\AgdaSymbol{)} \AgdaSymbol{(}\AgdaInductiveConstructor{⊥∥⊤} \AgdaBound{f₁∥f₂}\AgdaSymbol{)} \AgdaSymbol{=} \<[49]%
\>[49]\<%
\\
\>[2]\AgdaIndent{4}{}\<[4]%
\>[4]\AgdaInductiveConstructor{⟨} \AgdaBound{from} \AgdaInductiveConstructor{⇒} \AgdaBound{to} \AgdaInductiveConstructor{⟩∷} \AgdaSymbol{(}\AgdaBound{p₁} \AgdaFunction{∧ₚ} \AgdaBound{p₂}\AgdaSymbol{)} \AgdaBound{f₁∥f₂}\<%
\\
\>[0]\AgdaIndent{2}{}\<[2]%
\>[2]\AgdaFunction{\_∧ₚ\_} \AgdaSymbol{(}\AgdaInductiveConstructor{⟨} \AgdaBound{from} \AgdaInductiveConstructor{⇒} \AgdaBound{to} \AgdaInductiveConstructor{⟩∷} \AgdaBound{p₁}\AgdaSymbol{)} \AgdaSymbol{(}\AgdaInductiveConstructor{⊥∷} \AgdaBound{p₂}\AgdaSymbol{)} \AgdaSymbol{(}\AgdaInductiveConstructor{⊤∥⊥} \AgdaBound{f₁∥f₂}\AgdaSymbol{)} \AgdaSymbol{=} \<[49]%
\>[49]\<%
\\
\>[2]\AgdaIndent{4}{}\<[4]%
\>[4]\AgdaInductiveConstructor{⟨} \AgdaBound{from} \AgdaInductiveConstructor{⇒} \AgdaBound{to} \AgdaInductiveConstructor{⟩∷} \AgdaSymbol{(}\AgdaBound{p₁} \AgdaFunction{∧ₚ} \AgdaBound{p₂}\AgdaSymbol{)} \AgdaBound{f₁∥f₂}\<%
\\
%
\\
\>[0]\AgdaIndent{2}{}\<[2]%
\>[2]\AgdaFunction{⟶-prepend-⊥⊥} \AgdaSymbol{:} \AgdaSymbol{∀} \AgdaSymbol{\{}\AgdaBound{n}\AgdaSymbol{\}\{}\AgdaBound{f₁} \AgdaBound{f₂} \AgdaSymbol{:} \AgdaFunction{Form} \AgdaBound{n}\AgdaSymbol{\}}\<%
\\
\>[2]\AgdaIndent{4}{}\<[4]%
\>[4]\AgdaSymbol{→} \AgdaSymbol{\{}\AgdaBound{p₁} \AgdaSymbol{:} \AgdaDatatype{Patch} \AgdaBound{f₁}\AgdaSymbol{\}} \AgdaSymbol{→} \AgdaSymbol{\{}\AgdaBound{p₂} \AgdaSymbol{:} \AgdaDatatype{Patch} \AgdaBound{f₂}\AgdaSymbol{\}}\<%
\\
\>[2]\AgdaIndent{4}{}\<[4]%
\>[4]\AgdaSymbol{→} \AgdaSymbol{(}\AgdaBound{p₁} \AgdaFunction{⟶} \AgdaBound{p₂}\AgdaSymbol{)} \AgdaSymbol{→} \AgdaSymbol{(}\AgdaInductiveConstructor{⊥∷} \AgdaBound{p₁} \AgdaFunction{⟶} \AgdaInductiveConstructor{⊥∷} \AgdaBound{p₂}\AgdaSymbol{)}\<%
\\
\>[0]\AgdaIndent{2}{}\<[2]%
\>[2]\AgdaFunction{⟶-prepend-⊥⊥} \AgdaSymbol{\{p₁} \AgdaSymbol{=} \AgdaBound{p₁}\AgdaSymbol{\}\{}\AgdaBound{p₂}\AgdaSymbol{\}} \AgdaBound{p₁⟶p₂} \AgdaSymbol{(}\AgdaBound{x} \AgdaInductiveConstructor{∷} \AgdaBound{xs}\AgdaSymbol{)} \AgdaSymbol{(}\AgdaInductiveConstructor{⊥⊏} \AgdaSymbol{.}\AgdaBound{x} \AgdaBound{p₁⊏xs}\AgdaSymbol{)} \<[58]%
\>[58]\<%
\\
\>[2]\AgdaIndent{4}{}\<[4]%
\>[4]\AgdaKeyword{with} \AgdaFunction{patch} \AgdaBound{p₁} \AgdaBound{xs} \AgdaBound{p₁⊏xs} \AgdaSymbol{|} \AgdaBound{p₁⟶p₂} \AgdaBound{xs} \AgdaBound{p₁⊏xs}\<%
\\
\>[0]\AgdaIndent{2}{}\<[2]%
\>[2]\AgdaSymbol{...} \AgdaSymbol{|} \AgdaSymbol{.(}\AgdaFunction{patch} \AgdaBound{p₂} \AgdaBound{xs} \AgdaBound{p₂⊏xs}\AgdaSymbol{)} \AgdaSymbol{|} \AgdaBound{p₂⊏xs} \AgdaInductiveConstructor{,} \AgdaInductiveConstructor{refl} \AgdaSymbol{=} \AgdaSymbol{(}\AgdaInductiveConstructor{⊥⊏} \AgdaBound{x} \AgdaBound{p₂⊏xs}\AgdaSymbol{)} \AgdaInductiveConstructor{,} \AgdaInductiveConstructor{refl}\<%
\\
\>[0]\AgdaIndent{2}{}\<[2]%
\>[2]\<%
\\
\>[0]\AgdaIndent{2}{}\<[2]%
\>[2]\AgdaFunction{⟶-prepend-⊤⊤} \AgdaSymbol{:} \AgdaSymbol{∀} \AgdaSymbol{\{}\AgdaBound{n}\AgdaSymbol{\}\{}\AgdaBound{f₁} \AgdaBound{f₂} \AgdaSymbol{:} \AgdaFunction{Form} \AgdaBound{n}\AgdaSymbol{\}}\<%
\\
\>[2]\AgdaIndent{4}{}\<[4]%
\>[4]\AgdaSymbol{→} \AgdaSymbol{\{}\AgdaBound{p₁} \AgdaSymbol{:} \AgdaDatatype{Patch} \AgdaBound{f₁}\AgdaSymbol{\}} \AgdaSymbol{→} \AgdaSymbol{\{}\AgdaBound{p₂} \AgdaSymbol{:} \AgdaDatatype{Patch} \AgdaBound{f₂}\AgdaSymbol{\}}\<%
\\
\>[2]\AgdaIndent{4}{}\<[4]%
\>[4]\AgdaSymbol{→} \AgdaSymbol{(}\AgdaBound{from} \AgdaBound{to} \AgdaSymbol{:} \AgdaBound{A}\AgdaSymbol{)}\<%
\\
\>[2]\AgdaIndent{4}{}\<[4]%
\>[4]\AgdaSymbol{→} \AgdaSymbol{(}\AgdaBound{p₁} \AgdaFunction{⟶} \AgdaBound{p₂}\AgdaSymbol{)} \AgdaSymbol{→} \AgdaSymbol{(}\AgdaInductiveConstructor{⟨} \AgdaBound{from} \AgdaInductiveConstructor{⇒} \AgdaBound{to} \AgdaInductiveConstructor{⟩∷} \AgdaBound{p₁} \AgdaFunction{⟶} \AgdaInductiveConstructor{⟨} \AgdaBound{from} \AgdaInductiveConstructor{⇒} \AgdaBound{to} \AgdaInductiveConstructor{⟩∷} \AgdaBound{p₂}\AgdaSymbol{)}\<%
\\
\>[0]\AgdaIndent{2}{}\<[2]%
\>[2]\AgdaFunction{⟶-prepend-⊤⊤} \AgdaSymbol{\{p₁} \AgdaSymbol{=} \AgdaBound{p₁}\AgdaSymbol{\}\{}\AgdaBound{p₂}\AgdaSymbol{\}} \AgdaSymbol{.}\AgdaBound{x} \AgdaBound{to} \AgdaBound{p₁⟷p₂} \AgdaSymbol{(}\AgdaBound{x} \AgdaInductiveConstructor{∷} \AgdaBound{xs}\AgdaSymbol{)} \AgdaSymbol{(}\AgdaInductiveConstructor{⊤⊏} \AgdaSymbol{.}\AgdaBound{x} \AgdaSymbol{.}\AgdaBound{to} \AgdaBound{p₁⊏xs}\AgdaSymbol{)} \<[68]%
\>[68]\<%
\\
\>[2]\AgdaIndent{4}{}\<[4]%
\>[4]\AgdaKeyword{with} \AgdaFunction{patch} \AgdaBound{p₁} \AgdaBound{xs} \AgdaBound{p₁⊏xs} \AgdaSymbol{|} \AgdaBound{p₁⟷p₂} \AgdaBound{xs} \AgdaBound{p₁⊏xs} \<[44]%
\>[44]\<%
\\
\>[0]\AgdaIndent{2}{}\<[2]%
\>[2]\AgdaSymbol{...} \AgdaSymbol{|} \AgdaSymbol{.(}\AgdaFunction{patch} \AgdaBound{p₂} \AgdaBound{xs} \AgdaBound{p₂⊏xs}\AgdaSymbol{)} \AgdaSymbol{|} \AgdaBound{p₂⊏xs} \AgdaInductiveConstructor{,} \AgdaInductiveConstructor{refl} \AgdaSymbol{=} \AgdaInductiveConstructor{⊤⊏} \AgdaBound{x} \AgdaBound{to} \AgdaBound{p₂⊏xs} \AgdaInductiveConstructor{,} \AgdaInductiveConstructor{refl}\<%
\\
%
\\
\>[0]\AgdaIndent{2}{}\<[2]%
\>[2]\AgdaKeyword{module} \AgdaModule{⟷-∧-lemmas} \AgdaKeyword{where}\<%
\\
%
\\
\>[2]\AgdaIndent{4}{}\<[4]%
\>[4]\AgdaFunction{∥-comm} \AgdaSymbol{:} \AgdaSymbol{∀} \AgdaSymbol{\{}\AgdaBound{n}\AgdaSymbol{\}\{}\AgdaBound{f₁} \AgdaBound{f₂} \AgdaSymbol{:} \AgdaFunction{Form} \AgdaBound{n}\AgdaSymbol{\}} \AgdaSymbol{→} \AgdaBound{f₁} \AgdaDatatype{∥} \AgdaBound{f₂} \AgdaSymbol{→} \AgdaBound{f₂} \AgdaDatatype{∥} \AgdaBound{f₁}\<%
\\
\>[2]\AgdaIndent{4}{}\<[4]%
\>[4]\AgdaFunction{∥-comm} \AgdaInductiveConstructor{[]∥[]} \AgdaSymbol{=} \AgdaInductiveConstructor{[]∥[]}\<%
\\
\>[2]\AgdaIndent{4}{}\<[4]%
\>[4]\AgdaFunction{∥-comm} \AgdaSymbol{(}\AgdaInductiveConstructor{⊥∥⊥} \AgdaBound{f₁∥f₂}\AgdaSymbol{)} \AgdaSymbol{=} \AgdaInductiveConstructor{⊥∥⊥} \AgdaSymbol{(}\AgdaFunction{∥-comm} \AgdaBound{f₁∥f₂}\AgdaSymbol{)}\<%
\\
\>[2]\AgdaIndent{4}{}\<[4]%
\>[4]\AgdaFunction{∥-comm} \AgdaSymbol{(}\AgdaInductiveConstructor{⊤∥⊥} \AgdaBound{f₁∥f₂}\AgdaSymbol{)} \AgdaSymbol{=} \AgdaInductiveConstructor{⊥∥⊤} \AgdaSymbol{(}\AgdaFunction{∥-comm} \AgdaBound{f₁∥f₂}\AgdaSymbol{)}\<%
\\
\>[2]\AgdaIndent{4}{}\<[4]%
\>[4]\AgdaFunction{∥-comm} \AgdaSymbol{(}\AgdaInductiveConstructor{⊥∥⊤} \AgdaBound{f₁∥f₂}\AgdaSymbol{)} \AgdaSymbol{=} \AgdaInductiveConstructor{⊤∥⊥} \AgdaSymbol{(}\AgdaFunction{∥-comm} \AgdaBound{f₁∥f₂}\AgdaSymbol{)}\<%
\\
%
\\
\>[2]\AgdaIndent{4}{}\<[4]%
\>[4]\AgdaFunction{lemma-∥-unite} \AgdaSymbol{:} \AgdaSymbol{∀} \AgdaSymbol{\{}\AgdaBound{n}\AgdaSymbol{\}\{}\AgdaBound{f₁} \AgdaBound{f₂} \AgdaBound{f₃} \AgdaSymbol{:} \AgdaFunction{Form} \AgdaBound{n}\AgdaSymbol{\}} \<[45]%
\>[45]\<%
\\
\>[4]\AgdaIndent{6}{}\<[6]%
\>[6]\AgdaSymbol{→} \AgdaSymbol{(}\AgdaBound{f₁∥f₂} \AgdaSymbol{:} \AgdaBound{f₁} \AgdaDatatype{∥} \AgdaBound{f₂}\AgdaSymbol{)} \AgdaSymbol{→} \AgdaBound{f₂} \AgdaDatatype{∥} \AgdaBound{f₃} \AgdaSymbol{→} \AgdaBound{f₁} \AgdaDatatype{∥} \AgdaBound{f₃}\<%
\\
\>[4]\AgdaIndent{6}{}\<[6]%
\>[6]\AgdaSymbol{→} \AgdaFunction{unite} \AgdaBound{f₁∥f₂} \AgdaDatatype{∥} \AgdaBound{f₃}\<%
\\
\>[0]\AgdaIndent{4}{}\<[4]%
\>[4]\AgdaFunction{lemma-∥-unite} \AgdaInductiveConstructor{[]∥[]} \AgdaInductiveConstructor{[]∥[]} \AgdaInductiveConstructor{[]∥[]} \AgdaSymbol{=} \AgdaInductiveConstructor{[]∥[]}\<%
\\
\>[0]\AgdaIndent{4}{}\<[4]%
\>[4]\AgdaFunction{lemma-∥-unite} \AgdaSymbol{(}\AgdaInductiveConstructor{⊥∥⊥} \AgdaBound{f₁∥f₂}\AgdaSymbol{)} \AgdaSymbol{(}\AgdaInductiveConstructor{⊥∥⊥} \AgdaBound{f₂∥f₃}\AgdaSymbol{)} \AgdaSymbol{(}\AgdaInductiveConstructor{⊥∥⊥} \AgdaBound{f₁∥f₃}\AgdaSymbol{)} \AgdaSymbol{=} \<[56]%
\>[56]\<%
\\
\>[4]\AgdaIndent{6}{}\<[6]%
\>[6]\AgdaInductiveConstructor{⊥∥⊥} \AgdaSymbol{(}\AgdaFunction{lemma-∥-unite} \AgdaBound{f₁∥f₂} \AgdaBound{f₂∥f₃} \AgdaBound{f₁∥f₃}\AgdaSymbol{)}\<%
\\
\>[0]\AgdaIndent{4}{}\<[4]%
\>[4]\AgdaFunction{lemma-∥-unite} \AgdaSymbol{(}\AgdaInductiveConstructor{⊥∥⊥} \AgdaBound{f₁∥f₂}\AgdaSymbol{)} \AgdaSymbol{(}\AgdaInductiveConstructor{⊥∥⊤} \AgdaBound{f₂∥f₃}\AgdaSymbol{)} \AgdaSymbol{(}\AgdaInductiveConstructor{⊥∥⊤} \AgdaBound{f₁∥f₃}\AgdaSymbol{)} \AgdaSymbol{=} \<[56]%
\>[56]\<%
\\
\>[4]\AgdaIndent{6}{}\<[6]%
\>[6]\AgdaInductiveConstructor{⊥∥⊤} \AgdaSymbol{(}\AgdaFunction{lemma-∥-unite} \AgdaBound{f₁∥f₂} \AgdaBound{f₂∥f₃} \AgdaBound{f₁∥f₃}\AgdaSymbol{)}\<%
\\
\>[0]\AgdaIndent{4}{}\<[4]%
\>[4]\AgdaFunction{lemma-∥-unite} \AgdaSymbol{(}\AgdaInductiveConstructor{⊤∥⊥} \AgdaBound{f₁∥f₂}\AgdaSymbol{)} \AgdaSymbol{(}\AgdaInductiveConstructor{⊥∥⊥} \AgdaBound{f₂∥f₃}\AgdaSymbol{)} \AgdaSymbol{(}\AgdaInductiveConstructor{⊤∥⊥} \AgdaBound{f₁∥f₃}\AgdaSymbol{)} \AgdaSymbol{=} \<[56]%
\>[56]\<%
\\
\>[4]\AgdaIndent{6}{}\<[6]%
\>[6]\AgdaInductiveConstructor{⊤∥⊥} \AgdaSymbol{(}\AgdaFunction{lemma-∥-unite} \AgdaBound{f₁∥f₂} \AgdaBound{f₂∥f₃} \AgdaBound{f₁∥f₃}\AgdaSymbol{)}\<%
\\
\>[0]\AgdaIndent{4}{}\<[4]%
\>[4]\AgdaFunction{lemma-∥-unite} \AgdaSymbol{(}\AgdaInductiveConstructor{⊥∥⊤} \AgdaBound{f₁∥f₂}\AgdaSymbol{)} \AgdaSymbol{(}\AgdaInductiveConstructor{⊤∥⊥} \AgdaBound{f₂∥f₃}\AgdaSymbol{)} \AgdaSymbol{(}\AgdaInductiveConstructor{⊥∥⊥} \AgdaBound{f₁∥f₃}\AgdaSymbol{)} \AgdaSymbol{=} \<[56]%
\>[56]\<%
\\
\>[4]\AgdaIndent{6}{}\<[6]%
\>[6]\AgdaInductiveConstructor{⊤∥⊥} \AgdaSymbol{(}\AgdaFunction{lemma-∥-unite} \AgdaBound{f₁∥f₂} \AgdaBound{f₂∥f₃} \AgdaBound{f₁∥f₃}\AgdaSymbol{)}\<%
\\
%
\\
\>[0]\AgdaIndent{4}{}\<[4]%
\>[4]\AgdaFunction{∧-comm} \AgdaSymbol{:} \AgdaSymbol{∀} \AgdaSymbol{\{}\AgdaBound{n}\AgdaSymbol{\}\{}\AgdaBound{f₁} \AgdaBound{f₂} \AgdaSymbol{:} \AgdaFunction{Form} \AgdaBound{n}\AgdaSymbol{\}}\<%
\\
\>[4]\AgdaIndent{6}{}\<[6]%
\>[6]\AgdaSymbol{→} \AgdaSymbol{(}\AgdaBound{f₁∥f₂} \AgdaSymbol{:} \AgdaBound{f₁} \AgdaDatatype{∥} \AgdaBound{f₂}\AgdaSymbol{)}\<%
\\
\>[4]\AgdaIndent{6}{}\<[6]%
\>[6]\AgdaSymbol{→} \AgdaSymbol{(}\AgdaBound{p₁} \AgdaSymbol{:} \AgdaDatatype{Patch} \AgdaBound{f₁}\AgdaSymbol{)} \AgdaSymbol{→} \AgdaSymbol{(}\AgdaBound{p₂} \AgdaSymbol{:} \AgdaDatatype{Patch} \AgdaBound{f₂}\AgdaSymbol{)}\<%
\\
\>[4]\AgdaIndent{6}{}\<[6]%
\>[6]\AgdaSymbol{→} \AgdaSymbol{((}\AgdaBound{p₁} \AgdaFunction{∧ₚ} \AgdaBound{p₂}\AgdaSymbol{)} \AgdaBound{f₁∥f₂}\AgdaSymbol{)} \AgdaFunction{⟷} \AgdaSymbol{((}\AgdaBound{p₂} \AgdaFunction{∧ₚ} \AgdaBound{p₁}\AgdaSymbol{)} \AgdaSymbol{(}\AgdaFunction{∥-comm} \AgdaBound{f₁∥f₂}\AgdaSymbol{))}\<%
\\
\>[0]\AgdaIndent{4}{}\<[4]%
\>[4]\AgdaFunction{∧-comm} \AgdaInductiveConstructor{[]∥[]} \AgdaInductiveConstructor{O} \AgdaInductiveConstructor{O} \AgdaSymbol{=} \AgdaSymbol{(λ} \AgdaBound{x} \AgdaBound{x₁} \AgdaSymbol{→} \AgdaBound{x₁} \AgdaInductiveConstructor{,} \AgdaInductiveConstructor{refl}\AgdaSymbol{)} \AgdaInductiveConstructor{,} \AgdaSymbol{(λ} \AgdaBound{x} \AgdaBound{x₁} \AgdaSymbol{→} \AgdaBound{x₁} \AgdaInductiveConstructor{,} \AgdaInductiveConstructor{refl}\AgdaSymbol{)}\<%
\\
\>[0]\AgdaIndent{4}{}\<[4]%
\>[4]\AgdaFunction{∧-comm} \AgdaSymbol{(}\AgdaInductiveConstructor{⊥∥⊥} \AgdaBound{f₁∥f₂}\AgdaSymbol{)} \AgdaSymbol{(}\AgdaInductiveConstructor{⊥∷} \AgdaBound{p₁}\AgdaSymbol{)} \AgdaSymbol{(}\AgdaInductiveConstructor{⊥∷} \AgdaBound{p₂}\AgdaSymbol{)} \AgdaSymbol{=}\<%
\\
\>[4]\AgdaIndent{6}{}\<[6]%
\>[6]\AgdaSymbol{(}\AgdaFunction{⟶-prepend-⊥⊥} \AgdaSymbol{(}\AgdaField{fst} \AgdaFunction{p}\AgdaSymbol{))} \AgdaInductiveConstructor{,} \AgdaSymbol{(}\AgdaFunction{⟶-prepend-⊥⊥} \AgdaSymbol{(}\AgdaField{snd} \AgdaFunction{p}\AgdaSymbol{))} \AgdaKeyword{where}\<%
\\
\>[4]\AgdaIndent{6}{}\<[6]%
\>[6]\AgdaFunction{p} \AgdaSymbol{=} \AgdaFunction{∧-comm} \AgdaBound{f₁∥f₂} \AgdaBound{p₁} \AgdaBound{p₂}\<%
\\
\>[0]\AgdaIndent{4}{}\<[4]%
\>[4]\AgdaFunction{∧-comm} \AgdaSymbol{(}\AgdaInductiveConstructor{⊤∥⊥} \AgdaBound{f₁∥f₂}\AgdaSymbol{)} \AgdaSymbol{(}\AgdaInductiveConstructor{⟨} \AgdaBound{from} \AgdaInductiveConstructor{⇒} \AgdaBound{to} \AgdaInductiveConstructor{⟩∷} \AgdaBound{p₁}\AgdaSymbol{)} \AgdaSymbol{(}\AgdaInductiveConstructor{⊥∷} \AgdaBound{p₂}\AgdaSymbol{)} \AgdaSymbol{=}\<%
\\
\>[4]\AgdaIndent{6}{}\<[6]%
\>[6]\AgdaSymbol{(}\AgdaFunction{⟶-prepend-⊤⊤} \AgdaBound{from} \AgdaBound{to} \AgdaSymbol{(}\AgdaField{fst} \AgdaFunction{p}\AgdaSymbol{))} \AgdaInductiveConstructor{,} \AgdaSymbol{(}\AgdaFunction{⟶-prepend-⊤⊤} \AgdaBound{from} \AgdaBound{to} \AgdaSymbol{(}\AgdaField{snd} \AgdaFunction{p}\AgdaSymbol{))} \AgdaKeyword{where}\<%
\\
\>[4]\AgdaIndent{6}{}\<[6]%
\>[6]\AgdaFunction{p} \AgdaSymbol{=} \AgdaFunction{∧-comm} \AgdaBound{f₁∥f₂} \AgdaBound{p₁} \AgdaBound{p₂}\<%
\\
\>[0]\AgdaIndent{4}{}\<[4]%
\>[4]\AgdaFunction{∧-comm} \AgdaSymbol{(}\AgdaInductiveConstructor{⊥∥⊤} \AgdaBound{f₁∥f₂}\AgdaSymbol{)} \AgdaSymbol{(}\AgdaInductiveConstructor{⊥∷} \AgdaBound{p₁}\AgdaSymbol{)} \AgdaSymbol{(}\AgdaInductiveConstructor{⟨} \AgdaBound{from} \AgdaInductiveConstructor{⇒} \AgdaBound{to} \AgdaInductiveConstructor{⟩∷} \AgdaBound{p₂}\AgdaSymbol{)} \AgdaSymbol{=} \<[53]%
\>[53]\<%
\\
\>[4]\AgdaIndent{6}{}\<[6]%
\>[6]\AgdaSymbol{(}\AgdaFunction{⟶-prepend-⊤⊤} \AgdaBound{from} \AgdaBound{to} \AgdaSymbol{(}\AgdaField{fst} \AgdaFunction{p}\AgdaSymbol{))} \AgdaInductiveConstructor{,} \AgdaSymbol{(}\AgdaFunction{⟶-prepend-⊤⊤} \AgdaBound{from} \AgdaBound{to} \AgdaSymbol{(}\AgdaField{snd} \AgdaFunction{p}\AgdaSymbol{))} \AgdaKeyword{where}\<%
\\
\>[4]\AgdaIndent{6}{}\<[6]%
\>[6]\AgdaFunction{p} \AgdaSymbol{=} \AgdaFunction{∧-comm} \AgdaBound{f₁∥f₂} \AgdaBound{p₁} \AgdaBound{p₂}\<%
\\
%
\\
\>[0]\AgdaIndent{4}{}\<[4]%
\>[4]\AgdaFunction{∧-trans} \AgdaSymbol{:} \AgdaSymbol{∀} \AgdaSymbol{\{}\AgdaBound{n}\AgdaSymbol{\}\{}\AgdaBound{f₁} \AgdaBound{f₂} \AgdaBound{f₃} \AgdaSymbol{:} \AgdaFunction{Form} \AgdaBound{n}\AgdaSymbol{\}}\<%
\\
\>[4]\AgdaIndent{6}{}\<[6]%
\>[6]\AgdaSymbol{→} \AgdaSymbol{(}\AgdaBound{f₁∥f₂} \AgdaSymbol{:} \AgdaBound{f₁} \AgdaDatatype{∥} \AgdaBound{f₂}\AgdaSymbol{)} \AgdaSymbol{→} \AgdaSymbol{(}\AgdaBound{f₂∥f₃} \AgdaSymbol{:} \AgdaBound{f₂} \AgdaDatatype{∥} \AgdaBound{f₃}\AgdaSymbol{)} \AgdaSymbol{→} \AgdaSymbol{(}\AgdaBound{f₁∥f₃} \AgdaSymbol{:} \AgdaBound{f₁} \AgdaDatatype{∥} \AgdaBound{f₃}\AgdaSymbol{)}\<%
\\
\>[4]\AgdaIndent{6}{}\<[6]%
\>[6]\AgdaSymbol{→} \AgdaSymbol{(}\AgdaBound{p₁} \AgdaSymbol{:} \AgdaDatatype{Patch} \AgdaBound{f₁}\AgdaSymbol{)} \AgdaSymbol{→} \AgdaSymbol{(}\AgdaBound{p₂} \AgdaSymbol{:} \AgdaDatatype{Patch} \AgdaBound{f₂}\AgdaSymbol{)} \AgdaSymbol{→} \AgdaSymbol{(}\AgdaBound{p₃} \AgdaSymbol{:} \AgdaDatatype{Patch} \AgdaBound{f₃}\AgdaSymbol{)}\<%
\\
\>[4]\AgdaIndent{6}{}\<[6]%
\>[6]\AgdaSymbol{→} \AgdaSymbol{((}\AgdaBound{p₁} \AgdaFunction{∧ₚ} \AgdaBound{p₂}\AgdaSymbol{)} \AgdaBound{f₁∥f₂} \AgdaFunction{∧ₚ} \AgdaBound{p₃}\AgdaSymbol{)} \AgdaSymbol{(}\AgdaFunction{lemma-∥-unite} \AgdaBound{f₁∥f₂} \AgdaBound{f₂∥f₃} \AgdaBound{f₁∥f₃}\AgdaSymbol{)}\<%
\\
\>[6]\AgdaIndent{8}{}\<[8]%
\>[8]\AgdaFunction{⟷} \<[10]%
\>[10]\<%
\\
\>[6]\AgdaIndent{8}{}\<[8]%
\>[8]\AgdaSymbol{(}\AgdaBound{p₁} \AgdaFunction{∧ₚ} \AgdaSymbol{(}\AgdaBound{p₂} \AgdaFunction{∧ₚ} \AgdaBound{p₃}\AgdaSymbol{)} \AgdaBound{f₂∥f₃}\AgdaSymbol{)} \<[33]%
\>[33]\<%
\\
\>[8]\AgdaIndent{10}{}\<[10]%
\>[10]\AgdaSymbol{(}\AgdaFunction{∥-comm} \AgdaSymbol{(}\AgdaFunction{lemma-∥-unite} \AgdaBound{f₂∥f₃} \AgdaSymbol{(}\AgdaFunction{∥-comm} \AgdaBound{f₁∥f₃}\AgdaSymbol{)} \AgdaSymbol{(}\AgdaFunction{∥-comm} \AgdaBound{f₁∥f₂}\AgdaSymbol{)))}\<%
\\
%
\\
\>[0]\AgdaIndent{4}{}\<[4]%
\>[4]\AgdaFunction{∧-trans} \AgdaInductiveConstructor{[]∥[]} \AgdaInductiveConstructor{[]∥[]} \AgdaInductiveConstructor{[]∥[]} \AgdaInductiveConstructor{O} \AgdaInductiveConstructor{O} \AgdaInductiveConstructor{O} \AgdaSymbol{=} \<[38]%
\>[38]\<%
\\
\>[4]\AgdaIndent{6}{}\<[6]%
\>[6]\AgdaSymbol{(λ} \AgdaBound{x} \AgdaBound{x₁} \AgdaSymbol{→} \AgdaBound{x₁} \AgdaInductiveConstructor{,} \AgdaInductiveConstructor{refl}\AgdaSymbol{)} \AgdaInductiveConstructor{,} \AgdaSymbol{(λ} \AgdaBound{x} \AgdaBound{x₁} \AgdaSymbol{→} \AgdaBound{x₁} \AgdaInductiveConstructor{,} \AgdaInductiveConstructor{refl}\AgdaSymbol{)}\<%
\\
\>[0]\AgdaIndent{4}{}\<[4]%
\>[4]\AgdaFunction{∧-trans} \AgdaSymbol{(}\AgdaInductiveConstructor{⊥∥⊥} \AgdaBound{f₁∥f₂}\AgdaSymbol{)} \AgdaSymbol{(}\AgdaInductiveConstructor{⊥∥⊥} \AgdaBound{f₂∥f₃}\AgdaSymbol{)} \AgdaSymbol{(}\AgdaInductiveConstructor{⊥∥⊥} \AgdaBound{f₁∥f₃}\AgdaSymbol{)} \AgdaSymbol{(}\AgdaInductiveConstructor{⊥∷} \AgdaBound{p₁}\AgdaSymbol{)} \AgdaSymbol{(}\AgdaInductiveConstructor{⊥∷} \AgdaBound{p₂}\AgdaSymbol{)} \AgdaSymbol{(}\AgdaInductiveConstructor{⊥∷} \AgdaBound{p₃}\AgdaSymbol{)} \AgdaSymbol{=} \<[74]%
\>[74]\<%
\\
\>[4]\AgdaIndent{6}{}\<[6]%
\>[6]\AgdaSymbol{(}\AgdaFunction{⟶-prepend-⊥⊥} \AgdaSymbol{(}\AgdaField{fst} \AgdaFunction{p}\AgdaSymbol{))} \AgdaInductiveConstructor{,} \AgdaSymbol{(}\AgdaFunction{⟶-prepend-⊥⊥} \AgdaSymbol{(}\AgdaField{snd} \AgdaFunction{p}\AgdaSymbol{))} \AgdaKeyword{where}\<%
\\
\>[4]\AgdaIndent{6}{}\<[6]%
\>[6]\AgdaFunction{p} \AgdaSymbol{=} \AgdaFunction{∧-trans} \AgdaBound{f₁∥f₂} \AgdaBound{f₂∥f₃} \AgdaBound{f₁∥f₃} \AgdaBound{p₁} \AgdaBound{p₂} \AgdaBound{p₃}\<%
\\
\>[0]\AgdaIndent{4}{}\<[4]%
\>[4]\AgdaFunction{∧-trans} \AgdaSymbol{(}\AgdaInductiveConstructor{⊥∥⊥} \AgdaBound{f₁∥f₂}\AgdaSymbol{)} \AgdaSymbol{(}\AgdaInductiveConstructor{⊥∥⊤} \AgdaBound{f₂∥f₃}\AgdaSymbol{)} \AgdaSymbol{(}\AgdaInductiveConstructor{⊥∥⊤} \AgdaBound{f₁∥f₃}\AgdaSymbol{)} \AgdaSymbol{(}\AgdaInductiveConstructor{⊥∷} \AgdaBound{p₁}\AgdaSymbol{)} \AgdaSymbol{(}\AgdaInductiveConstructor{⊥∷} \AgdaBound{p₂}\AgdaSymbol{)} \<[64]%
\>[64]\<%
\\
\>[4]\AgdaIndent{6}{}\<[6]%
\>[6]\AgdaSymbol{(}\AgdaInductiveConstructor{⟨} \AgdaBound{from} \AgdaInductiveConstructor{⇒} \AgdaBound{to} \AgdaInductiveConstructor{⟩∷} \AgdaBound{p₃}\AgdaSymbol{)} \AgdaSymbol{=} \<[28]%
\>[28]\<%
\\
\>[4]\AgdaIndent{6}{}\<[6]%
\>[6]\AgdaSymbol{(}\AgdaFunction{⟶-prepend-⊤⊤} \AgdaBound{from} \AgdaBound{to} \AgdaSymbol{(}\AgdaField{fst} \AgdaFunction{p}\AgdaSymbol{))} \AgdaInductiveConstructor{,} \AgdaSymbol{(}\AgdaFunction{⟶-prepend-⊤⊤} \AgdaBound{from} \AgdaBound{to} \AgdaSymbol{(}\AgdaField{snd} \AgdaFunction{p}\AgdaSymbol{))} \AgdaKeyword{where}\<%
\\
\>[4]\AgdaIndent{6}{}\<[6]%
\>[6]\AgdaFunction{p} \AgdaSymbol{=} \AgdaFunction{∧-trans} \AgdaBound{f₁∥f₂} \AgdaBound{f₂∥f₃} \AgdaBound{f₁∥f₃} \AgdaBound{p₁} \AgdaBound{p₂} \AgdaBound{p₃}\<%
\\
\>[0]\AgdaIndent{4}{}\<[4]%
\>[4]\AgdaFunction{∧-trans} \AgdaSymbol{(}\AgdaInductiveConstructor{⊤∥⊥} \AgdaBound{f₁∥f₂}\AgdaSymbol{)} \AgdaSymbol{(}\AgdaInductiveConstructor{⊥∥⊥} \AgdaBound{f₂∥f₃}\AgdaSymbol{)} \AgdaSymbol{(}\AgdaInductiveConstructor{⊤∥⊥} \AgdaBound{f₁∥f₃}\AgdaSymbol{)} \<[48]%
\>[48]\<%
\\
\>[4]\AgdaIndent{6}{}\<[6]%
\>[6]\AgdaSymbol{(}\AgdaInductiveConstructor{⟨} \AgdaBound{from} \AgdaInductiveConstructor{⇒} \AgdaBound{to} \AgdaInductiveConstructor{⟩∷} \AgdaBound{p₁}\AgdaSymbol{)} \AgdaSymbol{(}\AgdaInductiveConstructor{⊥∷} \AgdaBound{p₂}\AgdaSymbol{)} \AgdaSymbol{(}\AgdaInductiveConstructor{⊥∷} \AgdaBound{p₃}\AgdaSymbol{)} \AgdaSymbol{=} \<[44]%
\>[44]\<%
\\
\>[4]\AgdaIndent{6}{}\<[6]%
\>[6]\AgdaSymbol{(}\AgdaFunction{⟶-prepend-⊤⊤} \AgdaBound{from} \AgdaBound{to} \AgdaSymbol{(}\AgdaField{fst} \AgdaFunction{p}\AgdaSymbol{))} \AgdaInductiveConstructor{,} \AgdaSymbol{(}\AgdaFunction{⟶-prepend-⊤⊤} \AgdaBound{from} \AgdaBound{to} \AgdaSymbol{(}\AgdaField{snd} \AgdaFunction{p}\AgdaSymbol{))} \AgdaKeyword{where} \<[76]%
\>[76]\<%
\\
\>[4]\AgdaIndent{6}{}\<[6]%
\>[6]\AgdaFunction{p} \AgdaSymbol{=} \AgdaFunction{∧-trans} \AgdaBound{f₁∥f₂} \AgdaBound{f₂∥f₃} \AgdaBound{f₁∥f₃} \AgdaBound{p₁} \AgdaBound{p₂} \AgdaBound{p₃}\<%
\\
\>[0]\AgdaIndent{4}{}\<[4]%
\>[4]\AgdaFunction{∧-trans} \AgdaSymbol{(}\AgdaInductiveConstructor{⊤∥⊥} \AgdaBound{f₁∥f₂}\AgdaSymbol{)} \AgdaSymbol{(}\AgdaInductiveConstructor{⊥∥⊤} \AgdaBound{f₂∥f₃}\AgdaSymbol{)} \AgdaSymbol{()} \AgdaBound{p₁} \AgdaBound{p₂} \AgdaBound{p₃}\<%
\\
\>[0]\AgdaIndent{4}{}\<[4]%
\>[4]\AgdaFunction{∧-trans} \AgdaSymbol{(}\AgdaInductiveConstructor{⊥∥⊤} \AgdaBound{f₁∥f₂}\AgdaSymbol{)} \AgdaSymbol{(}\AgdaInductiveConstructor{⊤∥⊥} \AgdaBound{f₂∥f₃}\AgdaSymbol{)} \AgdaSymbol{(}\AgdaInductiveConstructor{⊥∥⊥} \AgdaBound{f₁∥f₃}\AgdaSymbol{)} \AgdaSymbol{(}\AgdaInductiveConstructor{⊥∷} \AgdaBound{p₁}\AgdaSymbol{)} \<[56]%
\>[56]\<%
\\
\>[4]\AgdaIndent{6}{}\<[6]%
\>[6]\AgdaSymbol{(}\AgdaInductiveConstructor{⟨} \AgdaBound{from} \AgdaInductiveConstructor{⇒} \AgdaBound{to} \AgdaInductiveConstructor{⟩∷} \AgdaBound{p₂}\AgdaSymbol{)} \AgdaSymbol{(}\AgdaInductiveConstructor{⊥∷} \AgdaBound{p₃}\AgdaSymbol{)} \AgdaSymbol{=} \<[36]%
\>[36]\<%
\\
\>[4]\AgdaIndent{6}{}\<[6]%
\>[6]\AgdaSymbol{(}\AgdaFunction{⟶-prepend-⊤⊤} \AgdaBound{from} \AgdaBound{to} \AgdaSymbol{(}\AgdaField{fst} \AgdaFunction{p}\AgdaSymbol{))} \AgdaInductiveConstructor{,} \AgdaSymbol{(}\AgdaFunction{⟶-prepend-⊤⊤} \AgdaBound{from} \AgdaBound{to} \AgdaSymbol{(}\AgdaField{snd} \AgdaFunction{p}\AgdaSymbol{))} \AgdaKeyword{where} \<[76]%
\>[76]\<%
\\
\>[4]\AgdaIndent{6}{}\<[6]%
\>[6]\AgdaFunction{p} \AgdaSymbol{=} \AgdaFunction{∧-trans} \AgdaBound{f₁∥f₂} \AgdaBound{f₂∥f₃} \AgdaBound{f₁∥f₃} \AgdaBound{p₁} \AgdaBound{p₂} \AgdaBound{p₃}\<%
\\
%
\\
\>[0]\AgdaIndent{2}{}\<[2]%
\>[2]\AgdaKeyword{data} \AgdaDatatype{\_⋙?\_} \AgdaSymbol{:} \AgdaSymbol{∀} \AgdaSymbol{\{}\AgdaBound{n}\AgdaSymbol{\}\{}\AgdaBound{f₁} \AgdaBound{f₂} \AgdaSymbol{:} \AgdaFunction{Form} \AgdaBound{n}\AgdaSymbol{\}} \AgdaSymbol{→} \AgdaDatatype{Patch} \AgdaBound{f₁} \AgdaSymbol{→} \AgdaDatatype{Patch} \AgdaBound{f₂} \AgdaSymbol{→} \AgdaPrimitiveType{Set} \AgdaKeyword{where} \<[71]%
\>[71]\<%
\\
\>[2]\AgdaIndent{4}{}\<[4]%
\>[4]\AgdaInductiveConstructor{O⋙?O} \AgdaSymbol{:} \AgdaInductiveConstructor{O} \AgdaDatatype{⋙?} \AgdaInductiveConstructor{O}\<%
\\
\>[2]\AgdaIndent{4}{}\<[4]%
\>[4]\AgdaInductiveConstructor{⊥⋙?⊥} \AgdaSymbol{:} \AgdaSymbol{∀} \AgdaSymbol{\{}\AgdaBound{n}\AgdaSymbol{\}\{}\AgdaBound{f₁} \AgdaBound{f₂} \AgdaSymbol{:} \AgdaFunction{Form} \AgdaBound{n}\AgdaSymbol{\}} \AgdaSymbol{\{}\AgdaBound{p₁} \AgdaSymbol{:} \AgdaDatatype{Patch} \AgdaBound{f₁}\AgdaSymbol{\}} \AgdaSymbol{\{}\AgdaBound{p₂} \AgdaSymbol{:} \AgdaDatatype{Patch} \AgdaBound{f₂}\AgdaSymbol{\}}\<%
\\
\>[4]\AgdaIndent{6}{}\<[6]%
\>[6]\AgdaSymbol{→} \AgdaSymbol{(}\AgdaBound{p₁} \AgdaDatatype{⋙?} \AgdaBound{p₂}\AgdaSymbol{)} \AgdaSymbol{→} \AgdaSymbol{(}\AgdaInductiveConstructor{⊥∷} \AgdaBound{p₁}\AgdaSymbol{)} \AgdaDatatype{⋙?} \AgdaSymbol{(}\AgdaInductiveConstructor{⊥∷} \AgdaBound{p₂}\AgdaSymbol{)}\<%
\\
\>[0]\AgdaIndent{4}{}\<[4]%
\>[4]\AgdaInductiveConstructor{⊤⋙?⊥} \AgdaSymbol{:} \AgdaSymbol{∀} \AgdaSymbol{\{}\AgdaBound{n}\AgdaSymbol{\}\{}\AgdaBound{f₁} \AgdaBound{f₂} \AgdaSymbol{:} \AgdaFunction{Form} \AgdaBound{n}\AgdaSymbol{\}} \AgdaSymbol{\{}\AgdaBound{p₁} \AgdaSymbol{:} \AgdaDatatype{Patch} \AgdaBound{f₁}\AgdaSymbol{\}} \AgdaSymbol{\{}\AgdaBound{p₂} \AgdaSymbol{:} \AgdaDatatype{Patch} \AgdaBound{f₂}\AgdaSymbol{\}}\<%
\\
\>[4]\AgdaIndent{6}{}\<[6]%
\>[6]\AgdaSymbol{→} \AgdaSymbol{(}\AgdaBound{from} \AgdaBound{to} \AgdaSymbol{:} \AgdaBound{A}\AgdaSymbol{)} \AgdaSymbol{→} \AgdaSymbol{(}\AgdaBound{p₁} \AgdaDatatype{⋙?} \AgdaBound{p₂}\AgdaSymbol{)}\<%
\\
\>[4]\AgdaIndent{6}{}\<[6]%
\>[6]\AgdaSymbol{→} \AgdaSymbol{(}\AgdaInductiveConstructor{⟨} \AgdaBound{from} \AgdaInductiveConstructor{⇒} \AgdaBound{to} \AgdaInductiveConstructor{⟩∷} \AgdaBound{p₁}\AgdaSymbol{)} \AgdaDatatype{⋙?} \AgdaSymbol{(}\AgdaInductiveConstructor{⊥∷} \AgdaBound{p₂}\AgdaSymbol{)}\<%
\\
\>[0]\AgdaIndent{4}{}\<[4]%
\>[4]\AgdaInductiveConstructor{⊤⋙?⊤} \AgdaSymbol{:} \AgdaSymbol{∀} \AgdaSymbol{\{}\AgdaBound{n}\AgdaSymbol{\}\{}\AgdaBound{f₁} \AgdaBound{f₂} \AgdaSymbol{:} \AgdaFunction{Form} \AgdaBound{n}\AgdaSymbol{\}} \AgdaSymbol{\{}\AgdaBound{p₁} \AgdaSymbol{:} \AgdaDatatype{Patch} \AgdaBound{f₁}\AgdaSymbol{\}} \AgdaSymbol{\{}\AgdaBound{p₂} \AgdaSymbol{:} \AgdaDatatype{Patch} \AgdaBound{f₂}\AgdaSymbol{\}}\<%
\\
\>[4]\AgdaIndent{6}{}\<[6]%
\>[6]\AgdaSymbol{→} \AgdaSymbol{(}\AgdaBound{from} \AgdaBound{to} \AgdaBound{to₂} \AgdaSymbol{:} \AgdaBound{A}\AgdaSymbol{)} \AgdaSymbol{→} \AgdaSymbol{(}\AgdaBound{p₁} \AgdaDatatype{⋙?} \AgdaBound{p₂}\AgdaSymbol{)}\<%
\\
\>[4]\AgdaIndent{6}{}\<[6]%
\>[6]\AgdaSymbol{→} \AgdaSymbol{(}\AgdaInductiveConstructor{⟨} \AgdaBound{from} \AgdaInductiveConstructor{⇒} \AgdaBound{to} \AgdaInductiveConstructor{⟩∷} \AgdaBound{p₁}\AgdaSymbol{)} \AgdaDatatype{⋙?} \AgdaSymbol{(}\AgdaInductiveConstructor{⟨} \AgdaBound{to} \AgdaInductiveConstructor{⇒} \AgdaBound{to₂} \AgdaInductiveConstructor{⟩∷} \AgdaBound{p₂}\AgdaSymbol{)}\<%
\\
%
\\
\>[0]\AgdaIndent{2}{}\<[2]%
\>[2]\AgdaFunction{\_⋙ₚ\_} \AgdaSymbol{:} \AgdaSymbol{∀} \AgdaSymbol{\{}\AgdaBound{n}\AgdaSymbol{\}} \AgdaSymbol{\{}\AgdaBound{f₁} \AgdaBound{f₂} \AgdaSymbol{:} \AgdaFunction{Form} \AgdaBound{n}\AgdaSymbol{\}} \AgdaSymbol{(}\AgdaBound{p₁} \AgdaSymbol{:} \AgdaDatatype{Patch} \AgdaBound{f₁}\AgdaSymbol{)} \AgdaSymbol{(}\AgdaBound{p₂} \AgdaSymbol{:} \AgdaDatatype{Patch} \AgdaBound{f₂}\AgdaSymbol{)}\<%
\\
\>[2]\AgdaIndent{4}{}\<[4]%
\>[4]\AgdaSymbol{→} \AgdaSymbol{(}\AgdaBound{p₁} \AgdaDatatype{⋙?} \AgdaBound{p₂}\AgdaSymbol{)} \AgdaSymbol{→} \AgdaDatatype{Patch} \AgdaBound{f₁}\<%
\\
\>[0]\AgdaIndent{2}{}\<[2]%
\>[2]\AgdaFunction{\_⋙ₚ\_} \AgdaInductiveConstructor{O} \AgdaInductiveConstructor{O} \AgdaInductiveConstructor{O⋙?O} \AgdaSymbol{=} \AgdaInductiveConstructor{O}\<%
\\
\>[0]\AgdaIndent{2}{}\<[2]%
\>[2]\AgdaFunction{\_⋙ₚ\_} \AgdaSymbol{(}\AgdaInductiveConstructor{⊥∷} \AgdaBound{p₁}\AgdaSymbol{)} \AgdaSymbol{(}\AgdaInductiveConstructor{⊥∷} \AgdaBound{p₂}\AgdaSymbol{)} \AgdaSymbol{(}\AgdaInductiveConstructor{⊥⋙?⊥} \AgdaBound{p₁-p₂}\AgdaSymbol{)} \AgdaSymbol{=} \AgdaInductiveConstructor{⊥∷} \AgdaSymbol{((}\AgdaBound{p₁} \AgdaFunction{⋙ₚ} \AgdaBound{p₂}\AgdaSymbol{)} \AgdaBound{p₁-p₂}\AgdaSymbol{)}\<%
\\
\>[0]\AgdaIndent{2}{}\<[2]%
\>[2]\AgdaFunction{\_⋙ₚ\_} \AgdaSymbol{(}\AgdaInductiveConstructor{⟨} \AgdaBound{a} \AgdaInductiveConstructor{⇒} \AgdaBound{b} \AgdaInductiveConstructor{⟩∷} \AgdaBound{p₁}\AgdaSymbol{)} \AgdaSymbol{(}\AgdaInductiveConstructor{⊥∷} \AgdaBound{p₂}\AgdaSymbol{)} \AgdaSymbol{(}\AgdaInductiveConstructor{⊤⋙?⊥} \AgdaSymbol{.}\AgdaBound{a} \AgdaSymbol{.}\AgdaBound{b} \AgdaBound{p₁-p₂}\AgdaSymbol{)} \AgdaSymbol{=} \<[52]%
\>[52]\<%
\\
\>[2]\AgdaIndent{4}{}\<[4]%
\>[4]\AgdaInductiveConstructor{⟨} \AgdaBound{a} \AgdaInductiveConstructor{⇒} \AgdaBound{b} \AgdaInductiveConstructor{⟩∷} \AgdaSymbol{((}\AgdaBound{p₁} \AgdaFunction{⋙ₚ} \AgdaBound{p₂}\AgdaSymbol{)} \AgdaBound{p₁-p₂}\AgdaSymbol{)}\<%
\\
\>[0]\AgdaIndent{2}{}\<[2]%
\>[2]\AgdaFunction{\_⋙ₚ\_} \AgdaSymbol{(}\AgdaInductiveConstructor{⟨} \AgdaBound{a} \AgdaInductiveConstructor{⇒} \AgdaSymbol{.}\AgdaBound{b} \AgdaInductiveConstructor{⟩∷} \AgdaBound{p₁}\AgdaSymbol{)} \AgdaSymbol{(}\AgdaInductiveConstructor{⟨} \AgdaBound{b} \AgdaInductiveConstructor{⇒} \AgdaBound{c} \AgdaInductiveConstructor{⟩∷} \AgdaBound{p₂}\AgdaSymbol{)} \AgdaSymbol{(}\AgdaInductiveConstructor{⊤⋙?⊤} \AgdaSymbol{.}\AgdaBound{a} \AgdaSymbol{.}\AgdaBound{b} \AgdaSymbol{.}\AgdaBound{c} \AgdaBound{p₁-p₂}\AgdaSymbol{)} \AgdaSymbol{=} \<[64]%
\>[64]\<%
\\
\>[2]\AgdaIndent{4}{}\<[4]%
\>[4]\AgdaInductiveConstructor{⟨} \AgdaBound{a} \AgdaInductiveConstructor{⇒} \AgdaBound{c} \AgdaInductiveConstructor{⟩∷} \AgdaSymbol{((}\AgdaBound{p₁} \AgdaFunction{⋙ₚ} \AgdaBound{p₂}\AgdaSymbol{)} \AgdaBound{p₁-p₂}\AgdaSymbol{)}\<%
\\
%
\\
\>[0]\AgdaIndent{2}{}\<[2]%
\>[2]\AgdaKeyword{module} \AgdaModule{⟷-⋙-lemmas} \AgdaKeyword{where}\<%
\\
%
\\
\>[2]\AgdaIndent{4}{}\<[4]%
\>[4]\AgdaFunction{∧-⋙-lem} \AgdaSymbol{:} \AgdaSymbol{∀} \AgdaSymbol{\{}\AgdaBound{n}\AgdaSymbol{\}} \AgdaSymbol{\{}\AgdaBound{fᵃ} \AgdaBound{fᵇ} \AgdaBound{fᶜ'} \AgdaBound{fᶜ''} \AgdaSymbol{:} \AgdaFunction{Form} \AgdaBound{n}\AgdaSymbol{\}}\<%
\\
\>[4]\AgdaIndent{6}{}\<[6]%
\>[6]\AgdaSymbol{(}\AgdaBound{c'} \AgdaSymbol{:} \AgdaDatatype{Patch} \AgdaBound{fᶜ'}\AgdaSymbol{)(}\AgdaBound{c''} \AgdaSymbol{:} \AgdaDatatype{Patch} \AgdaBound{fᶜ''}\AgdaSymbol{)}\<%
\\
\>[4]\AgdaIndent{6}{}\<[6]%
\>[6]\AgdaSymbol{(}\AgdaBound{a} \AgdaSymbol{:} \AgdaDatatype{Patch} \AgdaBound{fᵃ}\AgdaSymbol{)(}\AgdaBound{b} \AgdaSymbol{:} \AgdaDatatype{Patch} \AgdaBound{fᵇ}\AgdaSymbol{)}\<%
\\
\>[4]\AgdaIndent{6}{}\<[6]%
\>[6]\AgdaSymbol{(}\AgdaBound{c'∥c''} \AgdaSymbol{:} \AgdaBound{fᶜ'} \AgdaDatatype{∥} \AgdaBound{fᶜ''}\AgdaSymbol{)} \<[28]%
\>[28]\<%
\\
\>[4]\AgdaIndent{6}{}\<[6]%
\>[6]\AgdaSymbol{(}\AgdaBound{a∥b} \AgdaSymbol{:} \AgdaBound{fᵃ} \AgdaDatatype{∥} \AgdaBound{fᵇ}\AgdaSymbol{)}\<%
\\
\>[4]\AgdaIndent{6}{}\<[6]%
\>[6]\AgdaSymbol{(}\AgdaBound{a∧b>>>?c} \AgdaSymbol{:} \AgdaSymbol{(}\AgdaBound{a} \AgdaFunction{∧ₚ} \AgdaBound{b}\AgdaSymbol{)} \AgdaBound{a∥b} \AgdaDatatype{⋙?} \AgdaSymbol{((}\AgdaBound{c'} \AgdaFunction{∧ₚ} \AgdaBound{c''}\AgdaSymbol{)} \AgdaBound{c'∥c''}\AgdaSymbol{))}\<%
\\
\>[4]\AgdaIndent{6}{}\<[6]%
\>[6]\AgdaSymbol{(}\AgdaBound{a>>>?c'} \AgdaSymbol{:} \AgdaBound{a} \AgdaDatatype{⋙?} \AgdaBound{c'}\AgdaSymbol{)}\<%
\\
\>[4]\AgdaIndent{6}{}\<[6]%
\>[6]\AgdaSymbol{(}\AgdaBound{b>>>?c''} \AgdaSymbol{:} \AgdaBound{b} \AgdaDatatype{⋙?} \AgdaBound{c''}\AgdaSymbol{)}\<%
\\
\>[4]\AgdaIndent{6}{}\<[6]%
\>[6]\AgdaSymbol{→} \AgdaSymbol{(((}\AgdaBound{a} \AgdaFunction{∧ₚ} \AgdaBound{b}\AgdaSymbol{)} \AgdaBound{a∥b}\AgdaSymbol{)} \AgdaFunction{⋙ₚ} \AgdaSymbol{((}\AgdaBound{c'} \AgdaFunction{∧ₚ} \AgdaBound{c''}\AgdaSymbol{)} \AgdaBound{c'∥c''}\AgdaSymbol{))} \AgdaBound{a∧b>>>?c} \<[58]%
\>[58]\<%
\\
\>[6]\AgdaIndent{8}{}\<[8]%
\>[8]\AgdaFunction{⟷} \AgdaSymbol{((}\AgdaBound{a} \AgdaFunction{⋙ₚ} \AgdaBound{c'}\AgdaSymbol{)} \AgdaBound{a>>>?c'} \AgdaFunction{∧ₚ} \AgdaSymbol{(}\AgdaBound{b} \AgdaFunction{⋙ₚ} \AgdaBound{c''}\AgdaSymbol{)} \AgdaBound{b>>>?c''}\AgdaSymbol{)} \AgdaBound{a∥b}\<%
\\
%
\\
\>[0]\AgdaIndent{4}{}\<[4]%
\>[4]\AgdaFunction{∧-⋙-lem} \AgdaInductiveConstructor{O} \AgdaInductiveConstructor{O} \AgdaInductiveConstructor{O} \AgdaInductiveConstructor{O} \AgdaInductiveConstructor{[]∥[]} \AgdaInductiveConstructor{[]∥[]} \AgdaInductiveConstructor{O⋙?O} \AgdaInductiveConstructor{O⋙?O} \AgdaInductiveConstructor{O⋙?O} \AgdaSymbol{=} \<[49]%
\>[49]\<%
\\
\>[4]\AgdaIndent{6}{}\<[6]%
\>[6]\AgdaSymbol{(λ} \AgdaBound{x} \AgdaBound{x₁} \AgdaSymbol{→} \AgdaBound{x₁} \AgdaInductiveConstructor{,} \AgdaInductiveConstructor{refl}\AgdaSymbol{)} \AgdaInductiveConstructor{,} \AgdaSymbol{(λ} \AgdaBound{x} \AgdaBound{x₁} \AgdaSymbol{→} \AgdaBound{x₁} \AgdaInductiveConstructor{,} \AgdaInductiveConstructor{refl}\AgdaSymbol{)}\<%
\\
\>[0]\AgdaIndent{4}{}\<[4]%
\>[4]\AgdaFunction{∧-⋙-lem} \AgdaSymbol{(}\AgdaInductiveConstructor{⊥∷} \AgdaBound{c'}\AgdaSymbol{)} \AgdaSymbol{(}\AgdaInductiveConstructor{⊥∷} \AgdaBound{c''}\AgdaSymbol{)} \AgdaSymbol{(}\AgdaInductiveConstructor{⊥∷} \AgdaBound{a}\AgdaSymbol{)} \AgdaSymbol{(}\AgdaInductiveConstructor{⊥∷} \AgdaBound{b}\AgdaSymbol{)} \AgdaSymbol{(}\AgdaInductiveConstructor{⊥∥⊥} \AgdaBound{c'∥c''}\AgdaSymbol{)} \AgdaSymbol{(}\AgdaInductiveConstructor{⊥∥⊥} \AgdaBound{a∥b}\AgdaSymbol{)} \<[66]%
\>[66]\<%
\\
\>[4]\AgdaIndent{6}{}\<[6]%
\>[6]\AgdaSymbol{(}\AgdaInductiveConstructor{⊥⋙?⊥} \AgdaBound{a∧b⋙?c}\AgdaSymbol{)} \AgdaSymbol{(}\AgdaInductiveConstructor{⊥⋙?⊥} \AgdaBound{a⋙?c'}\AgdaSymbol{)} \AgdaSymbol{(}\AgdaInductiveConstructor{⊥⋙?⊥} \AgdaBound{b⋙c''}\AgdaSymbol{)} \AgdaSymbol{=} \<[48]%
\>[48]\<%
\\
\>[4]\AgdaIndent{6}{}\<[6]%
\>[6]\AgdaSymbol{(}\AgdaFunction{⟶-prepend-⊥⊥} \AgdaSymbol{(}\AgdaField{fst} \AgdaFunction{p}\AgdaSymbol{))} \AgdaInductiveConstructor{,} \AgdaSymbol{(}\AgdaFunction{⟶-prepend-⊥⊥} \AgdaSymbol{(}\AgdaField{snd} \AgdaFunction{p}\AgdaSymbol{))} \AgdaKeyword{where}\<%
\\
\>[4]\AgdaIndent{6}{}\<[6]%
\>[6]\AgdaFunction{p} \AgdaSymbol{=} \AgdaFunction{∧-⋙-lem} \AgdaBound{c'} \AgdaBound{c''} \AgdaBound{a} \AgdaBound{b} \AgdaBound{c'∥c''} \AgdaBound{a∥b} \AgdaBound{a∧b⋙?c} \AgdaBound{a⋙?c'} \AgdaBound{b⋙c''}\<%
\\
\>[0]\AgdaIndent{4}{}\<[4]%
\>[4]\AgdaFunction{∧-⋙-lem} \AgdaSymbol{(}\AgdaInductiveConstructor{⊥∷} \AgdaBound{c'}\AgdaSymbol{)} \AgdaSymbol{(}\AgdaInductiveConstructor{⊥∷} \AgdaBound{c''}\AgdaSymbol{)} \AgdaSymbol{(}\AgdaInductiveConstructor{⊥∷} \AgdaBound{a}\AgdaSymbol{)} \<[36]%
\>[36]\<%
\\
\>[4]\AgdaIndent{6}{}\<[6]%
\>[6]\AgdaSymbol{(}\AgdaInductiveConstructor{⟨} \AgdaBound{from} \AgdaInductiveConstructor{⇒} \AgdaBound{to} \AgdaInductiveConstructor{⟩∷} \AgdaBound{b}\AgdaSymbol{)} \AgdaSymbol{(}\AgdaInductiveConstructor{⊥∥⊥} \AgdaBound{c'∥c''}\AgdaSymbol{)} \AgdaSymbol{(}\AgdaInductiveConstructor{⊥∥⊤} \AgdaBound{a∥b}\AgdaSymbol{)} \<[48]%
\>[48]\<%
\\
\>[4]\AgdaIndent{6}{}\<[6]%
\>[6]\AgdaSymbol{(}\AgdaInductiveConstructor{⊤⋙?⊥} \AgdaSymbol{.}\AgdaBound{from} \AgdaSymbol{.}\AgdaBound{to} \AgdaBound{a∧b⋙?c}\AgdaSymbol{)} \AgdaSymbol{(}\AgdaInductiveConstructor{⊥⋙?⊥} \AgdaBound{a⋙?c'}\AgdaSymbol{)} \AgdaSymbol{(}\AgdaInductiveConstructor{⊤⋙?⊥} \AgdaSymbol{.}\AgdaBound{from} \AgdaSymbol{.}\AgdaBound{to} \AgdaBound{b⋙c''}\AgdaSymbol{)} \AgdaSymbol{=} \<[68]%
\>[68]\<%
\\
\>[4]\AgdaIndent{6}{}\<[6]%
\>[6]\AgdaSymbol{(}\AgdaFunction{⟶-prepend-⊤⊤} \AgdaBound{from} \AgdaBound{to} \AgdaSymbol{(}\AgdaField{fst} \AgdaFunction{p}\AgdaSymbol{))} \AgdaInductiveConstructor{,} \AgdaSymbol{(}\AgdaFunction{⟶-prepend-⊤⊤} \AgdaBound{from} \AgdaBound{to} \AgdaSymbol{(}\AgdaField{snd} \AgdaFunction{p}\AgdaSymbol{))} \AgdaKeyword{where}\<%
\\
\>[4]\AgdaIndent{6}{}\<[6]%
\>[6]\AgdaFunction{p} \AgdaSymbol{=} \AgdaFunction{∧-⋙-lem} \AgdaBound{c'} \AgdaBound{c''} \AgdaBound{a} \AgdaBound{b} \AgdaBound{c'∥c''} \AgdaBound{a∥b} \AgdaBound{a∧b⋙?c} \AgdaBound{a⋙?c'} \AgdaBound{b⋙c''}\<%
\\
\>[0]\AgdaIndent{4}{}\<[4]%
\>[4]\AgdaFunction{∧-⋙-lem} \AgdaSymbol{(}\AgdaInductiveConstructor{⊥∷} \AgdaBound{c'}\AgdaSymbol{)} \AgdaSymbol{(}\AgdaInductiveConstructor{⊥∷} \AgdaBound{c''}\AgdaSymbol{)} \AgdaSymbol{(}\AgdaInductiveConstructor{⟨} \AgdaBound{from} \AgdaInductiveConstructor{⇒} \AgdaBound{to} \AgdaInductiveConstructor{⟩∷} \AgdaBound{a}\AgdaSymbol{)} \AgdaSymbol{(}\AgdaInductiveConstructor{⊥∷} \AgdaBound{b}\AgdaSymbol{)} \AgdaSymbol{(}\AgdaInductiveConstructor{⊥∥⊥} \AgdaBound{c'∥c''}\AgdaSymbol{)} \AgdaSymbol{(}\AgdaInductiveConstructor{⊤∥⊥} \AgdaBound{a∥b}\AgdaSymbol{)} \<[78]%
\>[78]\<%
\\
\>[4]\AgdaIndent{6}{}\<[6]%
\>[6]\AgdaSymbol{(}\AgdaInductiveConstructor{⊤⋙?⊥} \AgdaSymbol{.}\AgdaBound{from} \AgdaSymbol{.}\AgdaBound{to} \AgdaBound{a∧b⋙?c}\AgdaSymbol{)} \AgdaSymbol{(}\AgdaInductiveConstructor{⊤⋙?⊥} \AgdaSymbol{.}\AgdaBound{from} \AgdaSymbol{.}\AgdaBound{to} \AgdaBound{a⋙?c'}\AgdaSymbol{)} \AgdaSymbol{(}\AgdaInductiveConstructor{⊥⋙?⊥} \AgdaBound{b⋙c''}\AgdaSymbol{)} \AgdaSymbol{=}\<%
\\
\>[4]\AgdaIndent{6}{}\<[6]%
\>[6]\AgdaSymbol{(}\AgdaFunction{⟶-prepend-⊤⊤} \AgdaBound{from} \AgdaBound{to} \AgdaSymbol{(}\AgdaField{fst} \AgdaFunction{p}\AgdaSymbol{))} \AgdaInductiveConstructor{,} \AgdaSymbol{(}\AgdaFunction{⟶-prepend-⊤⊤} \AgdaBound{from} \AgdaBound{to} \AgdaSymbol{(}\AgdaField{snd} \AgdaFunction{p}\AgdaSymbol{))} \AgdaKeyword{where}\<%
\\
\>[4]\AgdaIndent{6}{}\<[6]%
\>[6]\AgdaFunction{p} \AgdaSymbol{=} \AgdaFunction{∧-⋙-lem} \AgdaBound{c'} \AgdaBound{c''} \AgdaBound{a} \AgdaBound{b} \AgdaBound{c'∥c''} \AgdaBound{a∥b} \AgdaBound{a∧b⋙?c} \AgdaBound{a⋙?c'} \AgdaBound{b⋙c''}\<%
\\
\>[0]\AgdaIndent{4}{}\<[4]%
\>[4]\AgdaFunction{∧-⋙-lem} \AgdaSymbol{(}\AgdaInductiveConstructor{⊥∷} \AgdaBound{c'}\AgdaSymbol{)} \AgdaSymbol{(}\AgdaInductiveConstructor{⟨} \AgdaBound{from} \AgdaInductiveConstructor{⇒} \AgdaBound{to} \AgdaInductiveConstructor{⟩∷} \AgdaBound{c''}\AgdaSymbol{)} \AgdaSymbol{(}\AgdaInductiveConstructor{⊥∷} \AgdaBound{a}\AgdaSymbol{)} \AgdaSymbol{(}\AgdaInductiveConstructor{⟨} \AgdaBound{from₁} \AgdaInductiveConstructor{⇒} \AgdaSymbol{.}\AgdaBound{from} \AgdaInductiveConstructor{⟩∷} \AgdaBound{b}\AgdaSymbol{)} \<[71]%
\>[71]\<%
\\
\>[4]\AgdaIndent{6}{}\<[6]%
\>[6]\AgdaSymbol{(}\AgdaInductiveConstructor{⊥∥⊤} \AgdaBound{c'∥c''}\AgdaSymbol{)} \AgdaSymbol{(}\AgdaInductiveConstructor{⊥∥⊤} \AgdaBound{a∥b}\AgdaSymbol{)} \AgdaSymbol{(}\AgdaInductiveConstructor{⊤⋙?⊤} \AgdaSymbol{.}\AgdaBound{from₁} \AgdaSymbol{.}\AgdaBound{from} \AgdaSymbol{.}\AgdaBound{to} \AgdaBound{a∧b⋙?c}\AgdaSymbol{)} \AgdaSymbol{(}\AgdaInductiveConstructor{⊥⋙?⊥} \AgdaBound{a⋙?c'}\AgdaSymbol{)} \<[73]%
\>[73]\<%
\\
\>[4]\AgdaIndent{6}{}\<[6]%
\>[6]\AgdaSymbol{(}\AgdaInductiveConstructor{⊤⋙?⊤} \AgdaSymbol{.}\AgdaBound{from₁} \AgdaSymbol{.}\AgdaBound{from} \AgdaSymbol{.}\AgdaBound{to} \AgdaBound{b⋙c''}\AgdaSymbol{)} \AgdaSymbol{=} \<[38]%
\>[38]\<%
\\
\>[4]\AgdaIndent{6}{}\<[6]%
\>[6]\AgdaSymbol{(}\AgdaFunction{⟶-prepend-⊤⊤} \AgdaBound{from₁} \AgdaBound{to} \AgdaSymbol{(}\AgdaField{fst} \AgdaFunction{p}\AgdaSymbol{))} \AgdaInductiveConstructor{,} \AgdaFunction{⟶-prepend-⊤⊤} \AgdaBound{from₁} \AgdaBound{to} \AgdaSymbol{(}\AgdaField{snd} \AgdaFunction{p}\AgdaSymbol{)} \AgdaKeyword{where}\<%
\\
\>[4]\AgdaIndent{6}{}\<[6]%
\>[6]\AgdaFunction{p} \AgdaSymbol{=} \AgdaFunction{∧-⋙-lem} \AgdaBound{c'} \AgdaBound{c''} \AgdaBound{a} \AgdaBound{b} \AgdaBound{c'∥c''} \AgdaBound{a∥b} \AgdaBound{a∧b⋙?c} \AgdaBound{a⋙?c'} \AgdaBound{b⋙c''}\<%
\\
\>[0]\AgdaIndent{4}{}\<[4]%
\>[4]\AgdaFunction{∧-⋙-lem} \AgdaSymbol{(}\AgdaInductiveConstructor{⟨} \AgdaBound{from} \AgdaInductiveConstructor{⇒} \AgdaBound{to} \AgdaInductiveConstructor{⟩∷} \AgdaBound{c'}\AgdaSymbol{)} \AgdaSymbol{(}\AgdaInductiveConstructor{⊥∷} \AgdaBound{c''}\AgdaSymbol{)} \AgdaSymbol{(}\AgdaInductiveConstructor{⟨} \AgdaBound{from₁} \AgdaInductiveConstructor{⇒} \AgdaSymbol{.}\AgdaBound{from} \AgdaInductiveConstructor{⟩∷} \AgdaBound{a}\AgdaSymbol{)} \AgdaSymbol{(}\AgdaInductiveConstructor{⊥∷} \AgdaBound{b}\AgdaSymbol{)} \<[71]%
\>[71]\<%
\\
\>[4]\AgdaIndent{6}{}\<[6]%
\>[6]\AgdaSymbol{(}\AgdaInductiveConstructor{⊤∥⊥} \AgdaBound{c'∥c''}\AgdaSymbol{)} \AgdaSymbol{(}\AgdaInductiveConstructor{⊤∥⊥} \AgdaBound{a∥b}\AgdaSymbol{)} \AgdaSymbol{(}\AgdaInductiveConstructor{⊤⋙?⊤} \AgdaSymbol{.}\AgdaBound{from₁} \AgdaSymbol{.}\AgdaBound{from} \AgdaSymbol{.}\AgdaBound{to} \AgdaBound{a∧b⋙?c}\AgdaSymbol{)} \<[60]%
\>[60]\<%
\\
\>[4]\AgdaIndent{6}{}\<[6]%
\>[6]\AgdaSymbol{(}\AgdaInductiveConstructor{⊤⋙?⊤} \AgdaSymbol{.}\AgdaBound{from₁} \AgdaSymbol{.}\AgdaBound{from} \AgdaSymbol{.}\AgdaBound{to} \AgdaBound{a⋙?c'}\AgdaSymbol{)} \AgdaSymbol{(}\AgdaInductiveConstructor{⊥⋙?⊥} \AgdaBound{b⋙c''}\AgdaSymbol{)} \AgdaSymbol{=}\<%
\\
\>[4]\AgdaIndent{6}{}\<[6]%
\>[6]\AgdaSymbol{(}\AgdaFunction{⟶-prepend-⊤⊤} \AgdaBound{from₁} \AgdaBound{to} \AgdaSymbol{(}\AgdaField{fst} \AgdaFunction{p}\AgdaSymbol{))} \AgdaInductiveConstructor{,} \AgdaFunction{⟶-prepend-⊤⊤} \AgdaBound{from₁} \AgdaBound{to} \AgdaSymbol{(}\AgdaField{snd} \AgdaFunction{p}\AgdaSymbol{)} \AgdaKeyword{where}\<%
\\
\>[4]\AgdaIndent{6}{}\<[6]%
\>[6]\AgdaFunction{p} \AgdaSymbol{=} \AgdaFunction{∧-⋙-lem} \AgdaBound{c'} \AgdaBound{c''} \AgdaBound{a} \AgdaBound{b} \AgdaBound{c'∥c''} \AgdaBound{a∥b} \AgdaBound{a∧b⋙?c} \AgdaBound{a⋙?c'} \AgdaBound{b⋙c''}\<%
\\
%
\\
\>[0]\AgdaIndent{4}{}\<[4]%
\>[4]\AgdaFunction{⋙-assoc} \AgdaSymbol{:} \AgdaSymbol{∀} \AgdaSymbol{\{}\AgdaBound{n}\AgdaSymbol{\}\{}\AgdaBound{f₁} \AgdaBound{f₂} \AgdaBound{f₃} \AgdaSymbol{:} \AgdaFunction{Form} \AgdaBound{n}\AgdaSymbol{\}\{}\AgdaBound{p₁} \AgdaSymbol{:} \AgdaDatatype{Patch} \AgdaBound{f₁}\AgdaSymbol{\}\{}\AgdaBound{p₂} \AgdaSymbol{:} \AgdaDatatype{Patch} \AgdaBound{f₂}\AgdaSymbol{\}\{}\AgdaBound{p₃} \AgdaSymbol{:} \AgdaDatatype{Patch} \AgdaBound{f₃}\AgdaSymbol{\}}\<%
\\
\>[4]\AgdaIndent{6}{}\<[6]%
\>[6]\AgdaSymbol{→} \AgdaSymbol{(}\AgdaBound{p₁>>?p₂} \AgdaSymbol{:} \AgdaBound{p₁} \AgdaDatatype{⋙?} \AgdaBound{p₂}\AgdaSymbol{)}\<%
\\
\>[4]\AgdaIndent{6}{}\<[6]%
\>[6]\AgdaSymbol{→} \AgdaSymbol{(}\AgdaBound{[p₁>>ₚp₂]>>?p₃} \AgdaSymbol{:} \AgdaSymbol{(}\AgdaBound{p₁} \AgdaFunction{⋙ₚ} \AgdaBound{p₂}\AgdaSymbol{)} \AgdaBound{p₁>>?p₂} \AgdaDatatype{⋙?} \AgdaBound{p₃}\AgdaSymbol{)}\<%
\\
\>[4]\AgdaIndent{6}{}\<[6]%
\>[6]\AgdaSymbol{→} \AgdaSymbol{(}\AgdaBound{p₂>>?p₃} \AgdaSymbol{:} \AgdaBound{p₂} \AgdaDatatype{⋙?} \AgdaBound{p₃}\AgdaSymbol{)}\<%
\\
\>[4]\AgdaIndent{6}{}\<[6]%
\>[6]\AgdaSymbol{→} \AgdaSymbol{(}\AgdaBound{p₁>>?[p₂>>ₚp₃]} \AgdaSymbol{:} \AgdaBound{p₁} \AgdaDatatype{⋙?} \AgdaSymbol{(}\AgdaBound{p₂} \AgdaFunction{⋙ₚ} \AgdaBound{p₃}\AgdaSymbol{)} \AgdaBound{p₂>>?p₃}\AgdaSymbol{)}\<%
\\
\>[4]\AgdaIndent{6}{}\<[6]%
\>[6]\AgdaSymbol{→} \AgdaSymbol{(((}\AgdaBound{p₁} \AgdaFunction{⋙ₚ} \AgdaBound{p₂}\AgdaSymbol{)} \AgdaBound{p₁>>?p₂}\AgdaSymbol{)} \AgdaFunction{⋙ₚ} \AgdaBound{p₃}\AgdaSymbol{)} \AgdaBound{[p₁>>ₚp₂]>>?p₃}\<%
\\
\>[6]\AgdaIndent{8}{}\<[8]%
\>[8]\AgdaFunction{⟷}\<%
\\
\>[6]\AgdaIndent{8}{}\<[8]%
\>[8]\AgdaSymbol{(}\AgdaBound{p₁} \AgdaFunction{⋙ₚ} \AgdaSymbol{(}\AgdaBound{p₂} \AgdaFunction{⋙ₚ} \AgdaBound{p₃}\AgdaSymbol{)} \AgdaBound{p₂>>?p₃}\AgdaSymbol{)} \AgdaBound{p₁>>?[p₂>>ₚp₃]}\<%
\\
\>[0]\AgdaIndent{4}{}\<[4]%
\>[4]\AgdaFunction{⋙-assoc} \AgdaInductiveConstructor{O⋙?O} \AgdaInductiveConstructor{O⋙?O} \AgdaInductiveConstructor{O⋙?O} \AgdaInductiveConstructor{O⋙?O} \AgdaSymbol{=} \<[34]%
\>[34]\<%
\\
\>[4]\AgdaIndent{6}{}\<[6]%
\>[6]\AgdaSymbol{(λ} \AgdaBound{x} \AgdaBound{x₁} \AgdaSymbol{→} \AgdaBound{x₁} \AgdaInductiveConstructor{,} \AgdaInductiveConstructor{refl}\AgdaSymbol{)} \AgdaInductiveConstructor{,} \AgdaSymbol{(λ} \AgdaBound{x} \AgdaBound{x₁} \AgdaSymbol{→} \AgdaBound{x₁} \AgdaInductiveConstructor{,} \AgdaInductiveConstructor{refl}\AgdaSymbol{)}\<%
\\
\>[0]\AgdaIndent{4}{}\<[4]%
\>[4]\AgdaFunction{⋙-assoc} \AgdaSymbol{(}\AgdaInductiveConstructor{⊥⋙?⊥} \AgdaBound{p₁⋙?p₂}\AgdaSymbol{)} \AgdaSymbol{(}\AgdaInductiveConstructor{⊥⋙?⊥} \AgdaBound{[p₁⋙p₂]⋙?p₃}\AgdaSymbol{)} \<[45]%
\>[45]\<%
\\
\>[4]\AgdaIndent{6}{}\<[6]%
\>[6]\AgdaSymbol{(}\AgdaInductiveConstructor{⊥⋙?⊥} \AgdaBound{p₂⋙?p₃}\AgdaSymbol{)} \AgdaSymbol{(}\AgdaInductiveConstructor{⊥⋙?⊥} \AgdaBound{p₁⋙?[p₂⋙p₃]}\AgdaSymbol{)} \AgdaSymbol{=} \<[41]%
\>[41]\<%
\\
\>[4]\AgdaIndent{6}{}\<[6]%
\>[6]\AgdaSymbol{(}\AgdaFunction{⟶-prepend-⊥⊥} \AgdaSymbol{(}\AgdaField{fst} \AgdaFunction{p}\AgdaSymbol{))} \AgdaInductiveConstructor{,} \AgdaSymbol{(}\AgdaFunction{⟶-prepend-⊥⊥} \AgdaSymbol{(}\AgdaField{snd} \AgdaFunction{p}\AgdaSymbol{))} \AgdaKeyword{where}\<%
\\
\>[4]\AgdaIndent{6}{}\<[6]%
\>[6]\AgdaFunction{p} \AgdaSymbol{=} \AgdaFunction{⋙-assoc} \AgdaBound{p₁⋙?p₂} \AgdaBound{[p₁⋙p₂]⋙?p₃} \AgdaBound{p₂⋙?p₃} \AgdaBound{p₁⋙?[p₂⋙p₃]}\<%
\\
\>[0]\AgdaIndent{4}{}\<[4]%
\>[4]\AgdaFunction{⋙-assoc} \AgdaSymbol{(}\AgdaInductiveConstructor{⊤⋙?⊥} \AgdaBound{from} \AgdaBound{to} \AgdaBound{p₁⋙?p₂}\AgdaSymbol{)} \AgdaSymbol{(}\AgdaInductiveConstructor{⊤⋙?⊥} \AgdaSymbol{.}\AgdaBound{from} \AgdaSymbol{.}\AgdaBound{to} \AgdaBound{[p₁⋙p₂]⋙?p₃}\AgdaSymbol{)} \<[63]%
\>[63]\<%
\\
\>[4]\AgdaIndent{6}{}\<[6]%
\>[6]\AgdaSymbol{(}\AgdaInductiveConstructor{⊥⋙?⊥} \AgdaBound{p₂⋙?p₃}\AgdaSymbol{)} \AgdaSymbol{(}\AgdaInductiveConstructor{⊤⋙?⊥} \AgdaSymbol{.}\AgdaBound{from} \AgdaSymbol{.}\AgdaBound{to} \AgdaBound{p₁⋙?[p₂⋙p₃]}\AgdaSymbol{)} \AgdaSymbol{=} \<[51]%
\>[51]\<%
\\
\>[4]\AgdaIndent{6}{}\<[6]%
\>[6]\AgdaSymbol{(}\AgdaFunction{⟶-prepend-⊤⊤} \AgdaBound{from} \AgdaBound{to} \AgdaSymbol{(}\AgdaField{fst} \AgdaFunction{p}\AgdaSymbol{))} \AgdaInductiveConstructor{,} \AgdaFunction{⟶-prepend-⊤⊤} \AgdaBound{from} \AgdaBound{to} \AgdaSymbol{(}\AgdaField{snd} \AgdaFunction{p}\AgdaSymbol{)} \AgdaKeyword{where}\<%
\\
\>[4]\AgdaIndent{6}{}\<[6]%
\>[6]\AgdaFunction{p} \AgdaSymbol{=} \AgdaFunction{⋙-assoc} \AgdaBound{p₁⋙?p₂} \AgdaBound{[p₁⋙p₂]⋙?p₃} \AgdaBound{p₂⋙?p₃} \AgdaBound{p₁⋙?[p₂⋙p₃]}\<%
\\
\>[0]\AgdaIndent{4}{}\<[4]%
\>[4]\AgdaFunction{⋙-assoc} \AgdaSymbol{(}\AgdaInductiveConstructor{⊤⋙?⊤} \AgdaBound{from} \AgdaBound{to} \AgdaBound{to₂} \AgdaBound{p₁⋙?p₂}\AgdaSymbol{)} \AgdaSymbol{(}\AgdaInductiveConstructor{⊤⋙?⊥} \AgdaSymbol{.}\AgdaBound{from} \AgdaSymbol{.}\AgdaBound{to₂} \AgdaBound{[p₁⋙p₂]⋙?p₃}\AgdaSymbol{)} \<[68]%
\>[68]\<%
\\
\>[4]\AgdaIndent{6}{}\<[6]%
\>[6]\AgdaSymbol{(}\AgdaInductiveConstructor{⊤⋙?⊥} \AgdaSymbol{.}\AgdaBound{to} \AgdaSymbol{.}\AgdaBound{to₂} \AgdaBound{p₂⋙?p₃}\AgdaSymbol{)} \AgdaSymbol{(}\AgdaInductiveConstructor{⊤⋙?⊤} \AgdaSymbol{.}\AgdaBound{from} \AgdaSymbol{.}\AgdaBound{to} \AgdaSymbol{.}\AgdaBound{to₂} \AgdaBound{p₁⋙?[p₂⋙p₃]}\AgdaSymbol{)} \AgdaSymbol{=} \<[65]%
\>[65]\<%
\\
\>[4]\AgdaIndent{6}{}\<[6]%
\>[6]\AgdaSymbol{(}\AgdaFunction{⟶-prepend-⊤⊤} \AgdaBound{from} \AgdaBound{to₂} \AgdaSymbol{(}\AgdaField{fst} \AgdaFunction{p}\AgdaSymbol{))} \AgdaInductiveConstructor{,} \AgdaFunction{⟶-prepend-⊤⊤} \AgdaBound{from} \AgdaBound{to₂} \AgdaSymbol{(}\AgdaField{snd} \AgdaFunction{p}\AgdaSymbol{)} \AgdaKeyword{where}\<%
\\
\>[4]\AgdaIndent{6}{}\<[6]%
\>[6]\AgdaFunction{p} \AgdaSymbol{=} \AgdaFunction{⋙-assoc} \AgdaBound{p₁⋙?p₂} \AgdaBound{[p₁⋙p₂]⋙?p₃} \AgdaBound{p₂⋙?p₃} \AgdaBound{p₁⋙?[p₂⋙p₃]}\<%
\\
\>[0]\AgdaIndent{4}{}\<[4]%
\>[4]\AgdaFunction{⋙-assoc} \AgdaSymbol{(}\AgdaInductiveConstructor{⊤⋙?⊤} \AgdaBound{from} \AgdaBound{to} \AgdaBound{to₂} \AgdaBound{p₁⋙?p₂}\AgdaSymbol{)} \<[38]%
\>[38]\<%
\\
\>[4]\AgdaIndent{6}{}\<[6]%
\>[6]\AgdaSymbol{(}\AgdaInductiveConstructor{⊤⋙?⊤} \AgdaSymbol{.}\AgdaBound{from} \AgdaSymbol{.}\AgdaBound{to₂} \AgdaBound{to₃} \AgdaBound{[p₁⋙p₂]⋙?p₃}\AgdaSymbol{)} \<[40]%
\>[40]\<%
\\
\>[4]\AgdaIndent{6}{}\<[6]%
\>[6]\AgdaSymbol{(}\AgdaInductiveConstructor{⊤⋙?⊤} \AgdaSymbol{.}\AgdaBound{to} \AgdaSymbol{.}\AgdaBound{to₂} \AgdaSymbol{.}\AgdaBound{to₃} \AgdaBound{p₂⋙?p₃}\AgdaSymbol{)} \AgdaSymbol{(}\AgdaInductiveConstructor{⊤⋙?⊤} \AgdaSymbol{.}\AgdaBound{from} \AgdaSymbol{.}\AgdaBound{to} \AgdaSymbol{.}\AgdaBound{to₃} \AgdaBound{p₁⋙?[p₂⋙p₃]}\AgdaSymbol{)} \AgdaSymbol{=} \<[70]%
\>[70]\<%
\\
\>[4]\AgdaIndent{6}{}\<[6]%
\>[6]\AgdaSymbol{(}\AgdaFunction{⟶-prepend-⊤⊤} \AgdaBound{from} \AgdaBound{to₃} \AgdaSymbol{(}\AgdaField{fst} \AgdaFunction{p}\AgdaSymbol{))} \AgdaInductiveConstructor{,} \AgdaFunction{⟶-prepend-⊤⊤} \AgdaBound{from} \AgdaBound{to₃} \AgdaSymbol{(}\AgdaField{snd} \AgdaFunction{p}\AgdaSymbol{)} \AgdaKeyword{where}\<%
\\
\>[4]\AgdaIndent{6}{}\<[6]%
\>[6]\AgdaFunction{p} \AgdaSymbol{=} \AgdaFunction{⋙-assoc} \AgdaBound{p₁⋙?p₂} \AgdaBound{[p₁⋙p₂]⋙?p₃} \AgdaBound{p₂⋙?p₃} \AgdaBound{p₁⋙?[p₂⋙p₃]}\<%
\\
%
\\
\>[0]\AgdaIndent{2}{}\<[2]%
\>[2]\AgdaFunction{drop} \AgdaSymbol{:} \AgdaSymbol{∀} \AgdaSymbol{\{}\AgdaBound{n}\AgdaSymbol{\}} \AgdaSymbol{(}\AgdaBound{b₁} \AgdaBound{b₂} \AgdaSymbol{:} \AgdaDatatype{Bool}\AgdaSymbol{)} \AgdaSymbol{(}\AgdaBound{f₁} \AgdaBound{f₂} \AgdaSymbol{:} \AgdaFunction{Form} \AgdaBound{n}\AgdaSymbol{)}\<%
\\
\>[2]\AgdaIndent{4}{}\<[4]%
\>[4]\AgdaSymbol{→} \AgdaSymbol{(}\AgdaBound{b₁} \AgdaInductiveConstructor{∷} \AgdaBound{f₁}\AgdaSymbol{)} \AgdaDatatype{∥} \AgdaSymbol{(}\AgdaBound{b₂} \AgdaInductiveConstructor{∷} \AgdaBound{f₂}\AgdaSymbol{)} \AgdaSymbol{→} \AgdaBound{f₁} \AgdaDatatype{∥} \AgdaBound{f₂}\<%
\\
\>[0]\AgdaIndent{2}{}\<[2]%
\>[2]\AgdaFunction{drop} \AgdaSymbol{.}\AgdaInductiveConstructor{false} \AgdaSymbol{.}\AgdaInductiveConstructor{false} \AgdaBound{f₃} \AgdaBound{f₄} \AgdaSymbol{(}\AgdaInductiveConstructor{⊥∥⊥} \AgdaBound{p}\AgdaSymbol{)} \AgdaSymbol{=} \AgdaBound{p}\<%
\\
\>[0]\AgdaIndent{2}{}\<[2]%
\>[2]\AgdaFunction{drop} \AgdaSymbol{.}\AgdaInductiveConstructor{true} \AgdaSymbol{.}\AgdaInductiveConstructor{false} \AgdaBound{f₃} \AgdaBound{f₄} \AgdaSymbol{(}\AgdaInductiveConstructor{⊤∥⊥} \AgdaBound{p}\AgdaSymbol{)} \AgdaSymbol{=} \AgdaBound{p}\<%
\\
\>[0]\AgdaIndent{2}{}\<[2]%
\>[2]\AgdaFunction{drop} \AgdaSymbol{.}\AgdaInductiveConstructor{false} \AgdaSymbol{.}\AgdaInductiveConstructor{true} \AgdaBound{f₃} \AgdaBound{f₄} \AgdaSymbol{(}\AgdaInductiveConstructor{⊥∥⊤} \AgdaBound{p}\AgdaSymbol{)} \AgdaSymbol{=} \AgdaBound{p}\<%
\\
%
\\
\>[0]\AgdaIndent{2}{}\<[2]%
\>[2]\AgdaFunction{∥-dec} \AgdaSymbol{:} \AgdaSymbol{∀} \AgdaSymbol{\{}\AgdaBound{n}\AgdaSymbol{\}} \AgdaSymbol{(}\AgdaBound{f₁} \AgdaBound{f₂} \AgdaSymbol{:} \AgdaFunction{Form} \AgdaBound{n}\AgdaSymbol{)} \AgdaSymbol{→} \AgdaDatatype{Dec} \AgdaSymbol{(}\AgdaBound{f₁} \AgdaDatatype{∥} \AgdaBound{f₂}\AgdaSymbol{)}\<%
\\
\>[0]\AgdaIndent{2}{}\<[2]%
\>[2]\AgdaFunction{∥-dec} \AgdaInductiveConstructor{[]} \AgdaInductiveConstructor{[]} \AgdaSymbol{=} \AgdaInductiveConstructor{yes} \AgdaInductiveConstructor{[]∥[]}\<%
\\
\>[0]\AgdaIndent{2}{}\<[2]%
\>[2]\AgdaFunction{∥-dec} \AgdaSymbol{(}\AgdaInductiveConstructor{true} \AgdaInductiveConstructor{∷} \AgdaBound{f₁}\AgdaSymbol{)} \AgdaSymbol{(}\AgdaInductiveConstructor{true} \AgdaInductiveConstructor{∷} \AgdaBound{f₂}\AgdaSymbol{)} \AgdaSymbol{=} \AgdaInductiveConstructor{no} \AgdaSymbol{(}\AgdaFunction{no-tt} \AgdaBound{f₁} \AgdaBound{f₂}\AgdaSymbol{)} \AgdaKeyword{where}\<%
\\
\>[2]\AgdaIndent{4}{}\<[4]%
\>[4]\AgdaFunction{no-tt} \AgdaSymbol{:} \AgdaSymbol{∀} \AgdaSymbol{\{}\AgdaBound{n}\AgdaSymbol{\}} \AgdaSymbol{(}\AgdaBound{f₁} \AgdaBound{f₂} \AgdaSymbol{:} \AgdaFunction{Form} \AgdaBound{n}\AgdaSymbol{)} \AgdaSymbol{→} \AgdaFunction{¬} \AgdaSymbol{((}\AgdaInductiveConstructor{true} \AgdaInductiveConstructor{∷} \AgdaBound{f₁}\AgdaSymbol{)} \AgdaDatatype{∥} \AgdaSymbol{(}\AgdaInductiveConstructor{true} \AgdaInductiveConstructor{∷} \AgdaBound{f₂}\AgdaSymbol{))}\<%
\\
\>[2]\AgdaIndent{4}{}\<[4]%
\>[4]\AgdaFunction{no-tt} \AgdaSymbol{\_} \AgdaSymbol{\_} \AgdaSymbol{()}\<%
\\
\>[0]\AgdaIndent{2}{}\<[2]%
\>[2]\AgdaFunction{∥-dec} \AgdaSymbol{(}\AgdaInductiveConstructor{true} \AgdaInductiveConstructor{∷} \AgdaBound{f₁}\AgdaSymbol{)} \AgdaSymbol{(}\AgdaInductiveConstructor{false} \AgdaInductiveConstructor{∷} \AgdaBound{f₂}\AgdaSymbol{)} \AgdaKeyword{with} \AgdaFunction{∥-dec} \AgdaBound{f₁} \AgdaBound{f₂}\<%
\\
\>[0]\AgdaIndent{2}{}\<[2]%
\>[2]\AgdaSymbol{...} \AgdaSymbol{|} \AgdaInductiveConstructor{yes} \AgdaBound{a} \AgdaSymbol{=} \AgdaInductiveConstructor{yes} \AgdaSymbol{(}\AgdaInductiveConstructor{⊤∥⊥} \AgdaBound{a}\AgdaSymbol{)}\<%
\\
\>[0]\AgdaIndent{2}{}\<[2]%
\>[2]\AgdaSymbol{...} \AgdaSymbol{|} \AgdaInductiveConstructor{no} \AgdaBound{¬a} \AgdaSymbol{=} \AgdaInductiveConstructor{no} \AgdaSymbol{(}\AgdaBound{¬a} \AgdaFunction{∘} \AgdaFunction{drop} \AgdaInductiveConstructor{true} \AgdaInductiveConstructor{false} \AgdaBound{f₁} \AgdaBound{f₂}\AgdaSymbol{)} \AgdaKeyword{where}\<%
\\
\>[0]\AgdaIndent{2}{}\<[2]%
\>[2]\AgdaFunction{∥-dec} \AgdaSymbol{(}\AgdaInductiveConstructor{false} \AgdaInductiveConstructor{∷} \AgdaBound{f₁}\AgdaSymbol{)} \AgdaSymbol{(}\AgdaInductiveConstructor{true} \AgdaInductiveConstructor{∷} \AgdaBound{f₂}\AgdaSymbol{)} \AgdaKeyword{with} \AgdaFunction{∥-dec} \AgdaBound{f₁} \AgdaBound{f₂}\<%
\\
\>[0]\AgdaIndent{2}{}\<[2]%
\>[2]\AgdaSymbol{...} \AgdaSymbol{|} \AgdaInductiveConstructor{yes} \AgdaBound{a} \AgdaSymbol{=} \AgdaInductiveConstructor{yes} \AgdaSymbol{(}\AgdaInductiveConstructor{⊥∥⊤} \AgdaBound{a}\AgdaSymbol{)}\<%
\\
\>[0]\AgdaIndent{2}{}\<[2]%
\>[2]\AgdaSymbol{...} \AgdaSymbol{|} \AgdaInductiveConstructor{no} \AgdaBound{¬a} \AgdaSymbol{=} \AgdaInductiveConstructor{no} \AgdaSymbol{(}\AgdaBound{¬a} \AgdaFunction{∘} \AgdaFunction{drop} \AgdaInductiveConstructor{false} \AgdaInductiveConstructor{true} \AgdaBound{f₁} \AgdaBound{f₂}\AgdaSymbol{)}\<%
\\
\>[0]\AgdaIndent{2}{}\<[2]%
\>[2]\AgdaFunction{∥-dec} \AgdaSymbol{(}\AgdaInductiveConstructor{false} \AgdaInductiveConstructor{∷} \AgdaBound{f₁}\AgdaSymbol{)} \AgdaSymbol{(}\AgdaInductiveConstructor{false} \AgdaInductiveConstructor{∷} \AgdaBound{f₂}\AgdaSymbol{)} \AgdaKeyword{with} \AgdaFunction{∥-dec} \AgdaBound{f₁} \AgdaBound{f₂}\<%
\\
\>[0]\AgdaIndent{2}{}\<[2]%
\>[2]\AgdaSymbol{...} \AgdaSymbol{|} \AgdaInductiveConstructor{yes} \AgdaBound{a} \AgdaSymbol{=} \AgdaInductiveConstructor{yes} \AgdaSymbol{(}\AgdaInductiveConstructor{⊥∥⊥} \AgdaBound{a}\AgdaSymbol{)}\<%
\\
\>[0]\AgdaIndent{2}{}\<[2]%
\>[2]\AgdaSymbol{...} \AgdaSymbol{|} \AgdaInductiveConstructor{no} \AgdaBound{¬a} \AgdaSymbol{=} \AgdaInductiveConstructor{no} \AgdaSymbol{(}\AgdaBound{¬a} \AgdaFunction{∘} \AgdaFunction{drop} \AgdaInductiveConstructor{false} \AgdaInductiveConstructor{false} \AgdaBound{f₁} \AgdaBound{f₂}\AgdaSymbol{)}\<%
\\
\>[0]\AgdaIndent{2}{}\<[2]%
\>[2]\<%
\\
\>[0]\AgdaIndent{2}{}\<[2]%
\>[2]\AgdaFunction{⋙?-dec} \AgdaSymbol{:} \AgdaSymbol{∀} \AgdaSymbol{\{}\AgdaBound{n}\AgdaSymbol{\}} \AgdaSymbol{\{}\AgdaBound{f₁} \AgdaBound{f₂} \AgdaSymbol{:} \AgdaFunction{Form} \AgdaBound{n}\AgdaSymbol{\}} \AgdaSymbol{(}\AgdaBound{p₁} \AgdaSymbol{:} \AgdaDatatype{Patch} \AgdaBound{f₁}\AgdaSymbol{)} \AgdaSymbol{(}\AgdaBound{p₂} \AgdaSymbol{:} \AgdaDatatype{Patch} \AgdaBound{f₂}\AgdaSymbol{)}\<%
\\
\>[2]\AgdaIndent{4}{}\<[4]%
\>[4]\AgdaSymbol{→} \AgdaDatatype{Dec} \AgdaSymbol{(}\AgdaBound{p₁} \AgdaDatatype{⋙?} \AgdaBound{p₂}\AgdaSymbol{)}\<%
\\
\>[0]\AgdaIndent{2}{}\<[2]%
\>[2]\AgdaFunction{⋙?-dec} \AgdaInductiveConstructor{O} \AgdaInductiveConstructor{O} \AgdaSymbol{=} \AgdaInductiveConstructor{yes} \AgdaInductiveConstructor{O⋙?O}\<%
\\
\>[0]\AgdaIndent{2}{}\<[2]%
\>[2]\AgdaFunction{⋙?-dec} \AgdaSymbol{(}\AgdaInductiveConstructor{⊥∷} \AgdaBound{p₁}\AgdaSymbol{)} \AgdaSymbol{(}\AgdaInductiveConstructor{⊥∷} \AgdaBound{p₂}\AgdaSymbol{)} \AgdaKeyword{with} \AgdaFunction{⋙?-dec} \AgdaBound{p₁} \AgdaBound{p₂} \<[43]%
\>[43]\<%
\\
\>[0]\AgdaIndent{2}{}\<[2]%
\>[2]\AgdaSymbol{...} \AgdaSymbol{|} \AgdaInductiveConstructor{yes} \AgdaBound{a} \AgdaSymbol{=} \AgdaInductiveConstructor{yes} \AgdaSymbol{(}\AgdaInductiveConstructor{⊥⋙?⊥} \AgdaBound{a}\AgdaSymbol{)}\<%
\\
\>[0]\AgdaIndent{2}{}\<[2]%
\>[2]\AgdaSymbol{...} \AgdaSymbol{|} \AgdaInductiveConstructor{no} \AgdaBound{¬a} \AgdaSymbol{=} \AgdaInductiveConstructor{no} \AgdaSymbol{(}\AgdaBound{¬a} \AgdaFunction{∘} \AgdaFunction{⊥⊥-drop}\AgdaSymbol{)} \AgdaKeyword{where}\<%
\\
\>[2]\AgdaIndent{4}{}\<[4]%
\>[4]\AgdaFunction{⊥⊥-drop} \AgdaSymbol{:} \AgdaInductiveConstructor{⊥∷} \AgdaBound{p₁} \AgdaDatatype{⋙?} \AgdaInductiveConstructor{⊥∷} \AgdaBound{p₂} \AgdaSymbol{→} \AgdaBound{p₁} \AgdaDatatype{⋙?} \AgdaBound{p₂}\<%
\\
\>[2]\AgdaIndent{4}{}\<[4]%
\>[4]\AgdaFunction{⊥⊥-drop} \AgdaSymbol{(}\AgdaInductiveConstructor{⊥⋙?⊥} \AgdaBound{p}\AgdaSymbol{)} \AgdaSymbol{=} \AgdaBound{p}\<%
\\
\>[0]\AgdaIndent{2}{}\<[2]%
\>[2]\AgdaFunction{⋙?-dec} \AgdaSymbol{(}\AgdaInductiveConstructor{⊥∷} \AgdaBound{p₁}\AgdaSymbol{)} \AgdaSymbol{(}\AgdaInductiveConstructor{⟨} \AgdaBound{from} \AgdaInductiveConstructor{⇒} \AgdaBound{to} \AgdaInductiveConstructor{⟩∷} \AgdaBound{p₂}\AgdaSymbol{)} \AgdaSymbol{=} \AgdaInductiveConstructor{no} \AgdaFunction{fail} \AgdaKeyword{where}\<%
\\
\>[2]\AgdaIndent{4}{}\<[4]%
\>[4]\AgdaFunction{fail} \AgdaSymbol{:} \AgdaInductiveConstructor{⊥∷} \AgdaBound{p₁} \AgdaDatatype{⋙?} \AgdaSymbol{(}\AgdaInductiveConstructor{⟨} \AgdaBound{from} \AgdaInductiveConstructor{⇒} \AgdaBound{to} \AgdaInductiveConstructor{⟩∷} \AgdaBound{p₂}\AgdaSymbol{)} \AgdaSymbol{→} \AgdaDatatype{⊥}\<%
\\
\>[2]\AgdaIndent{4}{}\<[4]%
\>[4]\AgdaFunction{fail} \AgdaSymbol{()}\<%
\\
\>[0]\AgdaIndent{2}{}\<[2]%
\>[2]\AgdaFunction{⋙?-dec} \AgdaSymbol{(}\AgdaInductiveConstructor{⟨} \AgdaBound{from} \AgdaInductiveConstructor{⇒} \AgdaBound{to} \AgdaInductiveConstructor{⟩∷} \AgdaBound{p₁}\AgdaSymbol{)} \AgdaSymbol{(}\AgdaInductiveConstructor{⊥∷} \AgdaBound{p₂}\AgdaSymbol{)} \AgdaKeyword{with} \AgdaFunction{⋙?-dec} \AgdaBound{p₁} \AgdaBound{p₂}\<%
\\
\>[0]\AgdaIndent{2}{}\<[2]%
\>[2]\AgdaSymbol{...} \AgdaSymbol{|} \AgdaInductiveConstructor{yes} \AgdaBound{a} \AgdaSymbol{=} \AgdaInductiveConstructor{yes} \AgdaSymbol{(}\AgdaInductiveConstructor{⊤⋙?⊥} \AgdaBound{from} \AgdaBound{to} \AgdaBound{a}\AgdaSymbol{)}\<%
\\
\>[0]\AgdaIndent{2}{}\<[2]%
\>[2]\AgdaSymbol{...} \AgdaSymbol{|} \AgdaInductiveConstructor{no} \AgdaBound{¬a} \AgdaSymbol{=} \AgdaInductiveConstructor{no} \AgdaSymbol{(}\AgdaBound{¬a} \AgdaFunction{∘} \AgdaFunction{⊤⊥-drop}\AgdaSymbol{)} \AgdaKeyword{where}\<%
\\
\>[2]\AgdaIndent{4}{}\<[4]%
\>[4]\AgdaFunction{⊤⊥-drop} \AgdaSymbol{:} \AgdaSymbol{∀} \AgdaSymbol{\{}\AgdaBound{a} \AgdaBound{b} \AgdaSymbol{:} \AgdaBound{A}\AgdaSymbol{\}} \AgdaSymbol{→} \AgdaSymbol{(}\AgdaInductiveConstructor{⟨} \AgdaBound{a} \AgdaInductiveConstructor{⇒} \AgdaBound{b} \AgdaInductiveConstructor{⟩∷} \AgdaBound{p₁}\AgdaSymbol{)} \AgdaDatatype{⋙?} \AgdaInductiveConstructor{⊥∷} \AgdaBound{p₂} \AgdaSymbol{→} \AgdaBound{p₁} \AgdaDatatype{⋙?} \AgdaBound{p₂}\<%
\\
\>[2]\AgdaIndent{4}{}\<[4]%
\>[4]\AgdaFunction{⊤⊥-drop} \AgdaSymbol{(}\AgdaInductiveConstructor{⊤⋙?⊥} \AgdaSymbol{\_} \AgdaSymbol{\_} \AgdaBound{p}\AgdaSymbol{)} \AgdaSymbol{=} \AgdaBound{p}\<%
\\
\>[0]\AgdaIndent{2}{}\<[2]%
\>[2]\AgdaFunction{⋙?-dec} \AgdaSymbol{(}\AgdaInductiveConstructor{⟨} \AgdaBound{from₁} \AgdaInductiveConstructor{⇒} \AgdaBound{to₁} \AgdaInductiveConstructor{⟩∷} \AgdaBound{p₁}\AgdaSymbol{)} \AgdaSymbol{(}\AgdaInductiveConstructor{⟨} \AgdaBound{from₂} \AgdaInductiveConstructor{⇒} \AgdaBound{to₂} \AgdaInductiveConstructor{⟩∷} \AgdaBound{p₂}\AgdaSymbol{)} \AgdaKeyword{with} \AgdaFunction{⋙?-dec} \AgdaBound{p₁} \AgdaBound{p₂}\<%
\\
\>[0]\AgdaIndent{2}{}\<[2]%
\>[2]\AgdaSymbol{...} \AgdaSymbol{|} \AgdaInductiveConstructor{yes} \AgdaBound{a} \AgdaSymbol{=} \AgdaFunction{cmp} \AgdaSymbol{(}\AgdaBound{eqA} \AgdaBound{to₁} \AgdaBound{from₂}\AgdaSymbol{)} \AgdaKeyword{where}\<%
\\
\>[2]\AgdaIndent{4}{}\<[4]%
\>[4]\AgdaFunction{cmp} \AgdaSymbol{:} \AgdaDatatype{Dec} \AgdaSymbol{(}\AgdaBound{to₁} \AgdaDatatype{≡} \AgdaBound{from₂}\AgdaSymbol{)} \<[28]%
\>[28]\<%
\\
\>[4]\AgdaIndent{6}{}\<[6]%
\>[6]\AgdaSymbol{→} \AgdaDatatype{Dec} \AgdaSymbol{((}\AgdaInductiveConstructor{⟨} \AgdaBound{from₁} \AgdaInductiveConstructor{⇒} \AgdaBound{to₁} \AgdaInductiveConstructor{⟩∷} \AgdaBound{p₁}\AgdaSymbol{)} \AgdaDatatype{⋙?} \AgdaSymbol{(}\AgdaInductiveConstructor{⟨} \AgdaBound{from₂} \AgdaInductiveConstructor{⇒} \AgdaBound{to₂} \AgdaInductiveConstructor{⟩∷} \AgdaBound{p₂}\AgdaSymbol{))}\<%
\\
\>[0]\AgdaIndent{4}{}\<[4]%
\>[4]\AgdaFunction{cmp} \AgdaSymbol{(}\AgdaInductiveConstructor{yes} \AgdaBound{aa}\AgdaSymbol{)} \AgdaKeyword{rewrite} \AgdaBound{aa} \AgdaSymbol{=} \AgdaInductiveConstructor{yes} \AgdaSymbol{(}\AgdaInductiveConstructor{⊤⋙?⊤} \AgdaBound{from₁} \AgdaBound{from₂} \AgdaBound{to₂} \AgdaBound{a}\AgdaSymbol{)}\<%
\\
\>[0]\AgdaIndent{4}{}\<[4]%
\>[4]\AgdaFunction{cmp} \AgdaSymbol{(}\AgdaInductiveConstructor{no} \AgdaBound{¬a}\AgdaSymbol{)} \AgdaSymbol{=} \AgdaInductiveConstructor{no} \AgdaSymbol{(}\AgdaBound{¬a} \AgdaFunction{∘} \AgdaFunction{ends-eq}\AgdaSymbol{)} \AgdaKeyword{where}\<%
\\
\>[4]\AgdaIndent{6}{}\<[6]%
\>[6]\AgdaFunction{ends-eq} \AgdaSymbol{:} \AgdaSymbol{∀} \AgdaSymbol{\{}\AgdaBound{a} \AgdaBound{b} \AgdaBound{c} \AgdaBound{d} \AgdaSymbol{:} \AgdaBound{A}\AgdaSymbol{\}} \AgdaSymbol{→} \AgdaSymbol{(}\AgdaInductiveConstructor{⟨} \AgdaBound{a} \AgdaInductiveConstructor{⇒} \AgdaBound{b} \AgdaInductiveConstructor{⟩∷} \AgdaBound{p₁}\AgdaSymbol{)} \AgdaDatatype{⋙?} \AgdaSymbol{(}\AgdaInductiveConstructor{⟨} \AgdaBound{c} \AgdaInductiveConstructor{⇒} \AgdaBound{d} \AgdaInductiveConstructor{⟩∷} \AgdaBound{p₂}\AgdaSymbol{)} \AgdaSymbol{→} \AgdaBound{b} \AgdaDatatype{≡} \AgdaBound{c}\<%
\\
\>[4]\AgdaIndent{6}{}\<[6]%
\>[6]\AgdaFunction{ends-eq} \AgdaSymbol{(}\AgdaInductiveConstructor{⊤⋙?⊤} \AgdaBound{a} \AgdaBound{c} \AgdaBound{d} \AgdaBound{p}\AgdaSymbol{)} \AgdaSymbol{=} \AgdaInductiveConstructor{refl}\<%
\\
\>[0]\AgdaIndent{2}{}\<[2]%
\>[2]\AgdaSymbol{...} \AgdaSymbol{|} \AgdaInductiveConstructor{no} \AgdaBound{¬a} \AgdaSymbol{=} \AgdaInductiveConstructor{no} \AgdaSymbol{(}\AgdaBound{¬a} \AgdaFunction{∘} \AgdaFunction{⊤⊤-drop}\AgdaSymbol{)} \AgdaKeyword{where}\<%
\\
\>[2]\AgdaIndent{4}{}\<[4]%
\>[4]\AgdaFunction{⊤⊤-drop} \AgdaSymbol{:} \AgdaSymbol{∀} \AgdaSymbol{\{}\AgdaBound{a} \AgdaBound{b} \AgdaBound{c} \AgdaBound{d}\AgdaSymbol{\}} \AgdaSymbol{→} \AgdaSymbol{(}\AgdaInductiveConstructor{⟨} \AgdaBound{a} \AgdaInductiveConstructor{⇒} \AgdaBound{b} \AgdaInductiveConstructor{⟩∷} \AgdaBound{p₁}\AgdaSymbol{)} \AgdaDatatype{⋙?} \AgdaSymbol{(}\AgdaInductiveConstructor{⟨} \AgdaBound{c} \AgdaInductiveConstructor{⇒} \AgdaBound{d} \AgdaInductiveConstructor{⟩∷} \AgdaBound{p₂}\AgdaSymbol{)} \AgdaSymbol{→} \AgdaBound{p₁} \AgdaDatatype{⋙?} \AgdaBound{p₂}\<%
\\
\>[2]\AgdaIndent{4}{}\<[4]%
\>[4]\AgdaFunction{⊤⊤-drop} \AgdaSymbol{(}\AgdaInductiveConstructor{⊤⋙?⊤} \AgdaSymbol{\_} \AgdaSymbol{\_} \AgdaSymbol{\_} \AgdaBound{p}\AgdaSymbol{)} \AgdaSymbol{=} \AgdaBound{p}\<%
\\
\>\<%
\end{code}

\begin{code}%
\>\AgdaKeyword{module} \AgdaModule{TreeToReport} \AgdaKeyword{where}\<%
\\
%
\\
\>\AgdaKeyword{open} \AgdaKeyword{import} \AgdaModule{OXIj.BrutalDepTypes}\<%
\\
%
\\
\>\AgdaKeyword{open} \AgdaModule{≡-Prop}\<%
\\
%
\\
\>\AgdaKeyword{postulate} \AgdaPostulate{A} \AgdaSymbol{:} \AgdaPrimitiveType{Set}\<%
\\
%
\\
\>\AgdaKeyword{data} \AgdaDatatype{Tree} \AgdaSymbol{:} \AgdaPrimitiveType{Set} \AgdaKeyword{where}\<%
\\
\>[0]\AgdaIndent{2}{}\<[2]%
\>[2]\AgdaInductiveConstructor{Leaf} \AgdaSymbol{:} \AgdaSymbol{(}\AgdaBound{val} \AgdaSymbol{:} \AgdaPostulate{A}\AgdaSymbol{)} \AgdaSymbol{→} \AgdaDatatype{Tree}\<%
\\
\>[0]\AgdaIndent{2}{}\<[2]%
\>[2]\AgdaInductiveConstructor{Branch} \AgdaSymbol{:} \AgdaSymbol{(}\AgdaBound{left} \AgdaBound{right} \AgdaSymbol{:} \AgdaDatatype{Tree}\AgdaSymbol{)} \AgdaSymbol{→} \AgdaDatatype{Tree}\<%
\\
%
\\
\>\AgdaKeyword{data} \AgdaDatatype{Form} \AgdaSymbol{:} \AgdaPrimitiveType{Set} \AgdaKeyword{where}\<%
\\
\>[0]\AgdaIndent{2}{}\<[2]%
\>[2]\AgdaInductiveConstructor{Take} \AgdaSymbol{:} \AgdaDatatype{Form}\<%
\\
\>[0]\AgdaIndent{2}{}\<[2]%
\>[2]\AgdaInductiveConstructor{Skip} \AgdaSymbol{:} \AgdaDatatype{Form}\<%
\\
\>[0]\AgdaIndent{2}{}\<[2]%
\>[2]\AgdaInductiveConstructor{Branch} \AgdaSymbol{:} \AgdaSymbol{(}\AgdaBound{left} \AgdaBound{right} \AgdaSymbol{:} \AgdaDatatype{Form}\AgdaSymbol{)} \AgdaSymbol{→} \AgdaDatatype{Form}\<%
\\
%
\\
\>\AgdaKeyword{data} \AgdaDatatype{\_∥\_} \AgdaSymbol{:} \AgdaDatatype{Form} \AgdaSymbol{→} \AgdaDatatype{Form} \AgdaSymbol{→} \AgdaPrimitiveType{Set} \AgdaKeyword{where}\<%
\\
\>[0]\AgdaIndent{2}{}\<[2]%
\>[2]\AgdaInductiveConstructor{∅∥✶} \AgdaSymbol{:} \AgdaSymbol{(}\AgdaBound{s} \AgdaSymbol{:} \AgdaDatatype{Form}\AgdaSymbol{)} \AgdaSymbol{→} \AgdaInductiveConstructor{Skip} \AgdaDatatype{∥} \AgdaBound{s}\<%
\\
\>[0]\AgdaIndent{2}{}\<[2]%
\>[2]\AgdaInductiveConstructor{✶∥∅} \AgdaSymbol{:} \AgdaSymbol{(}\AgdaBound{s} \AgdaSymbol{:} \AgdaDatatype{Form}\AgdaSymbol{)} \AgdaSymbol{→} \AgdaBound{s} \AgdaDatatype{∥} \AgdaInductiveConstructor{Skip}\<%
\\
\>[0]\AgdaIndent{2}{}\<[2]%
\>[2]\AgdaInductiveConstructor{Branch-∥} \AgdaSymbol{:} \AgdaSymbol{∀} \AgdaSymbol{\{}\AgdaBound{L1} \AgdaBound{R1} \AgdaBound{L2} \AgdaBound{R2} \AgdaSymbol{:} \AgdaDatatype{Form}\AgdaSymbol{\}} \AgdaSymbol{→} \AgdaBound{L1} \AgdaDatatype{∥} \AgdaBound{L2} \AgdaSymbol{→} \AgdaBound{R1} \AgdaDatatype{∥} \AgdaBound{R2} \<[56]%
\>[56]\<%
\\
\>[2]\AgdaIndent{4}{}\<[4]%
\>[4]\AgdaSymbol{→} \AgdaInductiveConstructor{Branch} \AgdaBound{L1} \AgdaBound{R1} \AgdaDatatype{∥} \AgdaInductiveConstructor{Branch} \AgdaBound{L2} \AgdaBound{R2}\<%
\\
%
\\
\>\AgdaKeyword{data} \AgdaDatatype{Patch} \AgdaSymbol{:} \AgdaDatatype{Form} \AgdaSymbol{→} \AgdaPrimitiveType{Set} \AgdaKeyword{where}\<%
\\
\>[0]\AgdaIndent{2}{}\<[2]%
\>[2]\AgdaInductiveConstructor{I} \AgdaSymbol{:} \AgdaDatatype{Patch} \AgdaInductiveConstructor{Skip}\<%
\\
\>[0]\AgdaIndent{2}{}\<[2]%
\>[2]\AgdaInductiveConstructor{⟨\_⇒\_⟩} \AgdaSymbol{:} \AgdaSymbol{(}\AgdaBound{from} \AgdaBound{to} \AgdaSymbol{:} \AgdaDatatype{Tree}\AgdaSymbol{)} \AgdaSymbol{→} \AgdaDatatype{Patch} \AgdaInductiveConstructor{Take}\<%
\\
\>[0]\AgdaIndent{2}{}\<[2]%
\>[2]\AgdaInductiveConstructor{⟨\_∧\_⟩} \AgdaSymbol{:} \AgdaSymbol{∀} \AgdaSymbol{\{}\AgdaBound{sl} \AgdaBound{sr} \AgdaSymbol{:} \AgdaDatatype{Form}\AgdaSymbol{\}} \AgdaSymbol{→} \AgdaSymbol{(}\AgdaBound{left} \AgdaSymbol{:} \AgdaDatatype{Patch} \AgdaBound{sl}\AgdaSymbol{)} \AgdaSymbol{(}\AgdaBound{right} \AgdaSymbol{:} \AgdaDatatype{Patch} \AgdaBound{sr}\AgdaSymbol{)}\<%
\\
\>[2]\AgdaIndent{4}{}\<[4]%
\>[4]\AgdaSymbol{→} \AgdaDatatype{Patch} \AgdaSymbol{(}\AgdaInductiveConstructor{Branch} \AgdaBound{sl} \AgdaBound{sr}\AgdaSymbol{)}\<%
\\
%
\\
\>\AgdaKeyword{data} \AgdaDatatype{\_⊏\_} \AgdaSymbol{:} \AgdaSymbol{∀} \AgdaSymbol{\{}\AgdaBound{s} \AgdaSymbol{:} \AgdaDatatype{Form}\AgdaSymbol{\}} \AgdaSymbol{→} \AgdaDatatype{Patch} \AgdaBound{s} \AgdaSymbol{→} \AgdaDatatype{Tree} \AgdaSymbol{→} \AgdaPrimitiveType{Set} \AgdaKeyword{where}\<%
\\
\>[0]\AgdaIndent{2}{}\<[2]%
\>[2]\AgdaInductiveConstructor{⊏-I} \AgdaSymbol{:} \AgdaSymbol{(}\AgdaBound{t} \AgdaSymbol{:} \AgdaDatatype{Tree}\AgdaSymbol{)} \AgdaSymbol{→} \AgdaInductiveConstructor{I} \AgdaDatatype{⊏} \AgdaBound{t}\<%
\\
\>[0]\AgdaIndent{2}{}\<[2]%
\>[2]\AgdaInductiveConstructor{⊏-⇒} \AgdaSymbol{:} \AgdaSymbol{(}\AgdaBound{from} \AgdaBound{to} \AgdaSymbol{:} \AgdaDatatype{Tree}\AgdaSymbol{)} \AgdaSymbol{→} \AgdaInductiveConstructor{⟨} \AgdaBound{from} \AgdaInductiveConstructor{⇒} \AgdaBound{to} \AgdaInductiveConstructor{⟩} \AgdaDatatype{⊏} \AgdaBound{from}\<%
\\
\>[0]\AgdaIndent{2}{}\<[2]%
\>[2]\AgdaInductiveConstructor{⊏-∧} \AgdaSymbol{:} \AgdaSymbol{\{}\AgdaBound{sl} \AgdaBound{sr} \AgdaSymbol{:} \AgdaDatatype{Form}\AgdaSymbol{\}} \AgdaSymbol{\{}\AgdaBound{pl} \AgdaSymbol{:} \AgdaDatatype{Patch} \AgdaBound{sl}\AgdaSymbol{\}} \AgdaSymbol{\{}\AgdaBound{pr} \AgdaSymbol{:} \AgdaDatatype{Patch} \AgdaBound{sr}\AgdaSymbol{\}} \AgdaSymbol{\{}\AgdaBound{tl} \AgdaBound{tr} \AgdaSymbol{:} \AgdaDatatype{Tree}\AgdaSymbol{\}}\<%
\\
\>[2]\AgdaIndent{4}{}\<[4]%
\>[4]\AgdaSymbol{→} \AgdaBound{pl} \AgdaDatatype{⊏} \AgdaBound{tl} \AgdaSymbol{→} \AgdaBound{pr} \AgdaDatatype{⊏} \AgdaBound{tr} \AgdaSymbol{→} \AgdaInductiveConstructor{⟨} \AgdaBound{pl} \AgdaInductiveConstructor{∧} \AgdaBound{pr} \AgdaInductiveConstructor{⟩} \AgdaDatatype{⊏} \AgdaInductiveConstructor{Branch} \AgdaBound{tl} \AgdaBound{tr}\<%
\\
%
\\
\>\AgdaFunction{patch} \AgdaSymbol{:} \AgdaSymbol{\{}\AgdaBound{s} \AgdaSymbol{:} \AgdaDatatype{Form}\AgdaSymbol{\}} \AgdaSymbol{→} \AgdaSymbol{(}\AgdaBound{p} \AgdaSymbol{:} \AgdaDatatype{Patch} \AgdaBound{s}\AgdaSymbol{)} \AgdaSymbol{→} \AgdaSymbol{(}\AgdaBound{t} \AgdaSymbol{:} \AgdaDatatype{Tree}\AgdaSymbol{)} \AgdaSymbol{→} \AgdaBound{p} \AgdaDatatype{⊏} \AgdaBound{t} \AgdaSymbol{→} \AgdaDatatype{Tree}\<%
\\
%
\\
\>\AgdaFunction{patch} \AgdaInductiveConstructor{I} \AgdaBound{t} \AgdaSymbol{(}\AgdaInductiveConstructor{⊏-I} \AgdaSymbol{.}\AgdaBound{t}\AgdaSymbol{)} \AgdaSymbol{=} \AgdaBound{t}\<%
\\
%
\\
\>\AgdaFunction{patch} \AgdaInductiveConstructor{⟨} \AgdaSymbol{.}\AgdaBound{t} \AgdaInductiveConstructor{⇒} \AgdaBound{to} \AgdaInductiveConstructor{⟩} \AgdaBound{t} \AgdaSymbol{(}\AgdaInductiveConstructor{⊏-⇒} \AgdaSymbol{.}\AgdaBound{t} \AgdaSymbol{.}\AgdaBound{to}\AgdaSymbol{)} \AgdaSymbol{=} \AgdaBound{to}\<%
\\
%
\\
\>\AgdaFunction{patch} \AgdaInductiveConstructor{⟨} \AgdaBound{pl} \AgdaInductiveConstructor{∧} \AgdaBound{pr} \AgdaInductiveConstructor{⟩} \AgdaSymbol{(}\AgdaInductiveConstructor{Branch} \AgdaBound{tl} \AgdaBound{tr}\AgdaSymbol{)} \AgdaSymbol{(}\AgdaInductiveConstructor{⊏-∧} \AgdaBound{l-a} \AgdaBound{r-a}\AgdaSymbol{)} \AgdaSymbol{=} \<[49]%
\>[49]\<%
\\
\>[0]\AgdaIndent{2}{}\<[2]%
\>[2]\AgdaInductiveConstructor{Branch} \AgdaSymbol{(}\AgdaFunction{patch} \AgdaBound{pl} \AgdaBound{tl} \AgdaBound{l-a}\AgdaSymbol{)} \AgdaSymbol{(}\AgdaFunction{patch} \AgdaBound{pr} \AgdaBound{tr} \AgdaBound{r-a}\AgdaSymbol{)}\<%
\\
%
\\
\>\AgdaFunction{\_⟶\_} \AgdaSymbol{:} \AgdaSymbol{∀} \AgdaSymbol{\{}\AgdaBound{s₁} \AgdaBound{s₂} \AgdaSymbol{:} \AgdaDatatype{Form}\AgdaSymbol{\}}\<%
\\
\>[0]\AgdaIndent{2}{}\<[2]%
\>[2]\AgdaSymbol{→} \AgdaSymbol{(}\AgdaBound{p₁} \AgdaSymbol{:} \AgdaDatatype{Patch} \AgdaBound{s₁}\AgdaSymbol{)} \AgdaSymbol{→} \AgdaSymbol{(}\AgdaBound{p₂} \AgdaSymbol{:} \AgdaDatatype{Patch} \AgdaBound{s₂}\AgdaSymbol{)} \AgdaSymbol{→} \AgdaPrimitiveType{Set}\<%
\\
\>\AgdaFunction{\_⟶\_} \AgdaSymbol{\{}\AgdaBound{n}\AgdaSymbol{\}} \AgdaBound{p₁} \AgdaBound{p₂} \AgdaSymbol{=} \AgdaSymbol{∀} \AgdaSymbol{\{}\AgdaBound{x} \AgdaSymbol{:} \AgdaDatatype{Tree}\AgdaSymbol{\}}\<%
\\
\>[0]\AgdaIndent{2}{}\<[2]%
\>[2]\AgdaSymbol{→} \AgdaSymbol{(}\AgdaBound{p₁-x} \AgdaSymbol{:} \AgdaBound{p₁} \AgdaDatatype{⊏} \AgdaBound{x}\AgdaSymbol{)} \AgdaSymbol{→} \AgdaRecord{Σ} \AgdaSymbol{(}\AgdaBound{p₂} \AgdaDatatype{⊏} \AgdaBound{x}\AgdaSymbol{)} \AgdaSymbol{(λ} \AgdaBound{p₂-x} \AgdaSymbol{→} \<[43]%
\>[43]\<%
\\
\>[2]\AgdaIndent{4}{}\<[4]%
\>[4]\AgdaSymbol{(}\AgdaFunction{patch} \AgdaBound{p₁} \AgdaBound{x} \AgdaBound{p₁-x} \AgdaDatatype{≡} \AgdaFunction{patch} \AgdaBound{p₂} \AgdaBound{x} \AgdaBound{p₂-x}\AgdaSymbol{))}\<%
\\
%
\\
\>\AgdaFunction{\_⟷\_} \AgdaSymbol{:} \AgdaSymbol{∀} \AgdaSymbol{\{}\AgdaBound{f₁} \AgdaBound{f₂} \AgdaSymbol{:} \AgdaDatatype{Form}\AgdaSymbol{\}}\<%
\\
\>[0]\AgdaIndent{2}{}\<[2]%
\>[2]\AgdaSymbol{→} \AgdaSymbol{(}\AgdaBound{p₁} \AgdaSymbol{:} \AgdaDatatype{Patch} \AgdaBound{f₁}\AgdaSymbol{)} \AgdaSymbol{→} \AgdaSymbol{(}\AgdaBound{p₂} \AgdaSymbol{:} \AgdaDatatype{Patch} \AgdaBound{f₂}\AgdaSymbol{)} \AgdaSymbol{→} \AgdaPrimitiveType{Set}\<%
\\
\>\AgdaBound{p₁} \AgdaFunction{⟷} \AgdaBound{p₂} \AgdaSymbol{=} \AgdaSymbol{(}\AgdaBound{p₁} \AgdaFunction{⟶} \AgdaBound{p₂}\AgdaSymbol{)} \AgdaRecord{∧} \AgdaSymbol{(}\AgdaBound{p₂} \AgdaFunction{⟶} \AgdaBound{p₁}\AgdaSymbol{)}\<%
\\
%
\\
\>\AgdaFunction{\_⟷-bad\_} \AgdaSymbol{:} \AgdaSymbol{∀} \AgdaSymbol{\{}\AgdaBound{s₁} \AgdaBound{s₂} \AgdaSymbol{:} \AgdaDatatype{Form}\AgdaSymbol{\}} \<[27]%
\>[27]\<%
\\
\>[0]\AgdaIndent{2}{}\<[2]%
\>[2]\AgdaSymbol{→} \AgdaSymbol{(}\AgdaBound{p₁} \AgdaSymbol{:} \AgdaDatatype{Patch} \AgdaBound{s₁}\AgdaSymbol{)} \AgdaSymbol{→} \AgdaSymbol{(}\AgdaBound{p₂} \AgdaSymbol{:} \AgdaDatatype{Patch} \AgdaBound{s₂}\AgdaSymbol{)} \AgdaSymbol{→} \AgdaPrimitiveType{Set}\<%
\\
\>\AgdaBound{p₁} \AgdaFunction{⟷-bad} \AgdaBound{p₂} \AgdaSymbol{=} \AgdaSymbol{∀} \AgdaSymbol{(}\AgdaBound{x} \AgdaSymbol{:} \AgdaDatatype{Tree}\AgdaSymbol{)} \AgdaSymbol{→} \AgdaSymbol{(}\AgdaBound{p₁⊏x} \AgdaSymbol{:} \AgdaBound{p₁} \AgdaDatatype{⊏} \AgdaBound{x}\AgdaSymbol{)} \AgdaSymbol{→} \AgdaSymbol{(}\AgdaBound{p₂⊏x} \AgdaSymbol{:} \AgdaBound{p₂} \AgdaDatatype{⊏} \AgdaBound{x}\AgdaSymbol{)} \<[63]%
\>[63]\<%
\\
\>[0]\AgdaIndent{2}{}\<[2]%
\>[2]\AgdaSymbol{→} \AgdaFunction{patch} \AgdaBound{p₁} \AgdaBound{x} \AgdaBound{p₁⊏x} \AgdaDatatype{≡} \AgdaFunction{patch} \AgdaBound{p₂} \AgdaBound{x} \AgdaBound{p₂⊏x}\<%
\\
%
\\
\>\AgdaFunction{⟷-bad-test} \AgdaSymbol{:} \AgdaSymbol{∀} \AgdaSymbol{\{}\AgdaBound{a} \AgdaBound{b} \AgdaBound{c} \AgdaBound{d}\AgdaSymbol{\}} \AgdaSymbol{→} \AgdaBound{a} \AgdaFunction{≠} \AgdaBound{c} \AgdaSymbol{→} \AgdaInductiveConstructor{⟨} \AgdaBound{a} \AgdaInductiveConstructor{⇒} \AgdaBound{b} \AgdaInductiveConstructor{⟩} \AgdaFunction{⟷-bad} \AgdaInductiveConstructor{⟨} \AgdaBound{c} \AgdaInductiveConstructor{⇒} \AgdaBound{d} \AgdaInductiveConstructor{⟩}\<%
\\
\>\AgdaFunction{⟷-bad-test} \AgdaBound{a≠c} \AgdaBound{a} \AgdaSymbol{(}\AgdaInductiveConstructor{⊏-⇒} \AgdaSymbol{.}\AgdaBound{a} \AgdaSymbol{\_)} \AgdaSymbol{(}\AgdaInductiveConstructor{⊏-⇒} \AgdaSymbol{.}\AgdaBound{a} \AgdaSymbol{\_)} \AgdaSymbol{=} \AgdaFunction{⊥-elim} \AgdaSymbol{(}\AgdaBound{a≠c} \AgdaInductiveConstructor{refl}\AgdaSymbol{)}\<%
\\
%
\\
\>\AgdaFunction{⟷-refl} \AgdaSymbol{:} \AgdaSymbol{∀} \AgdaSymbol{\{}\AgdaBound{s} \AgdaSymbol{:} \AgdaDatatype{Form}\AgdaSymbol{\}} \AgdaSymbol{(}\AgdaBound{p} \AgdaSymbol{:} \AgdaDatatype{Patch} \AgdaBound{s}\AgdaSymbol{)} \AgdaSymbol{→} \AgdaBound{p} \AgdaFunction{⟷} \AgdaBound{p}\<%
\\
\>\AgdaFunction{⟷-refl} \AgdaBound{p} \AgdaSymbol{=} \AgdaSymbol{(λ} \AgdaBound{x} \AgdaSymbol{→} \AgdaBound{x} \AgdaInductiveConstructor{,} \AgdaInductiveConstructor{refl}\AgdaSymbol{)} \AgdaInductiveConstructor{,} \AgdaSymbol{(λ} \AgdaBound{x} \AgdaSymbol{→} \AgdaBound{x} \AgdaInductiveConstructor{,} \AgdaInductiveConstructor{refl}\AgdaSymbol{)}\<%
\\
%
\\
\>\AgdaFunction{⟷-sym} \AgdaSymbol{:} \AgdaSymbol{∀} \AgdaSymbol{\{}\AgdaBound{s₁} \AgdaBound{s₂}\AgdaSymbol{\}} \AgdaSymbol{\{}\AgdaBound{p₁} \AgdaSymbol{:} \AgdaDatatype{Patch} \AgdaBound{s₁}\AgdaSymbol{\}} \AgdaSymbol{\{}\AgdaBound{p₂} \AgdaSymbol{:} \AgdaDatatype{Patch} \AgdaBound{s₂}\AgdaSymbol{\}}\<%
\\
\>[0]\AgdaIndent{2}{}\<[2]%
\>[2]\AgdaSymbol{→} \AgdaSymbol{(}\AgdaBound{p₁} \AgdaFunction{⟷} \AgdaBound{p₂}\AgdaSymbol{)} \AgdaSymbol{→} \AgdaSymbol{(}\AgdaBound{p₂} \AgdaFunction{⟷} \AgdaBound{p₁}\AgdaSymbol{)}\<%
\\
\>\AgdaFunction{⟷-sym} \AgdaSymbol{(}\AgdaBound{1⟶2} \AgdaInductiveConstructor{,} \AgdaBound{2⟶1}\AgdaSymbol{)} \AgdaSymbol{=} \AgdaSymbol{(}\AgdaBound{2⟶1} \AgdaInductiveConstructor{,} \AgdaBound{1⟶2}\AgdaSymbol{)}\<%
\\
%
\\
\>\AgdaFunction{⟶-trans} \AgdaSymbol{:} \AgdaSymbol{\{}\AgdaBound{s₁} \AgdaBound{s₂} \AgdaBound{s₃} \AgdaSymbol{:} \AgdaDatatype{Form}\AgdaSymbol{\}} \<[28]%
\>[28]\<%
\\
\>[0]\AgdaIndent{2}{}\<[2]%
\>[2]\AgdaSymbol{→} \AgdaSymbol{\{}\AgdaBound{p₁} \AgdaSymbol{:} \AgdaDatatype{Patch} \AgdaBound{s₁}\AgdaSymbol{\}\{}\AgdaBound{p₂} \AgdaSymbol{:} \AgdaDatatype{Patch} \AgdaBound{s₂}\AgdaSymbol{\}\{}\AgdaBound{p₃} \AgdaSymbol{:} \AgdaDatatype{Patch} \AgdaBound{s₃}\AgdaSymbol{\}}\<%
\\
\>[0]\AgdaIndent{2}{}\<[2]%
\>[2]\AgdaSymbol{→} \AgdaSymbol{(}\AgdaBound{p₁} \AgdaFunction{⟶} \AgdaBound{p₂}\AgdaSymbol{)} \AgdaSymbol{→} \AgdaSymbol{(}\AgdaBound{p₂} \AgdaFunction{⟶} \AgdaBound{p₃}\AgdaSymbol{)} \AgdaSymbol{→} \AgdaSymbol{(}\AgdaBound{p₁} \AgdaFunction{⟶} \AgdaBound{p₃}\AgdaSymbol{)}\<%
\\
\>\AgdaFunction{⟶-trans} \AgdaSymbol{\{p₁} \AgdaSymbol{=} \AgdaBound{p₁}\AgdaSymbol{\}\{}\AgdaBound{p₂}\AgdaSymbol{\}\{}\AgdaBound{p₃}\AgdaSymbol{\}} \AgdaBound{1⟶2} \AgdaBound{2⟶3} \AgdaSymbol{\{}\AgdaBound{x}\AgdaSymbol{\}} \AgdaBound{p⊏x} \<[42]%
\>[42]\<%
\\
\>[0]\AgdaIndent{2}{}\<[2]%
\>[2]\AgdaKeyword{with} \AgdaFunction{patch} \AgdaBound{p₁} \AgdaBound{x} \AgdaBound{p⊏x} \AgdaSymbol{|} \AgdaBound{1⟶2} \AgdaBound{p⊏x} \<[32]%
\>[32]\<%
\\
\>\AgdaSymbol{...} \AgdaSymbol{|} \AgdaSymbol{.\_} \AgdaSymbol{|} \AgdaBound{p₂⊏x} \AgdaInductiveConstructor{,} \AgdaInductiveConstructor{refl} \AgdaSymbol{=} \AgdaBound{2⟶3} \AgdaBound{p₂⊏x}\<%
\\
%
\\
\>\AgdaFunction{⟷-trans} \AgdaSymbol{:} \AgdaSymbol{\{}\AgdaBound{s₁} \AgdaBound{s₂} \AgdaBound{s₃} \AgdaSymbol{:} \AgdaDatatype{Form}\AgdaSymbol{\}} \<[28]%
\>[28]\<%
\\
\>[0]\AgdaIndent{2}{}\<[2]%
\>[2]\AgdaSymbol{→} \AgdaSymbol{\{}\AgdaBound{p₁} \AgdaSymbol{:} \AgdaDatatype{Patch} \AgdaBound{s₁}\AgdaSymbol{\}\{}\AgdaBound{p₂} \AgdaSymbol{:} \AgdaDatatype{Patch} \AgdaBound{s₂}\AgdaSymbol{\}\{}\AgdaBound{p₃} \AgdaSymbol{:} \AgdaDatatype{Patch} \AgdaBound{s₃}\AgdaSymbol{\}}\<%
\\
\>[0]\AgdaIndent{2}{}\<[2]%
\>[2]\AgdaSymbol{→} \AgdaSymbol{(}\AgdaBound{p₁} \AgdaFunction{⟷} \AgdaBound{p₂}\AgdaSymbol{)} \AgdaSymbol{→} \AgdaSymbol{(}\AgdaBound{p₂} \AgdaFunction{⟷} \AgdaBound{p₃}\AgdaSymbol{)} \AgdaSymbol{→} \AgdaSymbol{(}\AgdaBound{p₁} \AgdaFunction{⟷} \AgdaBound{p₃}\AgdaSymbol{)}\<%
\\
\>\AgdaFunction{⟷-trans} \AgdaSymbol{(}\AgdaBound{1⟶2} \AgdaInductiveConstructor{,} \AgdaBound{2⟶1}\AgdaSymbol{)} \AgdaSymbol{(}\AgdaBound{2⟶3} \AgdaInductiveConstructor{,} \AgdaBound{3⟶2}\AgdaSymbol{)} \AgdaSymbol{=} \<[34]%
\>[34]\<%
\\
\>[0]\AgdaIndent{2}{}\<[2]%
\>[2]\AgdaSymbol{(}\AgdaFunction{⟶-trans} \AgdaBound{1⟶2} \AgdaBound{2⟶3}\AgdaSymbol{)} \AgdaInductiveConstructor{,} \AgdaSymbol{(}\AgdaFunction{⟶-trans} \AgdaBound{3⟶2} \AgdaBound{2⟶1}\AgdaSymbol{)}\<%
\\
%
\\
\>\AgdaFunction{⟶-branch} \AgdaSymbol{:} \AgdaSymbol{∀} \AgdaSymbol{\{}\AgdaBound{L1} \AgdaBound{L2} \AgdaBound{R1} \AgdaBound{R2} \AgdaSymbol{:} \AgdaDatatype{Form}\AgdaSymbol{\}}\<%
\\
\>[0]\AgdaIndent{2}{}\<[2]%
\>[2]\AgdaSymbol{→} \AgdaSymbol{\{}\AgdaBound{l₁} \AgdaSymbol{:} \AgdaDatatype{Patch} \AgdaBound{L1}\AgdaSymbol{\}} \AgdaSymbol{\{}\AgdaBound{l₂} \AgdaSymbol{:} \AgdaDatatype{Patch} \AgdaBound{L2}\AgdaSymbol{\}} \AgdaSymbol{\{}\AgdaBound{r₁} \AgdaSymbol{:} \AgdaDatatype{Patch} \AgdaBound{R1}\AgdaSymbol{\}} \AgdaSymbol{\{}\AgdaBound{r₂} \AgdaSymbol{:} \AgdaDatatype{Patch} \AgdaBound{R2}\AgdaSymbol{\}}\<%
\\
\>[0]\AgdaIndent{2}{}\<[2]%
\>[2]\AgdaSymbol{→} \AgdaSymbol{(}\AgdaBound{l₁} \AgdaFunction{⟶} \AgdaBound{l₂}\AgdaSymbol{)} \AgdaSymbol{→} \AgdaSymbol{(}\AgdaBound{r₁} \AgdaFunction{⟶} \AgdaBound{r₂}\AgdaSymbol{)} \AgdaSymbol{→} \AgdaInductiveConstructor{⟨} \AgdaBound{l₁} \AgdaInductiveConstructor{∧} \AgdaBound{r₁} \AgdaInductiveConstructor{⟩} \AgdaFunction{⟶} \AgdaInductiveConstructor{⟨} \AgdaBound{l₂} \AgdaInductiveConstructor{∧} \AgdaBound{r₂} \AgdaInductiveConstructor{⟩}\<%
\\
\>\AgdaFunction{⟶-branch} \AgdaSymbol{\{l₁} \AgdaSymbol{=} \AgdaBound{l₁}\AgdaSymbol{\}\{}\AgdaBound{l₂}\AgdaSymbol{\}\{}\AgdaBound{r₁}\AgdaSymbol{\}\{}\AgdaBound{r₂}\AgdaSymbol{\}} \AgdaBound{l₁⟶l₂} \AgdaBound{r₁⟶r₂} \AgdaSymbol{(}\AgdaInductiveConstructor{⊏-∧} \AgdaSymbol{\{tl} \AgdaSymbol{=} \AgdaBound{tl}\AgdaSymbol{\}\{}\AgdaBound{tr}\AgdaSymbol{\}} \AgdaBound{l⊏L} \AgdaBound{r⊏R}\AgdaSymbol{)} \<[71]%
\>[71]\<%
\\
\>[0]\AgdaIndent{2}{}\<[2]%
\>[2]\AgdaKeyword{with} \AgdaFunction{patch} \AgdaBound{l₁} \AgdaBound{tl} \AgdaBound{l⊏L} \AgdaSymbol{|} \AgdaBound{l₁⟶l₂} \AgdaBound{l⊏L} \<[35]%
\>[35]\<%
\\
\>[2]\AgdaIndent{4}{}\<[4]%
\>[4]\AgdaSymbol{|} \AgdaFunction{patch} \AgdaBound{r₁} \AgdaBound{tr} \AgdaBound{r⊏R} \AgdaSymbol{|} \AgdaBound{r₁⟶r₂} \AgdaBound{r⊏R}\<%
\\
\>\AgdaSymbol{...} \AgdaSymbol{|} \AgdaSymbol{.\_} \AgdaSymbol{|} \AgdaBound{l₂⊏L} \AgdaInductiveConstructor{,} \AgdaInductiveConstructor{refl} \AgdaSymbol{|} \AgdaSymbol{.\_} \AgdaSymbol{|} \AgdaBound{r₂⊏R} \AgdaInductiveConstructor{,} \AgdaInductiveConstructor{refl} \AgdaSymbol{=} \<[44]%
\>[44]\<%
\\
\>[0]\AgdaIndent{2}{}\<[2]%
\>[2]\AgdaInductiveConstructor{⊏-∧} \AgdaBound{l₂⊏L} \AgdaBound{r₂⊏R} \AgdaInductiveConstructor{,} \AgdaInductiveConstructor{refl}\<%
\\
%
\\
\>\AgdaFunction{⟷-branch} \AgdaSymbol{:} \AgdaSymbol{∀} \AgdaSymbol{\{}\AgdaBound{L1} \AgdaBound{L2} \AgdaBound{R1} \AgdaBound{R2} \AgdaSymbol{:} \AgdaDatatype{Form}\AgdaSymbol{\}}\<%
\\
\>[0]\AgdaIndent{2}{}\<[2]%
\>[2]\AgdaSymbol{→} \AgdaSymbol{\{}\AgdaBound{l₁} \AgdaSymbol{:} \AgdaDatatype{Patch} \AgdaBound{L1}\AgdaSymbol{\}} \AgdaSymbol{\{}\AgdaBound{l₂} \AgdaSymbol{:} \AgdaDatatype{Patch} \AgdaBound{L2}\AgdaSymbol{\}} \AgdaSymbol{\{}\AgdaBound{r₁} \AgdaSymbol{:} \AgdaDatatype{Patch} \AgdaBound{R1}\AgdaSymbol{\}} \AgdaSymbol{\{}\AgdaBound{r₂} \AgdaSymbol{:} \AgdaDatatype{Patch} \AgdaBound{R2}\AgdaSymbol{\}}\<%
\\
\>[0]\AgdaIndent{2}{}\<[2]%
\>[2]\AgdaSymbol{→} \AgdaSymbol{(}\AgdaBound{l₁} \AgdaFunction{⟷} \AgdaBound{l₂}\AgdaSymbol{)} \AgdaSymbol{→} \AgdaSymbol{(}\AgdaBound{r₁} \AgdaFunction{⟷} \AgdaBound{r₂}\AgdaSymbol{)} \AgdaSymbol{→} \AgdaInductiveConstructor{⟨} \AgdaBound{l₁} \AgdaInductiveConstructor{∧} \AgdaBound{r₁} \AgdaInductiveConstructor{⟩} \AgdaFunction{⟷} \AgdaInductiveConstructor{⟨} \AgdaBound{l₂} \AgdaInductiveConstructor{∧} \AgdaBound{r₂} \AgdaInductiveConstructor{⟩}\<%
\\
\>\AgdaFunction{⟷-branch} \AgdaSymbol{(}\AgdaBound{l₁⟶l₂} \AgdaInductiveConstructor{,} \AgdaBound{l₂⟶l₁}\AgdaSymbol{)} \AgdaSymbol{(}\AgdaBound{r₁⟶r₂} \AgdaInductiveConstructor{,} \AgdaBound{r₂⟶r₁}\AgdaSymbol{)} \AgdaSymbol{=} \<[43]%
\>[43]\<%
\\
\>[0]\AgdaIndent{2}{}\<[2]%
\>[2]\AgdaSymbol{(}\AgdaFunction{⟶-branch} \AgdaBound{l₁⟶l₂} \AgdaBound{r₁⟶r₂}\AgdaSymbol{)} \AgdaInductiveConstructor{,} \AgdaSymbol{(}\AgdaFunction{⟶-branch} \AgdaBound{l₂⟶l₁} \AgdaBound{r₂⟶r₁}\AgdaSymbol{)}\<%
\\
%
\\
\>\AgdaFunction{\_\textasciicircum\_} \AgdaSymbol{:} \AgdaSymbol{(}\AgdaBound{sl} \AgdaBound{sr} \AgdaSymbol{:} \AgdaDatatype{Form}\AgdaSymbol{)} \AgdaSymbol{→} \AgdaBound{sl} \AgdaDatatype{∥} \AgdaBound{sr} \AgdaSymbol{→} \AgdaDatatype{Form}\<%
\\
\>\AgdaFunction{\_\textasciicircum\_} \AgdaSymbol{.}\AgdaInductiveConstructor{Skip} \AgdaBound{sr} \AgdaSymbol{(}\AgdaInductiveConstructor{∅∥✶} \AgdaSymbol{.}\AgdaBound{sr}\AgdaSymbol{)} \AgdaSymbol{=} \AgdaBound{sr}\<%
\\
\>\AgdaFunction{\_\textasciicircum\_} \AgdaBound{sl} \AgdaSymbol{.}\AgdaInductiveConstructor{Skip} \AgdaSymbol{(}\AgdaInductiveConstructor{✶∥∅} \AgdaSymbol{.}\AgdaBound{sl}\AgdaSymbol{)} \AgdaSymbol{=} \AgdaBound{sl}\<%
\\
\>\AgdaFunction{\_\textasciicircum\_} \AgdaSymbol{(}\AgdaInductiveConstructor{Branch} \AgdaBound{L1} \AgdaBound{R1}\AgdaSymbol{)} \AgdaSymbol{(}\AgdaInductiveConstructor{Branch} \AgdaBound{L2} \AgdaBound{R2}\AgdaSymbol{)} \AgdaSymbol{(}\AgdaInductiveConstructor{Branch-∥} \AgdaBound{pl} \AgdaBound{pr}\AgdaSymbol{)} \AgdaSymbol{=} \<[53]%
\>[53]\<%
\\
\>[0]\AgdaIndent{2}{}\<[2]%
\>[2]\AgdaInductiveConstructor{Branch} \AgdaSymbol{((}\AgdaBound{L1} \AgdaFunction{\textasciicircum} \AgdaBound{L2}\AgdaSymbol{)} \AgdaBound{pl}\AgdaSymbol{)} \AgdaSymbol{((}\AgdaBound{R1} \AgdaFunction{\textasciicircum} \AgdaBound{R2}\AgdaSymbol{)} \AgdaBound{pr}\AgdaSymbol{)}\<%
\\
%
\\
\>\AgdaFunction{unite} \AgdaSymbol{:} \AgdaSymbol{\{}\AgdaBound{f₁} \AgdaBound{f₂} \AgdaSymbol{:} \AgdaDatatype{Form}\AgdaSymbol{\}} \AgdaSymbol{→} \AgdaBound{f₁} \AgdaDatatype{∥} \AgdaBound{f₂} \AgdaSymbol{→} \AgdaDatatype{Form}\<%
\\
\>\AgdaFunction{unite} \AgdaSymbol{\{}\AgdaBound{f₁}\AgdaSymbol{\}} \AgdaSymbol{\{}\AgdaBound{f₂}\AgdaSymbol{\}} \AgdaBound{p} \AgdaSymbol{=} \AgdaSymbol{(}\AgdaBound{f₁} \AgdaFunction{\textasciicircum} \AgdaBound{f₂}\AgdaSymbol{)} \AgdaBound{p}\<%
\\
%
\\
\>\AgdaFunction{\_⋀\_} \AgdaSymbol{:} \AgdaSymbol{\{}\AgdaBound{s₁} \AgdaBound{s₂} \AgdaSymbol{:} \AgdaDatatype{Form}\AgdaSymbol{\}} \AgdaSymbol{(}\AgdaBound{p₁} \AgdaSymbol{:} \AgdaDatatype{Patch} \AgdaBound{s₁}\AgdaSymbol{)} \AgdaSymbol{(}\AgdaBound{p₂} \AgdaSymbol{:} \AgdaDatatype{Patch} \AgdaBound{s₂}\AgdaSymbol{)}\<%
\\
\>[0]\AgdaIndent{2}{}\<[2]%
\>[2]\AgdaSymbol{→} \AgdaSymbol{(}\AgdaBound{s₁∥s₂} \AgdaSymbol{:} \AgdaBound{s₁} \AgdaDatatype{∥} \AgdaBound{s₂}\AgdaSymbol{)} \AgdaSymbol{→} \AgdaDatatype{Patch} \AgdaSymbol{(}\AgdaFunction{unite} \AgdaBound{s₁∥s₂}\AgdaSymbol{)}\<%
\\
\>\AgdaFunction{\_⋀\_} \AgdaInductiveConstructor{I} \AgdaInductiveConstructor{I} \AgdaSymbol{(}\AgdaInductiveConstructor{∅∥✶} \AgdaSymbol{.}\AgdaInductiveConstructor{Skip}\AgdaSymbol{)} \AgdaSymbol{=} \AgdaInductiveConstructor{I}\<%
\\
\>\AgdaFunction{\_⋀\_} \AgdaInductiveConstructor{I} \AgdaInductiveConstructor{I} \AgdaSymbol{(}\AgdaInductiveConstructor{✶∥∅} \AgdaSymbol{.}\AgdaInductiveConstructor{Skip}\AgdaSymbol{)} \AgdaSymbol{=} \AgdaInductiveConstructor{I}\<%
\\
\>\AgdaFunction{\_⋀\_} \AgdaInductiveConstructor{I} \AgdaInductiveConstructor{⟨} \AgdaBound{from} \AgdaInductiveConstructor{⇒} \AgdaBound{to} \AgdaInductiveConstructor{⟩} \AgdaSymbol{(}\AgdaInductiveConstructor{∅∥✶} \AgdaSymbol{.}\AgdaInductiveConstructor{Take}\AgdaSymbol{)} \AgdaSymbol{=} \AgdaInductiveConstructor{⟨} \AgdaBound{from} \AgdaInductiveConstructor{⇒} \AgdaBound{to} \AgdaInductiveConstructor{⟩}\<%
\\
\>\AgdaFunction{\_⋀\_} \AgdaInductiveConstructor{I} \AgdaSymbol{(}\AgdaInductiveConstructor{⟨\_∧\_⟩} \AgdaSymbol{\{}\AgdaBound{sl}\AgdaSymbol{\}} \AgdaSymbol{\{}\AgdaBound{sr}\AgdaSymbol{\}} \AgdaBound{pl} \AgdaBound{pr}\AgdaSymbol{)} \AgdaSymbol{(}\AgdaInductiveConstructor{∅∥✶} \AgdaSymbol{.(}\AgdaInductiveConstructor{Branch} \AgdaBound{sl} \AgdaBound{sr}\AgdaSymbol{)}\AgdaSymbol{)} \AgdaSymbol{=} \AgdaInductiveConstructor{⟨} \AgdaBound{pl} \AgdaInductiveConstructor{∧} \AgdaBound{pr} \AgdaInductiveConstructor{⟩}\<%
\\
\>\AgdaFunction{\_⋀\_} \AgdaInductiveConstructor{⟨} \AgdaBound{from} \AgdaInductiveConstructor{⇒} \AgdaBound{to} \AgdaInductiveConstructor{⟩} \AgdaInductiveConstructor{I} \AgdaSymbol{(}\AgdaInductiveConstructor{✶∥∅} \AgdaSymbol{.}\AgdaInductiveConstructor{Take}\AgdaSymbol{)} \AgdaSymbol{=} \AgdaInductiveConstructor{⟨} \AgdaBound{from} \AgdaInductiveConstructor{⇒} \AgdaBound{to} \AgdaInductiveConstructor{⟩}\<%
\\
\>\AgdaFunction{\_⋀\_} \AgdaSymbol{(}\AgdaInductiveConstructor{⟨\_∧\_⟩} \AgdaSymbol{\{}\AgdaBound{sl}\AgdaSymbol{\}} \AgdaSymbol{\{}\AgdaBound{sr}\AgdaSymbol{\}} \AgdaBound{pl} \AgdaBound{pr}\AgdaSymbol{)} \AgdaInductiveConstructor{I} \AgdaSymbol{(}\AgdaInductiveConstructor{✶∥∅} \AgdaSymbol{.(}\AgdaInductiveConstructor{Branch} \AgdaBound{sl} \AgdaBound{sr}\AgdaSymbol{)}\AgdaSymbol{)} \AgdaSymbol{=} \AgdaInductiveConstructor{⟨} \AgdaBound{pl} \AgdaInductiveConstructor{∧} \AgdaBound{pr} \AgdaInductiveConstructor{⟩}\<%
\\
\>\AgdaFunction{\_⋀\_} \AgdaInductiveConstructor{⟨} \AgdaBound{p₁} \AgdaInductiveConstructor{∧} \AgdaBound{p₂} \AgdaInductiveConstructor{⟩} \AgdaInductiveConstructor{⟨} \AgdaBound{p₃} \AgdaInductiveConstructor{∧} \AgdaBound{p₄} \AgdaInductiveConstructor{⟩} \AgdaSymbol{(}\AgdaInductiveConstructor{Branch-∥} \AgdaBound{s₁∥s₂} \AgdaBound{s₃∥s₄}\AgdaSymbol{)} \AgdaSymbol{=} \<[53]%
\>[53]\<%
\\
\>[0]\AgdaIndent{2}{}\<[2]%
\>[2]\AgdaInductiveConstructor{⟨} \AgdaSymbol{(}\AgdaBound{p₁} \AgdaFunction{⋀} \AgdaBound{p₃}\AgdaSymbol{)} \AgdaBound{s₁∥s₂} \AgdaInductiveConstructor{∧} \AgdaSymbol{(}\AgdaBound{p₂} \AgdaFunction{⋀} \AgdaBound{p₄}\AgdaSymbol{)} \AgdaBound{s₃∥s₄} \AgdaInductiveConstructor{⟩}\<%
\\
%
\\
\>\AgdaFunction{∥-sym} \AgdaSymbol{:} \AgdaSymbol{∀} \AgdaSymbol{\{}\AgdaBound{s₁} \AgdaBound{s₂} \AgdaSymbol{:} \AgdaDatatype{Form}\AgdaSymbol{\}} \AgdaSymbol{→} \AgdaBound{s₁} \AgdaDatatype{∥} \AgdaBound{s₂} \AgdaSymbol{→} \AgdaBound{s₂} \AgdaDatatype{∥} \AgdaBound{s₁}\<%
\\
\>\AgdaFunction{∥-sym} \AgdaSymbol{\{}\AgdaSymbol{.}\AgdaInductiveConstructor{Skip}\AgdaSymbol{\}} \AgdaSymbol{\{}\AgdaBound{s₂}\AgdaSymbol{\}} \AgdaSymbol{(}\AgdaInductiveConstructor{∅∥✶} \AgdaSymbol{.}\AgdaBound{s₂}\AgdaSymbol{)} \AgdaSymbol{=} \AgdaInductiveConstructor{✶∥∅} \AgdaBound{s₂}\<%
\\
\>\AgdaFunction{∥-sym} \AgdaSymbol{\{}\AgdaBound{s₁}\AgdaSymbol{\}} \AgdaSymbol{(}\AgdaInductiveConstructor{✶∥∅} \AgdaSymbol{.}\AgdaBound{s₁}\AgdaSymbol{)} \AgdaSymbol{=} \AgdaInductiveConstructor{∅∥✶} \AgdaBound{s₁}\<%
\\
\>\AgdaFunction{∥-sym} \AgdaSymbol{(}\AgdaInductiveConstructor{Branch-∥} \AgdaBound{s₁∥s₂} \AgdaBound{s₁∥s₃}\AgdaSymbol{)} \AgdaSymbol{=} \<[31]%
\>[31]\<%
\\
\>[0]\AgdaIndent{2}{}\<[2]%
\>[2]\AgdaInductiveConstructor{Branch-∥} \AgdaSymbol{(}\AgdaFunction{∥-sym} \AgdaBound{s₁∥s₂}\AgdaSymbol{)} \AgdaSymbol{(}\AgdaFunction{∥-sym} \AgdaBound{s₁∥s₃}\AgdaSymbol{)}\<%
\\
%
\\
\>\AgdaFunction{lemma-∥-unite} \AgdaSymbol{:} \AgdaSymbol{\{}\AgdaBound{s₁} \AgdaBound{s₂} \AgdaBound{s₃} \AgdaSymbol{:} \AgdaDatatype{Form}\AgdaSymbol{\}} \<[34]%
\>[34]\<%
\\
\>[0]\AgdaIndent{2}{}\<[2]%
\>[2]\AgdaSymbol{→} \AgdaSymbol{(}\AgdaBound{s₁∥s₂} \AgdaSymbol{:} \AgdaBound{s₁} \AgdaDatatype{∥} \AgdaBound{s₂}\AgdaSymbol{)} \AgdaSymbol{→} \AgdaBound{s₂} \AgdaDatatype{∥} \AgdaBound{s₃} \AgdaSymbol{→} \AgdaBound{s₁} \AgdaDatatype{∥} \AgdaBound{s₃}\<%
\\
\>[0]\AgdaIndent{2}{}\<[2]%
\>[2]\AgdaSymbol{→} \AgdaFunction{unite} \AgdaBound{s₁∥s₂} \AgdaDatatype{∥} \AgdaBound{s₃}\<%
\\
\>\AgdaFunction{lemma-∥-unite} \AgdaSymbol{\{}\AgdaSymbol{.}\AgdaInductiveConstructor{Skip}\AgdaSymbol{\}} \AgdaSymbol{\{}\AgdaBound{s₂}\AgdaSymbol{\}} \AgdaSymbol{(}\AgdaInductiveConstructor{∅∥✶} \AgdaSymbol{.}\AgdaBound{s₂}\AgdaSymbol{)} \AgdaBound{s₂∥s₃} \AgdaBound{s₁∥s₃} \AgdaSymbol{=} \AgdaBound{s₂∥s₃}\<%
\\
\>\AgdaFunction{lemma-∥-unite} \AgdaSymbol{\{}\AgdaBound{s₁}\AgdaSymbol{\}} \AgdaSymbol{(}\AgdaInductiveConstructor{✶∥∅} \AgdaSymbol{.}\AgdaBound{s₁}\AgdaSymbol{)} \AgdaBound{s₂∥s₃} \AgdaBound{s₁∥s₃} \AgdaSymbol{=} \AgdaBound{s₁∥s₃}\<%
\\
\>\AgdaFunction{lemma-∥-unite} \AgdaSymbol{(}\AgdaInductiveConstructor{Branch-∥} \AgdaSymbol{\{}\AgdaBound{L1}\AgdaSymbol{\}} \AgdaSymbol{\{}\AgdaBound{R1}\AgdaSymbol{\}} \AgdaSymbol{\{}\AgdaBound{L2}\AgdaSymbol{\}} \AgdaSymbol{\{}\AgdaBound{R2}\AgdaSymbol{\}} \AgdaBound{L1∥L2} \AgdaBound{R1∥R2}\AgdaSymbol{)} \<[57]%
\>[57]\<%
\\
\>[0]\AgdaIndent{2}{}\<[2]%
\>[2]\AgdaSymbol{(}\AgdaInductiveConstructor{✶∥∅} \AgdaSymbol{.(}\AgdaInductiveConstructor{Branch} \AgdaBound{L2} \AgdaBound{R2}\AgdaSymbol{)}\AgdaSymbol{)} \AgdaBound{s₁∥s₃} \<[30]%
\>[30]\<%
\\
\>[2]\AgdaIndent{4}{}\<[4]%
\>[4]\AgdaSymbol{=} \AgdaInductiveConstructor{✶∥∅} \AgdaSymbol{(}\AgdaInductiveConstructor{Branch} \AgdaSymbol{((}\AgdaBound{L1} \AgdaFunction{\textasciicircum} \AgdaBound{L2}\AgdaSymbol{)} \AgdaBound{L1∥L2}\AgdaSymbol{)} \AgdaSymbol{((}\AgdaBound{R1} \AgdaFunction{\textasciicircum} \AgdaBound{R2}\AgdaSymbol{)} \AgdaBound{R1∥R2}\AgdaSymbol{))}\<%
\\
\>\AgdaFunction{lemma-∥-unite} \AgdaSymbol{(}\AgdaInductiveConstructor{Branch-∥} \AgdaBound{L1∥L2} \AgdaBound{R1∥R2}\AgdaSymbol{)} \AgdaSymbol{(}\AgdaInductiveConstructor{Branch-∥} \AgdaBound{s₂∥s₃} \AgdaBound{s₂∥s₄}\AgdaSymbol{)} \<[60]%
\>[60]\<%
\\
\>[0]\AgdaIndent{2}{}\<[2]%
\>[2]\AgdaSymbol{(}\AgdaInductiveConstructor{Branch-∥} \AgdaBound{s₁∥s₃} \AgdaBound{s₁∥s₄}\AgdaSymbol{)} \<[25]%
\>[25]\<%
\\
\>[0]\AgdaIndent{2}{}\<[2]%
\>[2]\AgdaSymbol{=} \AgdaInductiveConstructor{Branch-∥} \AgdaSymbol{(}\AgdaFunction{lemma-∥-unite} \AgdaBound{L1∥L2} \AgdaBound{s₂∥s₃} \AgdaBound{s₁∥s₃}\AgdaSymbol{)}\<%
\\
\>[2]\AgdaIndent{4}{}\<[4]%
\>[4]\AgdaSymbol{(}\AgdaFunction{lemma-∥-unite} \AgdaBound{R1∥R2} \AgdaBound{s₂∥s₄} \AgdaBound{s₁∥s₄}\AgdaSymbol{)}\<%
\\
%
\\
\>\AgdaFunction{∥-sym²≡id} \AgdaSymbol{:} \AgdaSymbol{\{}\AgdaBound{s₁} \AgdaBound{s₂} \AgdaSymbol{:} \AgdaDatatype{Form}\AgdaSymbol{\}} \AgdaSymbol{→} \AgdaSymbol{(}\AgdaBound{s₁∥s₂} \AgdaSymbol{:} \AgdaBound{s₁} \AgdaDatatype{∥} \AgdaBound{s₂}\AgdaSymbol{)}\<%
\\
\>[0]\AgdaIndent{2}{}\<[2]%
\>[2]\AgdaSymbol{→} \AgdaBound{s₁∥s₂} \AgdaDatatype{≡} \AgdaFunction{∥-sym} \AgdaSymbol{(}\AgdaFunction{∥-sym} \AgdaBound{s₁∥s₂}\AgdaSymbol{)}\<%
\\
\>\AgdaFunction{∥-sym²≡id} \AgdaSymbol{\{}\AgdaSymbol{.}\AgdaInductiveConstructor{Skip}\AgdaSymbol{\}} \AgdaSymbol{\{}\AgdaBound{s₂}\AgdaSymbol{\}} \AgdaSymbol{(}\AgdaInductiveConstructor{∅∥✶} \AgdaSymbol{.}\AgdaBound{s₂}\AgdaSymbol{)} \AgdaSymbol{=} \AgdaInductiveConstructor{refl}\<%
\\
\>\AgdaFunction{∥-sym²≡id} \AgdaSymbol{\{}\AgdaBound{s₁}\AgdaSymbol{\}} \AgdaSymbol{(}\AgdaInductiveConstructor{✶∥∅} \AgdaSymbol{.}\AgdaBound{s₁}\AgdaSymbol{)} \AgdaSymbol{=} \AgdaInductiveConstructor{refl}\<%
\\
\>\AgdaFunction{∥-sym²≡id} \AgdaSymbol{(}\AgdaInductiveConstructor{Branch-∥} \AgdaBound{L1∥L2} \AgdaBound{R1∥R2}\AgdaSymbol{)} \<[33]%
\>[33]\<%
\\
\>[0]\AgdaIndent{2}{}\<[2]%
\>[2]\AgdaKeyword{with} \AgdaFunction{∥-sym} \AgdaSymbol{(}\AgdaFunction{∥-sym} \AgdaBound{L1∥L2}\AgdaSymbol{)} \AgdaSymbol{|} \AgdaFunction{∥-sym²≡id} \AgdaBound{L1∥L2}\<%
\\
\>[2]\AgdaIndent{5}{}\<[5]%
\>[5]\AgdaSymbol{|} \AgdaFunction{∥-sym} \AgdaSymbol{(}\AgdaFunction{∥-sym} \AgdaBound{R1∥R2}\AgdaSymbol{)} \AgdaSymbol{|} \AgdaFunction{∥-sym²≡id} \AgdaBound{R1∥R2}\<%
\\
\>\AgdaFunction{∥-sym²≡id} \AgdaSymbol{(}\AgdaInductiveConstructor{Branch-∥} \AgdaSymbol{.}\AgdaBound{i1} \AgdaSymbol{.}\AgdaBound{i2}\AgdaSymbol{)} \AgdaSymbol{|} \AgdaBound{i1} \AgdaSymbol{|} \AgdaInductiveConstructor{refl} \AgdaSymbol{|} \AgdaBound{i2} \AgdaSymbol{|} \AgdaInductiveConstructor{refl} \AgdaSymbol{=} \AgdaInductiveConstructor{refl}\<%
\\
%
\\
\>\AgdaFunction{⋀-comm} \AgdaSymbol{:} \AgdaSymbol{∀} \AgdaSymbol{\{}\AgdaBound{s₁} \AgdaBound{s₂} \AgdaSymbol{:} \AgdaDatatype{Form}\AgdaSymbol{\}}\<%
\\
\>[0]\AgdaIndent{2}{}\<[2]%
\>[2]\AgdaSymbol{→} \AgdaSymbol{(}\AgdaBound{s₁∥s₂} \AgdaSymbol{:} \AgdaBound{s₁} \AgdaDatatype{∥} \AgdaBound{s₂}\AgdaSymbol{)}\<%
\\
\>[0]\AgdaIndent{2}{}\<[2]%
\>[2]\AgdaSymbol{→} \AgdaSymbol{(}\AgdaBound{p₁} \AgdaSymbol{:} \AgdaDatatype{Patch} \AgdaBound{s₁}\AgdaSymbol{)} \AgdaSymbol{→} \AgdaSymbol{(}\AgdaBound{p₂} \AgdaSymbol{:} \AgdaDatatype{Patch} \AgdaBound{s₂}\AgdaSymbol{)}\<%
\\
\>[0]\AgdaIndent{2}{}\<[2]%
\>[2]\AgdaSymbol{→} \AgdaSymbol{((}\AgdaBound{p₁} \AgdaFunction{⋀} \AgdaBound{p₂}\AgdaSymbol{)} \AgdaBound{s₁∥s₂}\AgdaSymbol{)} \AgdaFunction{⟷} \AgdaSymbol{((}\AgdaBound{p₂} \AgdaFunction{⋀} \AgdaBound{p₁}\AgdaSymbol{)} \AgdaSymbol{(}\AgdaFunction{∥-sym} \AgdaBound{s₁∥s₂}\AgdaSymbol{))}\<%
\\
\>\AgdaFunction{⋀-comm} \AgdaSymbol{(}\AgdaInductiveConstructor{∅∥✶} \AgdaSymbol{.}\AgdaInductiveConstructor{Skip}\AgdaSymbol{)} \AgdaInductiveConstructor{I} \AgdaInductiveConstructor{I} \AgdaSymbol{=} \AgdaFunction{⟷-refl} \AgdaInductiveConstructor{I}\<%
\\
\>\AgdaFunction{⋀-comm} \AgdaSymbol{(}\AgdaInductiveConstructor{∅∥✶} \AgdaSymbol{.}\AgdaInductiveConstructor{Take}\AgdaSymbol{)} \AgdaInductiveConstructor{I} \AgdaInductiveConstructor{⟨} \AgdaBound{from} \AgdaInductiveConstructor{⇒} \AgdaBound{to} \AgdaInductiveConstructor{⟩} \AgdaSymbol{=} \AgdaFunction{⟷-refl} \AgdaInductiveConstructor{⟨} \AgdaBound{from} \AgdaInductiveConstructor{⇒} \AgdaBound{to} \AgdaInductiveConstructor{⟩}\<%
\\
\>\AgdaFunction{⋀-comm} \AgdaSymbol{(}\AgdaInductiveConstructor{∅∥✶} \AgdaSymbol{.\_}\AgdaSymbol{)} \AgdaInductiveConstructor{I} \AgdaInductiveConstructor{⟨} \AgdaBound{p₂} \AgdaInductiveConstructor{∧} \AgdaBound{p₃} \AgdaInductiveConstructor{⟩} \AgdaSymbol{=} \AgdaFunction{⟷-refl} \AgdaInductiveConstructor{⟨} \AgdaBound{p₂} \AgdaInductiveConstructor{∧} \AgdaBound{p₃} \AgdaInductiveConstructor{⟩}\<%
\\
\>\AgdaFunction{⋀-comm} \AgdaSymbol{(}\AgdaInductiveConstructor{✶∥∅} \AgdaSymbol{.}\AgdaInductiveConstructor{Skip}\AgdaSymbol{)} \AgdaInductiveConstructor{I} \AgdaInductiveConstructor{I} \AgdaSymbol{=} \AgdaFunction{⟷-refl} \AgdaInductiveConstructor{I}\<%
\\
\>\AgdaFunction{⋀-comm} \AgdaSymbol{(}\AgdaInductiveConstructor{✶∥∅} \AgdaSymbol{.}\AgdaInductiveConstructor{Take}\AgdaSymbol{)} \AgdaInductiveConstructor{⟨} \AgdaBound{from} \AgdaInductiveConstructor{⇒} \AgdaBound{to} \AgdaInductiveConstructor{⟩} \AgdaInductiveConstructor{I} \AgdaSymbol{=} \AgdaFunction{⟷-refl} \AgdaInductiveConstructor{⟨} \AgdaBound{from} \AgdaInductiveConstructor{⇒} \AgdaBound{to} \AgdaInductiveConstructor{⟩}\<%
\\
\>\AgdaFunction{⋀-comm} \AgdaSymbol{(}\AgdaInductiveConstructor{✶∥∅} \AgdaSymbol{.\_}\AgdaSymbol{)} \AgdaInductiveConstructor{⟨} \AgdaBound{p₁} \AgdaInductiveConstructor{∧} \AgdaBound{p₂} \AgdaInductiveConstructor{⟩} \AgdaInductiveConstructor{I} \AgdaSymbol{=} \AgdaFunction{⟷-refl} \AgdaInductiveConstructor{⟨} \AgdaBound{p₁} \AgdaInductiveConstructor{∧} \AgdaBound{p₂} \AgdaInductiveConstructor{⟩}\<%
\\
\>\AgdaFunction{⋀-comm} \AgdaSymbol{(}\AgdaInductiveConstructor{Branch-∥} \AgdaBound{L1∥L2} \AgdaBound{R1∥R2}\AgdaSymbol{)} \AgdaInductiveConstructor{⟨} \AgdaBound{l₁} \AgdaInductiveConstructor{∧} \AgdaBound{r₁} \AgdaInductiveConstructor{⟩} \AgdaInductiveConstructor{⟨} \AgdaBound{l₂} \AgdaInductiveConstructor{∧} \AgdaBound{r₂} \AgdaInductiveConstructor{⟩} \<[54]%
\>[54]\<%
\\
\>[0]\AgdaIndent{2}{}\<[2]%
\>[2]\AgdaSymbol{=} \AgdaFunction{⟷-branch} \AgdaFunction{L} \AgdaFunction{R}\<%
\\
\>[0]\AgdaIndent{2}{}\<[2]%
\>[2]\AgdaKeyword{where}\<%
\\
\>[2]\AgdaIndent{4}{}\<[4]%
\>[4]\AgdaFunction{L} \AgdaSymbol{=} \AgdaFunction{⋀-comm} \AgdaBound{L1∥L2} \AgdaBound{l₁} \AgdaBound{l₂}\<%
\\
\>[2]\AgdaIndent{4}{}\<[4]%
\>[4]\AgdaFunction{R} \AgdaSymbol{=} \AgdaFunction{⋀-comm} \AgdaBound{R1∥R2} \AgdaBound{r₁} \AgdaBound{r₂}\<%
\\
%
\\
\>\AgdaFunction{⋀-assoc} \AgdaSymbol{:} \AgdaSymbol{∀} \AgdaSymbol{\{}\AgdaBound{s₁} \AgdaBound{s₂} \AgdaBound{s₃} \AgdaSymbol{:} \AgdaDatatype{Form}\AgdaSymbol{\}} \<[30]%
\>[30]\<%
\\
\>[0]\AgdaIndent{2}{}\<[2]%
\>[2]\AgdaSymbol{→} \AgdaSymbol{(}\AgdaBound{s₁∥s₂} \AgdaSymbol{:} \AgdaBound{s₁} \AgdaDatatype{∥} \AgdaBound{s₂}\AgdaSymbol{)} \AgdaSymbol{→} \AgdaSymbol{(}\AgdaBound{s₂∥s₃} \AgdaSymbol{:} \AgdaBound{s₂} \AgdaDatatype{∥} \AgdaBound{s₃}\AgdaSymbol{)} \AgdaSymbol{→} \AgdaSymbol{(}\AgdaBound{s₁∥s₃} \AgdaSymbol{:} \AgdaBound{s₁} \AgdaDatatype{∥} \AgdaBound{s₃}\AgdaSymbol{)}\<%
\\
\>[0]\AgdaIndent{2}{}\<[2]%
\>[2]\AgdaSymbol{→} \AgdaSymbol{(}\AgdaBound{p₁} \AgdaSymbol{:} \AgdaDatatype{Patch} \AgdaBound{s₁}\AgdaSymbol{)} \AgdaSymbol{→} \AgdaSymbol{(}\AgdaBound{p₂} \AgdaSymbol{:} \AgdaDatatype{Patch} \AgdaBound{s₂}\AgdaSymbol{)} \AgdaSymbol{→} \AgdaSymbol{(}\AgdaBound{p₃} \AgdaSymbol{:} \AgdaDatatype{Patch} \AgdaBound{s₃}\AgdaSymbol{)}\<%
\\
\>[0]\AgdaIndent{2}{}\<[2]%
\>[2]\AgdaSymbol{→} \AgdaSymbol{(((}\AgdaBound{p₁} \AgdaFunction{⋀} \AgdaBound{p₂}\AgdaSymbol{)} \AgdaBound{s₁∥s₂}\AgdaSymbol{)} \AgdaFunction{⋀} \AgdaBound{p₃}\AgdaSymbol{)} \AgdaSymbol{(}\AgdaFunction{lemma-∥-unite} \AgdaBound{s₁∥s₂} \AgdaBound{s₂∥s₃} \AgdaBound{s₁∥s₃}\AgdaSymbol{)} \AgdaFunction{⟷} \<[65]%
\>[65]\<%
\\
\>[2]\AgdaIndent{4}{}\<[4]%
\>[4]\AgdaSymbol{(}\AgdaBound{p₁} \AgdaFunction{⋀} \AgdaSymbol{((}\AgdaBound{p₂} \AgdaFunction{⋀} \AgdaBound{p₃}\AgdaSymbol{)} \AgdaBound{s₂∥s₃}\AgdaSymbol{))} \<[29]%
\>[29]\<%
\\
\>[4]\AgdaIndent{6}{}\<[6]%
\>[6]\AgdaSymbol{(}\AgdaFunction{∥-sym} \AgdaSymbol{(}\AgdaFunction{lemma-∥-unite} \AgdaBound{s₂∥s₃} \AgdaSymbol{(}\AgdaFunction{∥-sym} \AgdaBound{s₁∥s₃}\AgdaSymbol{)} \AgdaSymbol{(}\AgdaFunction{∥-sym} \AgdaBound{s₁∥s₂}\AgdaSymbol{)))}\<%
\\
\>\AgdaFunction{⋀-assoc} \AgdaSymbol{(}\AgdaInductiveConstructor{∅∥✶} \AgdaSymbol{.}\AgdaInductiveConstructor{Skip}\AgdaSymbol{)} \AgdaSymbol{(}\AgdaInductiveConstructor{∅∥✶} \AgdaSymbol{.}\AgdaInductiveConstructor{Skip}\AgdaSymbol{)} \AgdaSymbol{(}\AgdaInductiveConstructor{∅∥✶} \AgdaSymbol{.}\AgdaInductiveConstructor{Skip}\AgdaSymbol{)} \AgdaInductiveConstructor{I} \AgdaInductiveConstructor{I} \AgdaInductiveConstructor{I} \AgdaSymbol{=} \AgdaFunction{⟷-refl} \AgdaInductiveConstructor{I}\<%
\\
\>\AgdaFunction{⋀-assoc} \AgdaSymbol{(}\AgdaInductiveConstructor{∅∥✶} \AgdaSymbol{.}\AgdaInductiveConstructor{Skip}\AgdaSymbol{)} \AgdaSymbol{(}\AgdaInductiveConstructor{∅∥✶} \AgdaSymbol{.}\AgdaInductiveConstructor{Take}\AgdaSymbol{)} \AgdaSymbol{(}\AgdaInductiveConstructor{∅∥✶} \AgdaSymbol{.}\AgdaInductiveConstructor{Take}\AgdaSymbol{)} \AgdaInductiveConstructor{I} \AgdaInductiveConstructor{I} \AgdaInductiveConstructor{⟨} \AgdaBound{from} \AgdaInductiveConstructor{⇒} \AgdaBound{to} \AgdaInductiveConstructor{⟩} \AgdaSymbol{=} \<[64]%
\>[64]\<%
\\
\>[0]\AgdaIndent{2}{}\<[2]%
\>[2]\AgdaFunction{⟷-refl} \AgdaInductiveConstructor{⟨} \AgdaBound{from} \AgdaInductiveConstructor{⇒} \AgdaBound{to} \AgdaInductiveConstructor{⟩}\<%
\\
\>\AgdaFunction{⋀-assoc} \AgdaSymbol{(}\AgdaInductiveConstructor{∅∥✶} \AgdaSymbol{.}\AgdaInductiveConstructor{Skip}\AgdaSymbol{)} \AgdaSymbol{(}\AgdaInductiveConstructor{∅∥✶} \AgdaSymbol{.\_}\AgdaSymbol{)} \AgdaSymbol{(}\AgdaInductiveConstructor{∅∥✶} \AgdaSymbol{.\_}\AgdaSymbol{)} \AgdaInductiveConstructor{I} \AgdaInductiveConstructor{I} \AgdaInductiveConstructor{⟨} \AgdaBound{p₃} \AgdaInductiveConstructor{∧} \AgdaBound{p₄} \AgdaInductiveConstructor{⟩} \AgdaSymbol{=} \AgdaFunction{⟷-refl} \AgdaInductiveConstructor{⟨} \AgdaBound{p₃} \AgdaInductiveConstructor{∧} \AgdaBound{p₄} \AgdaInductiveConstructor{⟩}\<%
\\
\>\AgdaFunction{⋀-assoc} \AgdaSymbol{(}\AgdaInductiveConstructor{∅∥✶} \AgdaSymbol{.}\AgdaInductiveConstructor{Skip}\AgdaSymbol{)} \AgdaSymbol{(}\AgdaInductiveConstructor{∅∥✶} \AgdaSymbol{.}\AgdaInductiveConstructor{Skip}\AgdaSymbol{)} \AgdaSymbol{(}\AgdaInductiveConstructor{✶∥∅} \AgdaSymbol{.}\AgdaInductiveConstructor{Skip}\AgdaSymbol{)} \AgdaInductiveConstructor{I} \AgdaInductiveConstructor{I} \AgdaInductiveConstructor{I} \AgdaSymbol{=} \AgdaFunction{⟷-refl} \AgdaInductiveConstructor{I}\<%
\\
\>\AgdaFunction{⋀-assoc} \AgdaSymbol{(}\AgdaInductiveConstructor{∅∥✶} \AgdaSymbol{.}\AgdaInductiveConstructor{Skip}\AgdaSymbol{)} \AgdaSymbol{(}\AgdaInductiveConstructor{✶∥∅} \AgdaSymbol{.}\AgdaInductiveConstructor{Skip}\AgdaSymbol{)} \AgdaSymbol{(}\AgdaInductiveConstructor{∅∥✶} \AgdaSymbol{.}\AgdaInductiveConstructor{Skip}\AgdaSymbol{)} \AgdaInductiveConstructor{I} \AgdaInductiveConstructor{I} \AgdaInductiveConstructor{I} \AgdaSymbol{=} \AgdaFunction{⟷-refl} \AgdaInductiveConstructor{I}\<%
\\
\>\AgdaFunction{⋀-assoc} \AgdaSymbol{(}\AgdaInductiveConstructor{∅∥✶} \AgdaSymbol{.}\AgdaInductiveConstructor{Take}\AgdaSymbol{)} \AgdaSymbol{(}\AgdaInductiveConstructor{✶∥∅} \AgdaSymbol{.}\AgdaInductiveConstructor{Take}\AgdaSymbol{)} \AgdaSymbol{(}\AgdaInductiveConstructor{∅∥✶} \AgdaSymbol{.}\AgdaInductiveConstructor{Skip}\AgdaSymbol{)} \AgdaInductiveConstructor{I} \AgdaInductiveConstructor{⟨} \AgdaBound{from} \AgdaInductiveConstructor{⇒} \AgdaBound{to} \AgdaInductiveConstructor{⟩} \AgdaInductiveConstructor{I} \AgdaSymbol{=} \<[64]%
\>[64]\<%
\\
\>[0]\AgdaIndent{2}{}\<[2]%
\>[2]\AgdaFunction{⟷-refl} \AgdaInductiveConstructor{⟨} \AgdaBound{from} \AgdaInductiveConstructor{⇒} \AgdaBound{to} \AgdaInductiveConstructor{⟩}\<%
\\
\>\AgdaFunction{⋀-assoc} \AgdaSymbol{(}\AgdaInductiveConstructor{∅∥✶} \AgdaSymbol{.\_}\AgdaSymbol{)} \AgdaSymbol{(}\AgdaInductiveConstructor{✶∥∅} \AgdaSymbol{.\_}\AgdaSymbol{)} \AgdaSymbol{(}\AgdaInductiveConstructor{∅∥✶} \AgdaSymbol{.}\AgdaInductiveConstructor{Skip}\AgdaSymbol{)} \AgdaInductiveConstructor{I} \AgdaInductiveConstructor{⟨} \AgdaBound{p₂} \AgdaInductiveConstructor{∧} \AgdaBound{p₃} \AgdaInductiveConstructor{⟩} \AgdaInductiveConstructor{I} \AgdaSymbol{=} \AgdaFunction{⟷-refl} \AgdaInductiveConstructor{⟨} \AgdaBound{p₂} \AgdaInductiveConstructor{∧} \AgdaBound{p₃} \AgdaInductiveConstructor{⟩}\<%
\\
\>\AgdaFunction{⋀-assoc} \AgdaSymbol{(}\AgdaInductiveConstructor{∅∥✶} \AgdaSymbol{.}\AgdaInductiveConstructor{Skip}\AgdaSymbol{)} \AgdaSymbol{(}\AgdaInductiveConstructor{✶∥∅} \AgdaSymbol{.}\AgdaInductiveConstructor{Skip}\AgdaSymbol{)} \AgdaSymbol{(}\AgdaInductiveConstructor{✶∥∅} \AgdaSymbol{.}\AgdaInductiveConstructor{Skip}\AgdaSymbol{)} \AgdaInductiveConstructor{I} \AgdaInductiveConstructor{I} \AgdaInductiveConstructor{I} \AgdaSymbol{=} \AgdaFunction{⟷-refl} \AgdaInductiveConstructor{I}\<%
\\
\>\AgdaFunction{⋀-assoc} \AgdaSymbol{(}\AgdaInductiveConstructor{∅∥✶} \AgdaSymbol{.}\AgdaInductiveConstructor{Take}\AgdaSymbol{)} \AgdaSymbol{(}\AgdaInductiveConstructor{✶∥∅} \AgdaSymbol{.}\AgdaInductiveConstructor{Take}\AgdaSymbol{)} \AgdaSymbol{(}\AgdaInductiveConstructor{✶∥∅} \AgdaSymbol{.}\AgdaInductiveConstructor{Skip}\AgdaSymbol{)} \AgdaInductiveConstructor{I} \AgdaInductiveConstructor{⟨} \AgdaBound{from} \AgdaInductiveConstructor{⇒} \AgdaBound{to} \AgdaInductiveConstructor{⟩} \AgdaInductiveConstructor{I} \AgdaSymbol{=} \<[64]%
\>[64]\<%
\\
\>[0]\AgdaIndent{2}{}\<[2]%
\>[2]\AgdaFunction{⟷-refl} \AgdaInductiveConstructor{⟨} \AgdaBound{from} \AgdaInductiveConstructor{⇒} \AgdaBound{to} \AgdaInductiveConstructor{⟩}\<%
\\
\>\AgdaFunction{⋀-assoc} \AgdaSymbol{(}\AgdaInductiveConstructor{∅∥✶} \AgdaSymbol{.\_}\AgdaSymbol{)} \AgdaSymbol{(}\AgdaInductiveConstructor{✶∥∅} \AgdaSymbol{.\_}\AgdaSymbol{)} \AgdaSymbol{(}\AgdaInductiveConstructor{✶∥∅} \AgdaSymbol{.}\AgdaInductiveConstructor{Skip}\AgdaSymbol{)} \AgdaInductiveConstructor{I} \AgdaInductiveConstructor{⟨} \AgdaBound{p₂} \AgdaInductiveConstructor{∧} \AgdaBound{p₃} \AgdaInductiveConstructor{⟩} \AgdaInductiveConstructor{I} \AgdaSymbol{=} \AgdaFunction{⟷-refl} \AgdaInductiveConstructor{⟨} \AgdaBound{p₂} \AgdaInductiveConstructor{∧} \AgdaBound{p₃} \AgdaInductiveConstructor{⟩}\<%
\\
\>\AgdaFunction{⋀-assoc} \AgdaSymbol{(}\AgdaInductiveConstructor{∅∥✶} \AgdaSymbol{.\_}\AgdaSymbol{)} \AgdaSymbol{(}\AgdaInductiveConstructor{Branch-∥} \AgdaBound{s₂∥s₃} \AgdaBound{s₂∥s₄}\AgdaSymbol{)} \AgdaSymbol{(}\AgdaInductiveConstructor{∅∥✶} \AgdaSymbol{.\_}\AgdaSymbol{)} \AgdaInductiveConstructor{I} \AgdaInductiveConstructor{⟨} \AgdaBound{p₂} \AgdaInductiveConstructor{∧} \AgdaBound{p₃} \AgdaInductiveConstructor{⟩} \AgdaInductiveConstructor{⟨} \AgdaBound{p₄} \AgdaInductiveConstructor{∧} \AgdaBound{p₅} \AgdaInductiveConstructor{⟩} \AgdaSymbol{=}\<%
\\
\>[0]\AgdaIndent{2}{}\<[2]%
\>[2]\AgdaFunction{⟷-refl} \AgdaInductiveConstructor{⟨} \AgdaSymbol{(}\AgdaBound{p₂} \AgdaFunction{⋀} \AgdaBound{p₄}\AgdaSymbol{)} \AgdaBound{s₂∥s₃} \AgdaInductiveConstructor{∧} \AgdaSymbol{(}\AgdaBound{p₃} \AgdaFunction{⋀} \AgdaBound{p₅}\AgdaSymbol{)} \AgdaBound{s₂∥s₄} \AgdaInductiveConstructor{⟩}\<%
\\
\>\AgdaFunction{⋀-assoc} \AgdaSymbol{(}\AgdaInductiveConstructor{✶∥∅} \AgdaSymbol{.}\AgdaInductiveConstructor{Skip}\AgdaSymbol{)} \AgdaSymbol{(}\AgdaInductiveConstructor{∅∥✶} \AgdaSymbol{.}\AgdaInductiveConstructor{Skip}\AgdaSymbol{)} \AgdaSymbol{(}\AgdaInductiveConstructor{∅∥✶} \AgdaSymbol{.}\AgdaInductiveConstructor{Skip}\AgdaSymbol{)} \AgdaInductiveConstructor{I} \AgdaInductiveConstructor{I} \AgdaInductiveConstructor{I} \AgdaSymbol{=} \AgdaFunction{⟷-refl} \AgdaInductiveConstructor{I}\<%
\\
\>\AgdaFunction{⋀-assoc} \AgdaSymbol{(}\AgdaInductiveConstructor{✶∥∅} \AgdaSymbol{.}\AgdaInductiveConstructor{Skip}\AgdaSymbol{)} \AgdaSymbol{(}\AgdaInductiveConstructor{∅∥✶} \AgdaSymbol{.}\AgdaInductiveConstructor{Take}\AgdaSymbol{)} \AgdaSymbol{(}\AgdaInductiveConstructor{∅∥✶} \AgdaSymbol{.}\AgdaInductiveConstructor{Take}\AgdaSymbol{)} \AgdaInductiveConstructor{I} \AgdaInductiveConstructor{I} \AgdaInductiveConstructor{⟨} \AgdaBound{from} \AgdaInductiveConstructor{⇒} \AgdaBound{to} \AgdaInductiveConstructor{⟩} \AgdaSymbol{=} \<[64]%
\>[64]\<%
\\
\>[0]\AgdaIndent{2}{}\<[2]%
\>[2]\AgdaFunction{⟷-refl} \AgdaInductiveConstructor{⟨} \AgdaBound{from} \AgdaInductiveConstructor{⇒} \AgdaBound{to} \AgdaInductiveConstructor{⟩}\<%
\\
\>\AgdaFunction{⋀-assoc} \AgdaSymbol{(}\AgdaInductiveConstructor{✶∥∅} \AgdaSymbol{.}\AgdaInductiveConstructor{Skip}\AgdaSymbol{)} \AgdaSymbol{(}\AgdaInductiveConstructor{∅∥✶} \AgdaSymbol{.\_}\AgdaSymbol{)} \AgdaSymbol{(}\AgdaInductiveConstructor{∅∥✶} \AgdaSymbol{.\_}\AgdaSymbol{)} \AgdaInductiveConstructor{I} \AgdaInductiveConstructor{I} \AgdaInductiveConstructor{⟨} \AgdaBound{p₃} \AgdaInductiveConstructor{∧} \AgdaBound{p₄} \AgdaInductiveConstructor{⟩} \AgdaSymbol{=} \AgdaFunction{⟷-refl} \AgdaInductiveConstructor{⟨} \AgdaBound{p₃} \AgdaInductiveConstructor{∧} \AgdaBound{p₄} \AgdaInductiveConstructor{⟩}\<%
\\
\>\AgdaFunction{⋀-assoc} \AgdaSymbol{(}\AgdaInductiveConstructor{✶∥∅} \AgdaSymbol{.}\AgdaInductiveConstructor{Skip}\AgdaSymbol{)} \AgdaSymbol{(}\AgdaInductiveConstructor{∅∥✶} \AgdaSymbol{.}\AgdaInductiveConstructor{Skip}\AgdaSymbol{)} \AgdaSymbol{(}\AgdaInductiveConstructor{✶∥∅} \AgdaSymbol{.}\AgdaInductiveConstructor{Skip}\AgdaSymbol{)} \AgdaInductiveConstructor{I} \AgdaInductiveConstructor{I} \AgdaInductiveConstructor{I} \AgdaSymbol{=} \AgdaFunction{⟷-refl} \AgdaInductiveConstructor{I}\<%
\\
\>\AgdaFunction{⋀-assoc} \AgdaSymbol{(}\AgdaInductiveConstructor{✶∥∅} \AgdaSymbol{.}\AgdaInductiveConstructor{Take}\AgdaSymbol{)} \AgdaSymbol{(}\AgdaInductiveConstructor{∅∥✶} \AgdaSymbol{.}\AgdaInductiveConstructor{Skip}\AgdaSymbol{)} \AgdaSymbol{(}\AgdaInductiveConstructor{✶∥∅} \AgdaSymbol{.}\AgdaInductiveConstructor{Take}\AgdaSymbol{)} \AgdaInductiveConstructor{⟨} \AgdaBound{from} \AgdaInductiveConstructor{⇒} \AgdaBound{to} \AgdaInductiveConstructor{⟩} \AgdaInductiveConstructor{I} \AgdaInductiveConstructor{I} \AgdaSymbol{=} \<[64]%
\>[64]\<%
\\
\>[0]\AgdaIndent{2}{}\<[2]%
\>[2]\AgdaFunction{⟷-refl} \AgdaInductiveConstructor{⟨} \AgdaBound{from} \AgdaInductiveConstructor{⇒} \AgdaBound{to} \AgdaInductiveConstructor{⟩}\<%
\\
\>\AgdaFunction{⋀-assoc} \AgdaSymbol{(}\AgdaInductiveConstructor{✶∥∅} \AgdaSymbol{.\_}\AgdaSymbol{)} \AgdaSymbol{(}\AgdaInductiveConstructor{∅∥✶} \AgdaSymbol{.}\AgdaInductiveConstructor{Skip}\AgdaSymbol{)} \AgdaSymbol{(}\AgdaInductiveConstructor{✶∥∅} \AgdaSymbol{.\_}\AgdaSymbol{)} \AgdaInductiveConstructor{⟨} \AgdaBound{p₁} \AgdaInductiveConstructor{∧} \AgdaBound{p₂} \AgdaInductiveConstructor{⟩} \AgdaInductiveConstructor{I} \AgdaInductiveConstructor{I} \AgdaSymbol{=} \AgdaFunction{⟷-refl} \AgdaInductiveConstructor{⟨} \AgdaBound{p₁} \AgdaInductiveConstructor{∧} \AgdaBound{p₂} \AgdaInductiveConstructor{⟩}\<%
\\
\>\AgdaFunction{⋀-assoc} \AgdaSymbol{(}\AgdaInductiveConstructor{✶∥∅} \AgdaSymbol{.\_}\AgdaSymbol{)} \AgdaSymbol{(}\AgdaInductiveConstructor{∅∥✶} \AgdaSymbol{.\_}\AgdaSymbol{)} \AgdaSymbol{(}\AgdaInductiveConstructor{Branch-∥} \AgdaBound{s₁∥s₃} \AgdaBound{s₂∥s₄}\AgdaSymbol{)} \AgdaInductiveConstructor{⟨} \AgdaBound{p₁} \AgdaInductiveConstructor{∧} \AgdaBound{p₂} \AgdaInductiveConstructor{⟩} \AgdaInductiveConstructor{I} \AgdaInductiveConstructor{⟨} \AgdaBound{p₃} \AgdaInductiveConstructor{∧} \AgdaBound{p₄} \AgdaInductiveConstructor{⟩} \<[75]%
\>[75]\<%
\\
\>[0]\AgdaIndent{2}{}\<[2]%
\>[2]\AgdaKeyword{with} \AgdaFunction{∥-sym} \AgdaSymbol{(}\AgdaFunction{∥-sym} \AgdaBound{s₁∥s₃}\AgdaSymbol{)} \AgdaSymbol{|} \AgdaFunction{∥-sym²≡id} \AgdaBound{s₁∥s₃} \<[45]%
\>[45]\<%
\\
\>[2]\AgdaIndent{4}{}\<[4]%
\>[4]\AgdaSymbol{|} \AgdaFunction{∥-sym} \AgdaSymbol{(}\AgdaFunction{∥-sym} \AgdaBound{s₂∥s₄}\AgdaSymbol{)} \AgdaSymbol{|} \AgdaFunction{∥-sym²≡id} \AgdaBound{s₂∥s₄}\<%
\\
\>\AgdaSymbol{...} \AgdaSymbol{|} \AgdaSymbol{.}\AgdaBound{s₁∥s₃} \AgdaSymbol{|} \AgdaInductiveConstructor{refl} \AgdaSymbol{|} \AgdaSymbol{.}\AgdaBound{s₂∥s₄} \AgdaSymbol{|} \AgdaInductiveConstructor{refl} \AgdaSymbol{=} \<[38]%
\>[38]\<%
\\
\>[0]\AgdaIndent{2}{}\<[2]%
\>[2]\AgdaFunction{⟷-refl} \AgdaInductiveConstructor{⟨} \AgdaSymbol{(}\AgdaBound{p₁} \AgdaFunction{⋀} \AgdaBound{p₃}\AgdaSymbol{)} \AgdaBound{s₁∥s₃} \AgdaInductiveConstructor{∧} \AgdaSymbol{(}\AgdaBound{p₂} \AgdaFunction{⋀} \AgdaBound{p₄}\AgdaSymbol{)} \AgdaBound{s₂∥s₄} \AgdaInductiveConstructor{⟩}\<%
\\
\>\AgdaFunction{⋀-assoc} \AgdaSymbol{(}\AgdaInductiveConstructor{✶∥∅} \AgdaSymbol{.}\AgdaInductiveConstructor{Skip}\AgdaSymbol{)} \AgdaSymbol{(}\AgdaInductiveConstructor{✶∥∅} \AgdaSymbol{.}\AgdaInductiveConstructor{Skip}\AgdaSymbol{)} \AgdaSymbol{(}\AgdaInductiveConstructor{∅∥✶} \AgdaSymbol{.}\AgdaInductiveConstructor{Skip}\AgdaSymbol{)} \AgdaInductiveConstructor{I} \AgdaInductiveConstructor{I} \AgdaInductiveConstructor{I} \AgdaSymbol{=} \AgdaFunction{⟷-refl} \AgdaInductiveConstructor{I}\<%
\\
\>\AgdaFunction{⋀-assoc} \AgdaSymbol{(}\AgdaInductiveConstructor{✶∥∅} \AgdaSymbol{.}\AgdaInductiveConstructor{Skip}\AgdaSymbol{)} \AgdaSymbol{(}\AgdaInductiveConstructor{✶∥∅} \AgdaSymbol{.}\AgdaInductiveConstructor{Skip}\AgdaSymbol{)} \AgdaSymbol{(}\AgdaInductiveConstructor{✶∥∅} \AgdaSymbol{.}\AgdaInductiveConstructor{Skip}\AgdaSymbol{)} \AgdaInductiveConstructor{I} \AgdaInductiveConstructor{I} \AgdaInductiveConstructor{I} \AgdaSymbol{=} \AgdaFunction{⟷-refl} \AgdaInductiveConstructor{I}\<%
\\
\>\AgdaFunction{⋀-assoc} \AgdaSymbol{(}\AgdaInductiveConstructor{✶∥∅} \AgdaSymbol{.}\AgdaInductiveConstructor{Take}\AgdaSymbol{)} \AgdaSymbol{(}\AgdaInductiveConstructor{✶∥∅} \AgdaSymbol{.}\AgdaInductiveConstructor{Skip}\AgdaSymbol{)} \AgdaSymbol{(}\AgdaInductiveConstructor{✶∥∅} \AgdaSymbol{.}\AgdaInductiveConstructor{Take}\AgdaSymbol{)} \AgdaInductiveConstructor{⟨} \AgdaBound{from} \AgdaInductiveConstructor{⇒} \AgdaBound{to} \AgdaInductiveConstructor{⟩} \AgdaInductiveConstructor{I} \AgdaInductiveConstructor{I} \AgdaSymbol{=} \<[64]%
\>[64]\<%
\\
\>[0]\AgdaIndent{2}{}\<[2]%
\>[2]\AgdaFunction{⟷-refl} \AgdaInductiveConstructor{⟨} \AgdaBound{from} \AgdaInductiveConstructor{⇒} \AgdaBound{to} \AgdaInductiveConstructor{⟩}\<%
\\
\>\AgdaFunction{⋀-assoc} \AgdaSymbol{(}\AgdaInductiveConstructor{✶∥∅} \AgdaSymbol{.\_}\AgdaSymbol{)} \AgdaSymbol{(}\AgdaInductiveConstructor{✶∥∅} \AgdaSymbol{.}\AgdaInductiveConstructor{Skip}\AgdaSymbol{)} \AgdaSymbol{(}\AgdaInductiveConstructor{✶∥∅} \AgdaSymbol{.\_}\AgdaSymbol{)} \AgdaInductiveConstructor{⟨} \AgdaBound{p₁} \AgdaInductiveConstructor{∧} \AgdaBound{p₂} \AgdaInductiveConstructor{⟩} \AgdaInductiveConstructor{I} \AgdaInductiveConstructor{I} \AgdaSymbol{=} \AgdaFunction{⟷-refl} \AgdaInductiveConstructor{⟨} \AgdaBound{p₁} \AgdaInductiveConstructor{∧} \AgdaBound{p₂} \AgdaInductiveConstructor{⟩}\<%
\\
\>\AgdaFunction{⋀-assoc} \AgdaSymbol{(}\AgdaInductiveConstructor{Branch-∥} \AgdaBound{s₁∥s₃} \AgdaBound{s₂∥s₄}\AgdaSymbol{)} \AgdaSymbol{(}\AgdaInductiveConstructor{✶∥∅} \AgdaSymbol{.\_}\AgdaSymbol{)} \AgdaSymbol{(}\AgdaInductiveConstructor{✶∥∅} \AgdaSymbol{.\_}\AgdaSymbol{)} \AgdaInductiveConstructor{⟨} \AgdaBound{p₁} \AgdaInductiveConstructor{∧} \AgdaBound{p₂} \AgdaInductiveConstructor{⟩} \AgdaInductiveConstructor{⟨} \AgdaBound{p₃} \AgdaInductiveConstructor{∧} \AgdaBound{p₄} \AgdaInductiveConstructor{⟩} \AgdaInductiveConstructor{I}\<%
\\
\>[0]\AgdaIndent{2}{}\<[2]%
\>[2]\AgdaKeyword{with} \AgdaFunction{∥-sym} \AgdaSymbol{(}\AgdaFunction{∥-sym} \AgdaBound{s₁∥s₃}\AgdaSymbol{)} \AgdaSymbol{|} \AgdaFunction{∥-sym²≡id} \AgdaBound{s₁∥s₃} \<[45]%
\>[45]\<%
\\
\>[2]\AgdaIndent{4}{}\<[4]%
\>[4]\AgdaSymbol{|} \AgdaFunction{∥-sym} \AgdaSymbol{(}\AgdaFunction{∥-sym} \AgdaBound{s₂∥s₄}\AgdaSymbol{)} \AgdaSymbol{|} \AgdaFunction{∥-sym²≡id} \AgdaBound{s₂∥s₄}\<%
\\
\>\AgdaSymbol{...} \AgdaSymbol{|} \AgdaSymbol{.}\AgdaBound{s₁∥s₃} \AgdaSymbol{|} \AgdaInductiveConstructor{refl} \AgdaSymbol{|} \AgdaSymbol{.}\AgdaBound{s₂∥s₄} \AgdaSymbol{|} \AgdaInductiveConstructor{refl} \AgdaSymbol{=} \<[38]%
\>[38]\<%
\\
\>[0]\AgdaIndent{2}{}\<[2]%
\>[2]\AgdaFunction{⟷-refl} \AgdaInductiveConstructor{⟨} \AgdaSymbol{(}\AgdaBound{p₁} \AgdaFunction{⋀} \AgdaBound{p₃}\AgdaSymbol{)} \AgdaBound{s₁∥s₃} \AgdaInductiveConstructor{∧} \AgdaSymbol{(}\AgdaBound{p₂} \AgdaFunction{⋀} \AgdaBound{p₄}\AgdaSymbol{)} \AgdaBound{s₂∥s₄} \AgdaInductiveConstructor{⟩}\<%
\\
\>\AgdaFunction{⋀-assoc} \AgdaSymbol{(}\AgdaInductiveConstructor{Branch-∥} \AgdaBound{L1∥L2} \AgdaBound{R1∥R2}\AgdaSymbol{)} \AgdaSymbol{(}\AgdaInductiveConstructor{Branch-∥} \AgdaBound{L2∥L3} \AgdaBound{R2∥R3}\AgdaSymbol{)} \<[54]%
\>[54]\<%
\\
\>[0]\AgdaIndent{2}{}\<[2]%
\>[2]\AgdaSymbol{(}\AgdaInductiveConstructor{Branch-∥} \AgdaBound{L1∥L3} \AgdaBound{R1∥R3}\AgdaSymbol{)} \AgdaInductiveConstructor{⟨} \AgdaBound{p₁} \AgdaInductiveConstructor{∧} \AgdaBound{p₂} \AgdaInductiveConstructor{⟩} \AgdaInductiveConstructor{⟨} \AgdaBound{p₃} \AgdaInductiveConstructor{∧} \AgdaBound{p₄} \AgdaInductiveConstructor{⟩} \AgdaInductiveConstructor{⟨} \AgdaBound{p₅} \AgdaInductiveConstructor{∧} \AgdaBound{p₆} \AgdaInductiveConstructor{⟩} \AgdaSymbol{=} \<[63]%
\>[63]\<%
\\
\>[2]\AgdaIndent{4}{}\<[4]%
\>[4]\AgdaFunction{⟷-branch} \<[13]%
\>[13]\<%
\\
\>[4]\AgdaIndent{6}{}\<[6]%
\>[6]\AgdaSymbol{(}\AgdaFunction{⋀-assoc} \AgdaBound{L1∥L2} \AgdaBound{L2∥L3} \AgdaBound{L1∥L3} \AgdaBound{p₁} \AgdaBound{p₃} \AgdaBound{p₅}\AgdaSymbol{)} \<[43]%
\>[43]\<%
\\
\>[4]\AgdaIndent{6}{}\<[6]%
\>[6]\AgdaSymbol{(}\AgdaFunction{⋀-assoc} \AgdaBound{R1∥R2} \AgdaBound{R2∥R3} \AgdaBound{R1∥R3} \AgdaBound{p₂} \AgdaBound{p₄} \AgdaBound{p₆}\AgdaSymbol{)}\<%
\\
%
\\
\>\AgdaKeyword{data} \AgdaDatatype{\_⋙?\_} \AgdaSymbol{:} \AgdaSymbol{\{}\AgdaBound{s₁} \AgdaBound{s₂} \AgdaSymbol{:} \AgdaDatatype{Form}\AgdaSymbol{\}} \AgdaSymbol{(}\AgdaBound{p₁} \AgdaSymbol{:} \AgdaDatatype{Patch} \AgdaBound{s₁}\AgdaSymbol{)} \AgdaSymbol{(}\AgdaBound{p₂} \AgdaSymbol{:} \AgdaDatatype{Patch} \AgdaBound{s₂}\AgdaSymbol{)} \AgdaSymbol{→} \AgdaPrimitiveType{Set} \AgdaKeyword{where}\<%
\\
\>[0]\AgdaIndent{2}{}\<[2]%
\>[2]\AgdaInductiveConstructor{✶⋙?I} \AgdaSymbol{:} \AgdaSymbol{∀} \AgdaSymbol{\{}\AgdaBound{s} \AgdaSymbol{:} \AgdaDatatype{Form}\AgdaSymbol{\}} \AgdaSymbol{(}\AgdaBound{p} \AgdaSymbol{:} \AgdaDatatype{Patch} \AgdaBound{s}\AgdaSymbol{)} \AgdaSymbol{→} \AgdaBound{p} \AgdaDatatype{⋙?} \AgdaInductiveConstructor{I}\<%
\\
\>[0]\AgdaIndent{2}{}\<[2]%
\>[2]\AgdaInductiveConstructor{Here-⋙?} \AgdaSymbol{:} \AgdaSymbol{∀} \AgdaSymbol{(}\AgdaBound{t₁} \AgdaBound{t₂} \AgdaBound{t₃} \AgdaSymbol{:} \AgdaDatatype{Tree}\AgdaSymbol{)} \AgdaSymbol{→} \AgdaInductiveConstructor{⟨} \AgdaBound{t₁} \AgdaInductiveConstructor{⇒} \AgdaBound{t₂} \AgdaInductiveConstructor{⟩} \AgdaDatatype{⋙?} \AgdaInductiveConstructor{⟨} \AgdaBound{t₂} \AgdaInductiveConstructor{⇒} \AgdaBound{t₃} \AgdaInductiveConstructor{⟩}\<%
\\
\>[0]\AgdaIndent{2}{}\<[2]%
\>[2]\AgdaInductiveConstructor{Branch-⋙?} \AgdaSymbol{:} \AgdaSymbol{∀} \AgdaSymbol{\{}\AgdaBound{s₁} \AgdaBound{s₂} \AgdaBound{s₃} \AgdaBound{s₄}\AgdaSymbol{\}} \<[30]%
\>[30]\<%
\\
\>[2]\AgdaIndent{4}{}\<[4]%
\>[4]\AgdaSymbol{→} \AgdaSymbol{\{}\AgdaBound{p₁} \AgdaSymbol{:} \AgdaDatatype{Patch} \AgdaBound{s₁}\AgdaSymbol{\}} \AgdaSymbol{\{}\AgdaBound{p₂} \AgdaSymbol{:} \AgdaDatatype{Patch} \AgdaBound{s₂}\AgdaSymbol{\}} \AgdaSymbol{\{}\AgdaBound{p₃} \AgdaSymbol{:} \AgdaDatatype{Patch} \AgdaBound{s₃}\AgdaSymbol{\}} \AgdaSymbol{\{}\AgdaBound{p₄} \AgdaSymbol{:} \AgdaDatatype{Patch} \AgdaBound{s₄}\AgdaSymbol{\}}\<%
\\
\>[2]\AgdaIndent{4}{}\<[4]%
\>[4]\AgdaSymbol{→} \AgdaSymbol{(}\AgdaBound{L} \AgdaSymbol{:} \AgdaBound{p₁} \AgdaDatatype{⋙?} \AgdaBound{p₂}\AgdaSymbol{)} \AgdaSymbol{→} \AgdaSymbol{(}\AgdaBound{R} \AgdaSymbol{:} \AgdaBound{p₃} \AgdaDatatype{⋙?} \AgdaBound{p₄}\AgdaSymbol{)} \AgdaSymbol{→} \AgdaInductiveConstructor{⟨} \AgdaBound{p₁} \AgdaInductiveConstructor{∧} \AgdaBound{p₃} \AgdaInductiveConstructor{⟩} \AgdaDatatype{⋙?} \AgdaInductiveConstructor{⟨} \AgdaBound{p₂} \AgdaInductiveConstructor{∧} \AgdaBound{p₄} \AgdaInductiveConstructor{⟩}\<%
\\
%
\\
\>\AgdaFunction{⋙?-uniq} \AgdaSymbol{:} \AgdaSymbol{∀} \AgdaSymbol{\{}\AgdaBound{s₁} \AgdaBound{s₂} \AgdaSymbol{:} \AgdaDatatype{Form}\AgdaSymbol{\}} \AgdaSymbol{\{}\AgdaBound{p₁} \AgdaSymbol{:} \AgdaDatatype{Patch} \AgdaBound{s₁}\AgdaSymbol{\}} \AgdaSymbol{\{}\AgdaBound{p₂} \AgdaSymbol{:} \AgdaDatatype{Patch} \AgdaBound{s₂}\AgdaSymbol{\}}\<%
\\
\>[0]\AgdaIndent{2}{}\<[2]%
\>[2]\AgdaSymbol{→} \AgdaSymbol{(}\AgdaBound{?₁} \AgdaBound{?₂} \AgdaSymbol{:} \AgdaBound{p₁} \AgdaDatatype{⋙?} \AgdaBound{p₂}\AgdaSymbol{)} \AgdaSymbol{→} \AgdaBound{?₁} \AgdaDatatype{≡} \AgdaBound{?₂}\<%
\\
\>\AgdaFunction{⋙?-uniq} \AgdaSymbol{(}\AgdaInductiveConstructor{✶⋙?I} \AgdaBound{p₁}\AgdaSymbol{)} \AgdaSymbol{(}\AgdaInductiveConstructor{✶⋙?I} \AgdaSymbol{.}\AgdaBound{p₁}\AgdaSymbol{)} \AgdaSymbol{=} \AgdaInductiveConstructor{refl}\<%
\\
\>\AgdaFunction{⋙?-uniq} \AgdaSymbol{(}\AgdaInductiveConstructor{Here-⋙?} \AgdaBound{t₁} \AgdaBound{t₂} \AgdaBound{t₃}\AgdaSymbol{)} \AgdaSymbol{(}\AgdaInductiveConstructor{Here-⋙?} \AgdaSymbol{.}\AgdaBound{t₁} \AgdaSymbol{.}\AgdaBound{t₂} \AgdaSymbol{.}\AgdaBound{t₃}\AgdaSymbol{)} \AgdaSymbol{=} \AgdaInductiveConstructor{refl}\<%
\\
\>\AgdaFunction{⋙?-uniq} \AgdaSymbol{(}\AgdaInductiveConstructor{Branch-⋙?} \AgdaBound{?₁} \AgdaBound{?₂}\AgdaSymbol{)} \AgdaSymbol{(}\AgdaInductiveConstructor{Branch-⋙?} \AgdaBound{?₃} \AgdaBound{?₄}\AgdaSymbol{)} \<[44]%
\>[44]\<%
\\
\>[0]\AgdaIndent{2}{}\<[2]%
\>[2]\AgdaKeyword{rewrite} \AgdaFunction{⋙?-uniq} \AgdaBound{?₁} \AgdaBound{?₃} \AgdaSymbol{|} \AgdaFunction{⋙?-uniq} \AgdaBound{?₂} \AgdaBound{?₄} \AgdaSymbol{=} \AgdaInductiveConstructor{refl}\<%
\\
%
\\
\>\AgdaFunction{⋙} \AgdaSymbol{:} \AgdaSymbol{\{}\AgdaBound{s₁} \AgdaBound{s₂} \AgdaSymbol{:} \AgdaDatatype{Form}\AgdaSymbol{\}} \AgdaSymbol{\{}\AgdaBound{p₁} \AgdaSymbol{:} \AgdaDatatype{Patch} \AgdaBound{s₁}\AgdaSymbol{\}} \AgdaSymbol{\{}\AgdaBound{p₂} \AgdaSymbol{:} \AgdaDatatype{Patch} \AgdaBound{s₂}\AgdaSymbol{\}}\<%
\\
\>[0]\AgdaIndent{2}{}\<[2]%
\>[2]\AgdaSymbol{→} \AgdaBound{p₁} \AgdaDatatype{⋙?} \AgdaBound{p₂} \AgdaSymbol{→} \AgdaDatatype{Patch} \AgdaBound{s₁}\<%
\\
\>\AgdaFunction{⋙} \AgdaSymbol{(}\AgdaInductiveConstructor{✶⋙?I} \AgdaBound{p}\AgdaSymbol{)} \AgdaSymbol{=} \AgdaBound{p}\<%
\\
\>\AgdaFunction{⋙} \AgdaSymbol{(}\AgdaInductiveConstructor{Here-⋙?} \AgdaBound{t₁} \AgdaBound{t₂} \AgdaBound{t₃}\AgdaSymbol{)} \AgdaSymbol{=} \AgdaInductiveConstructor{⟨} \AgdaBound{t₁} \AgdaInductiveConstructor{⇒} \AgdaBound{t₃} \AgdaInductiveConstructor{⟩}\<%
\\
\>\AgdaFunction{⋙} \AgdaSymbol{(}\AgdaInductiveConstructor{Branch-⋙?} \AgdaBound{p₁⋙?p₂} \AgdaBound{p₃⋙?p₄}\AgdaSymbol{)} \AgdaSymbol{=} \AgdaInductiveConstructor{⟨} \AgdaFunction{⋙} \AgdaBound{p₁⋙?p₂} \AgdaInductiveConstructor{∧} \AgdaFunction{⋙} \AgdaBound{p₃⋙?p₄} \AgdaInductiveConstructor{⟩}\<%
\\
%
\\
\>\AgdaFunction{⋙-preserves} \AgdaSymbol{:} \AgdaSymbol{∀} \AgdaSymbol{\{}\AgdaBound{s₁} \AgdaBound{s₂} \AgdaSymbol{:} \AgdaDatatype{Form}\AgdaSymbol{\}} \AgdaSymbol{\{}\AgdaBound{p₁} \AgdaSymbol{:} \AgdaDatatype{Patch} \AgdaBound{s₁}\AgdaSymbol{\}} \AgdaSymbol{\{}\AgdaBound{p₂} \AgdaSymbol{:} \AgdaDatatype{Patch} \AgdaBound{s₂}\AgdaSymbol{\}}\<%
\\
\>[0]\AgdaIndent{2}{}\<[2]%
\>[2]\AgdaSymbol{→} \AgdaSymbol{(}\AgdaBound{?₁} \AgdaBound{?₂} \AgdaSymbol{:} \AgdaBound{p₁} \AgdaDatatype{⋙?} \AgdaBound{p₂}\AgdaSymbol{)} \AgdaSymbol{→} \AgdaFunction{⋙} \AgdaBound{?₁} \AgdaFunction{⟷} \AgdaFunction{⋙} \AgdaBound{?₂}\<%
\\
\>\AgdaFunction{⋙-preserves} \AgdaSymbol{(}\AgdaInductiveConstructor{✶⋙?I} \AgdaBound{p₁}\AgdaSymbol{)} \AgdaSymbol{(}\AgdaInductiveConstructor{✶⋙?I} \AgdaSymbol{.}\AgdaBound{p₁}\AgdaSymbol{)} \AgdaSymbol{=} \AgdaFunction{⟷-refl} \AgdaBound{p₁}\<%
\\
\>\AgdaFunction{⋙-preserves} \AgdaSymbol{(}\AgdaInductiveConstructor{Here-⋙?} \AgdaBound{t₁} \AgdaBound{t₂} \AgdaBound{t₃}\AgdaSymbol{)} \AgdaSymbol{(}\AgdaInductiveConstructor{Here-⋙?} \AgdaSymbol{.}\AgdaBound{t₁} \AgdaSymbol{.}\AgdaBound{t₂} \AgdaSymbol{.}\AgdaBound{t₃}\AgdaSymbol{)} \AgdaSymbol{=} \<[55]%
\>[55]\<%
\\
\>[0]\AgdaIndent{2}{}\<[2]%
\>[2]\AgdaFunction{⟷-refl} \AgdaInductiveConstructor{⟨} \AgdaBound{t₁} \AgdaInductiveConstructor{⇒} \AgdaBound{t₃} \AgdaInductiveConstructor{⟩}\<%
\\
\>\AgdaFunction{⋙-preserves} \AgdaSymbol{(}\AgdaInductiveConstructor{Branch-⋙?} \AgdaBound{?₁} \AgdaBound{?₂}\AgdaSymbol{)} \AgdaSymbol{(}\AgdaInductiveConstructor{Branch-⋙?} \AgdaBound{?₃} \AgdaBound{?₄}\AgdaSymbol{)} \AgdaSymbol{=} \<[50]%
\>[50]\<%
\\
\>[0]\AgdaIndent{2}{}\<[2]%
\>[2]\AgdaFunction{⟷-branch} \AgdaSymbol{(}\AgdaFunction{⋙-preserves} \AgdaBound{?₁} \AgdaBound{?₃}\AgdaSymbol{)} \AgdaSymbol{(}\AgdaFunction{⋙-preserves} \AgdaBound{?₂} \AgdaBound{?₄}\AgdaSymbol{)}\<%
\\
%
\\
\>\AgdaFunction{⋙-assoc} \AgdaSymbol{:} \AgdaSymbol{∀} \AgdaSymbol{\{}\AgdaBound{s₁} \AgdaBound{s₂} \AgdaBound{s₃} \AgdaSymbol{:} \AgdaDatatype{Form}\AgdaSymbol{\}} \AgdaSymbol{\{}\AgdaBound{p₁} \AgdaSymbol{:} \AgdaDatatype{Patch} \AgdaBound{s₁}\AgdaSymbol{\}} \AgdaSymbol{\{}\AgdaBound{p₂} \AgdaSymbol{:} \AgdaDatatype{Patch} \AgdaBound{s₂}\AgdaSymbol{\}} \AgdaSymbol{\{}\AgdaBound{p₃} \AgdaSymbol{:} \AgdaDatatype{Patch} \AgdaBound{s₃}\AgdaSymbol{\}}\<%
\\
\>[0]\AgdaIndent{2}{}\<[2]%
\>[2]\AgdaSymbol{→} \AgdaSymbol{(}\AgdaBound{1⋙?2} \AgdaSymbol{:} \AgdaBound{p₁} \AgdaDatatype{⋙?} \AgdaBound{p₂}\AgdaSymbol{)}\<%
\\
\>[0]\AgdaIndent{2}{}\<[2]%
\>[2]\AgdaSymbol{→} \AgdaSymbol{(}\AgdaBound{[1⋙2]⋙?3} \AgdaSymbol{:} \AgdaSymbol{(}\AgdaFunction{⋙} \AgdaBound{1⋙?2}\AgdaSymbol{)} \AgdaDatatype{⋙?} \AgdaBound{p₃}\AgdaSymbol{)}\<%
\\
\>[0]\AgdaIndent{2}{}\<[2]%
\>[2]\AgdaSymbol{→} \AgdaSymbol{(}\AgdaBound{2⋙?3} \AgdaSymbol{:} \AgdaBound{p₂} \AgdaDatatype{⋙?} \AgdaBound{p₃}\AgdaSymbol{)}\<%
\\
\>[0]\AgdaIndent{2}{}\<[2]%
\>[2]\AgdaSymbol{→} \AgdaSymbol{(}\AgdaBound{1⋙?[2⋙3]} \AgdaSymbol{:} \AgdaBound{p₁} \AgdaDatatype{⋙?} \AgdaSymbol{(}\AgdaFunction{⋙} \AgdaBound{2⋙?3}\AgdaSymbol{))}\<%
\\
\>[0]\AgdaIndent{2}{}\<[2]%
\>[2]\AgdaSymbol{→} \AgdaSymbol{(}\AgdaFunction{⋙} \AgdaBound{[1⋙2]⋙?3}\AgdaSymbol{)} \AgdaFunction{⟷} \AgdaSymbol{(}\AgdaFunction{⋙} \AgdaBound{1⋙?[2⋙3]}\AgdaSymbol{)}\<%
\\
\>\AgdaFunction{⋙-assoc} \AgdaSymbol{(}\AgdaInductiveConstructor{✶⋙?I} \AgdaBound{p₁}\AgdaSymbol{)} \AgdaSymbol{(}\AgdaInductiveConstructor{✶⋙?I} \AgdaSymbol{.}\AgdaBound{p₁}\AgdaSymbol{)} \AgdaSymbol{(}\AgdaInductiveConstructor{✶⋙?I} \AgdaSymbol{.}\AgdaInductiveConstructor{I}\AgdaSymbol{)} \AgdaSymbol{(}\AgdaInductiveConstructor{✶⋙?I} \AgdaSymbol{.}\AgdaBound{p₁}\AgdaSymbol{)} \AgdaSymbol{=} \AgdaFunction{⟷-refl} \AgdaBound{p₁}\<%
\\
\>\AgdaFunction{⋙-assoc} \AgdaSymbol{(}\AgdaInductiveConstructor{Here-⋙?} \AgdaBound{t₁} \AgdaBound{t₂} \AgdaBound{t₃}\AgdaSymbol{)} \AgdaSymbol{(}\AgdaInductiveConstructor{✶⋙?I} \AgdaSymbol{.(}\AgdaInductiveConstructor{⟨} \AgdaBound{t₁} \AgdaInductiveConstructor{⇒} \AgdaBound{t₃} \AgdaInductiveConstructor{⟩}\AgdaSymbol{)}\AgdaSymbol{)} \AgdaSymbol{(}\AgdaInductiveConstructor{✶⋙?I} \AgdaSymbol{.(}\AgdaInductiveConstructor{⟨} \AgdaBound{t₂} \AgdaInductiveConstructor{⇒} \AgdaBound{t₃} \AgdaInductiveConstructor{⟩}\AgdaSymbol{)}\AgdaSymbol{)} \<[71]%
\>[71]\<%
\\
\>[0]\AgdaIndent{2}{}\<[2]%
\>[2]\AgdaSymbol{(}\AgdaInductiveConstructor{Here-⋙?} \AgdaSymbol{.}\AgdaBound{t₁} \AgdaSymbol{.}\AgdaBound{t₂} \AgdaSymbol{.}\AgdaBound{t₃}\AgdaSymbol{)} \AgdaSymbol{=} \AgdaFunction{⟷-refl} \AgdaInductiveConstructor{⟨} \AgdaBound{t₁} \AgdaInductiveConstructor{⇒} \AgdaBound{t₃} \AgdaInductiveConstructor{⟩}\<%
\\
\>\AgdaFunction{⋙-assoc} \AgdaSymbol{(}\AgdaInductiveConstructor{Here-⋙?} \AgdaBound{t₁} \AgdaBound{t₂} \AgdaBound{t₃}\AgdaSymbol{)} \AgdaSymbol{(}\AgdaInductiveConstructor{Here-⋙?} \AgdaSymbol{.}\AgdaBound{t₁} \AgdaSymbol{.}\AgdaBound{t₃} \AgdaBound{t₄}\AgdaSymbol{)} \AgdaSymbol{(}\AgdaInductiveConstructor{Here-⋙?} \AgdaSymbol{.}\AgdaBound{t₂} \AgdaSymbol{.}\AgdaBound{t₃} \AgdaSymbol{.}\AgdaBound{t₄}\AgdaSymbol{)} \<[70]%
\>[70]\<%
\\
\>[0]\AgdaIndent{2}{}\<[2]%
\>[2]\AgdaSymbol{(}\AgdaInductiveConstructor{Here-⋙?} \AgdaSymbol{.}\AgdaBound{t₁} \AgdaSymbol{.}\AgdaBound{t₂} \AgdaSymbol{.}\AgdaBound{t₄}\AgdaSymbol{)} \AgdaSymbol{=} \AgdaFunction{⟷-refl} \AgdaInductiveConstructor{⟨} \AgdaBound{t₁} \AgdaInductiveConstructor{⇒} \AgdaBound{t₄} \AgdaInductiveConstructor{⟩}\<%
\\
\>\AgdaFunction{⋙-assoc} \<[8]%
\>[8]\<%
\\
\>[0]\AgdaIndent{2}{}\<[2]%
\>[2]\AgdaSymbol{(}\AgdaInductiveConstructor{Branch-⋙?} \AgdaBound{1⋙?2} \AgdaBound{3⋙?4}\AgdaSymbol{)} \<[24]%
\>[24]\<%
\\
\>[0]\AgdaIndent{2}{}\<[2]%
\>[2]\AgdaSymbol{(}\AgdaInductiveConstructor{✶⋙?I} \AgdaSymbol{.(}\AgdaInductiveConstructor{⟨} \AgdaFunction{⋙} \AgdaBound{1⋙?2} \AgdaInductiveConstructor{∧} \AgdaFunction{⋙} \AgdaBound{3⋙?4} \AgdaInductiveConstructor{⟩}\AgdaSymbol{)}\AgdaSymbol{)} \<[32]%
\>[32]\<%
\\
\>[0]\AgdaIndent{2}{}\<[2]%
\>[2]\AgdaSymbol{(}\AgdaInductiveConstructor{✶⋙?I} \AgdaSymbol{.\_}\AgdaSymbol{)} \<[12]%
\>[12]\<%
\\
\>[0]\AgdaIndent{2}{}\<[2]%
\>[2]\AgdaSymbol{(}\AgdaInductiveConstructor{Branch-⋙?} \AgdaBound{1⋙?[2⋙0]} \AgdaBound{3⋙?[4⋙0]}\AgdaSymbol{)} \<[32]%
\>[32]\<%
\\
\>[0]\AgdaIndent{2}{}\<[2]%
\>[2]\AgdaSymbol{=} \AgdaFunction{⟷-branch} \AgdaSymbol{(}\AgdaFunction{⋙-preserves} \AgdaBound{1⋙?2} \AgdaBound{1⋙?[2⋙0]}\AgdaSymbol{)} \<[41]%
\>[41]\<%
\\
\>[2]\AgdaIndent{4}{}\<[4]%
\>[4]\AgdaSymbol{(}\AgdaFunction{⋙-preserves} \AgdaBound{3⋙?4} \AgdaBound{3⋙?[4⋙0]}\AgdaSymbol{)}\<%
\\
\>\AgdaFunction{⋙-assoc} \<[8]%
\>[8]\<%
\\
\>[0]\AgdaIndent{2}{}\<[2]%
\>[2]\AgdaSymbol{(}\AgdaInductiveConstructor{Branch-⋙?} \AgdaBound{1⋙?2} \AgdaBound{3⋙?4}\AgdaSymbol{)} \<[24]%
\>[24]\<%
\\
\>[0]\AgdaIndent{2}{}\<[2]%
\>[2]\AgdaSymbol{(}\AgdaInductiveConstructor{Branch-⋙?} \AgdaBound{[1⋙2]⋙?5} \AgdaBound{[3⋙4]⋙?6}\AgdaSymbol{)} \<[32]%
\>[32]\<%
\\
\>[0]\AgdaIndent{2}{}\<[2]%
\>[2]\AgdaSymbol{(}\AgdaInductiveConstructor{Branch-⋙?} \AgdaBound{2⋙?5} \AgdaBound{4⋙?6}\AgdaSymbol{)} \<[24]%
\>[24]\<%
\\
\>[0]\AgdaIndent{2}{}\<[2]%
\>[2]\AgdaSymbol{(}\AgdaInductiveConstructor{Branch-⋙?} \AgdaBound{1⋙?[2⋙5]} \AgdaBound{3⋙?[4⋙6]}\AgdaSymbol{)} \<[32]%
\>[32]\<%
\\
\>[0]\AgdaIndent{2}{}\<[2]%
\>[2]\AgdaSymbol{=} \AgdaFunction{⟷-branch} \<[13]%
\>[13]\<%
\\
\>[2]\AgdaIndent{4}{}\<[4]%
\>[4]\AgdaSymbol{(}\AgdaFunction{⋙-assoc} \AgdaBound{1⋙?2} \AgdaBound{[1⋙2]⋙?5} \AgdaBound{2⋙?5} \AgdaBound{1⋙?[2⋙5]}\AgdaSymbol{)} \<[42]%
\>[42]\<%
\\
\>[2]\AgdaIndent{4}{}\<[4]%
\>[4]\AgdaSymbol{(}\AgdaFunction{⋙-assoc} \AgdaBound{3⋙?4} \AgdaBound{[3⋙4]⋙?6} \AgdaBound{4⋙?6} \AgdaBound{3⋙?[4⋙6]}\AgdaSymbol{)}\<%
\\
%
\\
\>\AgdaKeyword{module} \AgdaModule{⋙-try1} \AgdaKeyword{where}\<%
\\
\>[0]\AgdaIndent{2}{}\<[2]%
\>[2]\AgdaKeyword{data} \AgdaDatatype{\_⋙??\_} \AgdaSymbol{:} \AgdaSymbol{\{}\AgdaBound{s₁} \AgdaBound{s₂} \AgdaSymbol{:} \AgdaDatatype{Form}\AgdaSymbol{\}} \AgdaSymbol{(}\AgdaBound{p₁} \AgdaSymbol{:} \AgdaDatatype{Patch} \AgdaBound{s₁}\AgdaSymbol{)} \AgdaSymbol{(}\AgdaBound{p₂} \AgdaSymbol{:} \AgdaDatatype{Patch} \AgdaBound{s₂}\AgdaSymbol{)} \AgdaSymbol{→} \AgdaPrimitiveType{Set} \AgdaKeyword{where}\<%
\\
\>[2]\AgdaIndent{4}{}\<[4]%
\>[4]\AgdaInductiveConstructor{✶⋙?I} \AgdaSymbol{:} \AgdaSymbol{∀} \AgdaSymbol{\{}\AgdaBound{s} \AgdaSymbol{:} \AgdaDatatype{Form}\AgdaSymbol{\}} \AgdaSymbol{(}\AgdaBound{p} \AgdaSymbol{:} \AgdaDatatype{Patch} \AgdaBound{s}\AgdaSymbol{)} \AgdaSymbol{→} \AgdaBound{p} \AgdaDatatype{⋙??} \AgdaInductiveConstructor{I}\<%
\\
\>[2]\AgdaIndent{4}{}\<[4]%
\>[4]\AgdaInductiveConstructor{Here-⋙?} \AgdaSymbol{:} \AgdaSymbol{∀} \AgdaSymbol{(}\AgdaBound{t₁} \AgdaBound{t₂} \AgdaBound{t₃} \AgdaSymbol{:} \AgdaDatatype{Tree}\AgdaSymbol{)} \AgdaSymbol{→} \AgdaInductiveConstructor{⟨} \AgdaBound{t₁} \AgdaInductiveConstructor{⇒} \AgdaBound{t₂} \AgdaInductiveConstructor{⟩} \AgdaDatatype{⋙??} \AgdaInductiveConstructor{⟨} \AgdaBound{t₂} \AgdaInductiveConstructor{⇒} \AgdaBound{t₃} \AgdaInductiveConstructor{⟩}\<%
\\
\>[2]\AgdaIndent{4}{}\<[4]%
\>[4]\AgdaInductiveConstructor{There-⋙?} \AgdaSymbol{:} \AgdaSymbol{∀} \AgdaSymbol{\{}\AgdaBound{s₁} \AgdaBound{s₂} \AgdaSymbol{:} \AgdaDatatype{Form}\AgdaSymbol{\}} \AgdaSymbol{\{}\AgdaBound{t₁} \AgdaBound{t₂} \AgdaSymbol{:} \AgdaDatatype{Tree}\AgdaSymbol{\}} \AgdaSymbol{\{}\AgdaBound{p₁} \AgdaSymbol{:} \AgdaDatatype{Patch} \AgdaBound{s₁}\AgdaSymbol{\}} \AgdaSymbol{\{}\AgdaBound{p₂} \AgdaSymbol{:} \AgdaDatatype{Patch} \AgdaBound{s₂}\AgdaSymbol{\}}\<%
\\
\>[4]\AgdaIndent{6}{}\<[6]%
\>[6]\AgdaSymbol{→} \AgdaSymbol{(}\AgdaBound{t} \AgdaSymbol{:} \AgdaDatatype{Tree}\AgdaSymbol{)} \AgdaSymbol{→} \AgdaBound{p₁} \AgdaDatatype{⊏} \AgdaBound{t₁} \AgdaSymbol{→} \AgdaBound{p₂} \AgdaDatatype{⊏} \AgdaBound{t₂} \<[39]%
\>[39]\<%
\\
\>[4]\AgdaIndent{6}{}\<[6]%
\>[6]\AgdaSymbol{→} \AgdaInductiveConstructor{⟨} \AgdaBound{t} \AgdaInductiveConstructor{⇒} \AgdaInductiveConstructor{Branch} \AgdaBound{t₁} \AgdaBound{t₂} \AgdaInductiveConstructor{⟩} \AgdaDatatype{⋙??} \AgdaInductiveConstructor{⟨} \AgdaBound{p₁} \AgdaInductiveConstructor{∧} \AgdaBound{p₂} \AgdaInductiveConstructor{⟩}\<%
\\
\>[0]\AgdaIndent{4}{}\<[4]%
\>[4]\AgdaInductiveConstructor{Branch-⋙?} \AgdaSymbol{:} \AgdaSymbol{∀} \AgdaSymbol{\{}\AgdaBound{s₁} \AgdaBound{s₂} \AgdaBound{s₃} \AgdaBound{s₄}\AgdaSymbol{\}} \<[32]%
\>[32]\<%
\\
\>[4]\AgdaIndent{6}{}\<[6]%
\>[6]\AgdaSymbol{→} \AgdaSymbol{\{}\AgdaBound{p₁} \AgdaSymbol{:} \AgdaDatatype{Patch} \AgdaBound{s₁}\AgdaSymbol{\}} \AgdaSymbol{\{}\AgdaBound{p₂} \AgdaSymbol{:} \AgdaDatatype{Patch} \AgdaBound{s₂}\AgdaSymbol{\}} \AgdaSymbol{\{}\AgdaBound{p₃} \AgdaSymbol{:} \AgdaDatatype{Patch} \AgdaBound{s₃}\AgdaSymbol{\}} \AgdaSymbol{\{}\AgdaBound{p₄} \AgdaSymbol{:} \AgdaDatatype{Patch} \AgdaBound{s₄}\AgdaSymbol{\}}\<%
\\
\>[4]\AgdaIndent{6}{}\<[6]%
\>[6]\AgdaSymbol{→} \AgdaSymbol{(}\AgdaBound{L} \AgdaSymbol{:} \AgdaBound{p₁} \AgdaDatatype{⋙??} \AgdaBound{p₂}\AgdaSymbol{)} \AgdaSymbol{→} \AgdaSymbol{(}\AgdaBound{R} \AgdaSymbol{:} \AgdaBound{p₃} \AgdaDatatype{⋙??} \AgdaBound{p₄}\AgdaSymbol{)} \AgdaSymbol{→} \AgdaInductiveConstructor{⟨} \AgdaBound{p₁} \AgdaInductiveConstructor{∧} \AgdaBound{p₃} \AgdaInductiveConstructor{⟩} \AgdaDatatype{⋙??} \AgdaInductiveConstructor{⟨} \AgdaBound{p₂} \AgdaInductiveConstructor{∧} \AgdaBound{p₄} \AgdaInductiveConstructor{⟩}\<%
\\
\>\<%
\end{code}

\end{appendices}


\end{document}
