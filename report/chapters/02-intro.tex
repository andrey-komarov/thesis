\section{Определения}

Перед тем, как приступать к непосредственному описанию результатов,
введём некоторые неформальные определения, которыми будем в дальнейшем
руководствоваваться. 

\begin{definition}[Ревизия]
  Назовём \emph{ревизией} состояние, которое может быть сохранено в
  системе контроля версий. Фактически, это просто будет некоторое
  состояние какой-то структуры данных.
\end{definition}

Напрмер, \emph{ревизией} в git-репозитории является список файлов и их
содержимое. О том, какие будут рассмотрены примеры ревизий, будет
сказано позже.

\begin{definition}[Патч]
  \emph{Патч}~--- нечто, описывающее преобразование одной ревизии в
  другую.
\end{definition}

Например, если ревизия~--- обычный текстовый файл, то в качестве
патча, преобразующего файл \texttt{a.txt} в файл \texttt{b.txt} можно
взять выдачу утилиты UNIX diff: \texttt{diff a.txt b.txt}. 

\begin{definition}[Система контроля версий (VCS)]
  В рамках данной работы \emph{системой контроля версий} будем
  называть некую структуру, хранящую ревизии и патчи вместе с
  интерфейсом доступа к ней.
\end{definition}

В это определение вписываются все рассмотренные выше системы контроля
версий. 

Будем реализовывать на языке программирования Agda \emph{систему
  контроля версий}, поддерживающую следующие операции:
\begin{itemize}
\item неконфликтующее объединение патчей;
\item конфликтующее объединение патчей.
\end{itemize}

\begin{definition}[Неконфликтующее объединение]
  \emph{Неконфликтующее объединение} двух патчей~--- патч, обладающий
  эффектами обоих объединяемых патчей. При этом, объединяемые патчи не
  должны \emph{конфликтовать} друг с другом. Определение
  \emph{конфликтования} патчей будет дано позднее.
\end{definition}

Например, патчи <<добавить пустой файл \texttt{a.txt}>> и <<добавить
пустой файл \texttt{b.txt}>>~--- \emph{неконфликтующие} (при условии
отсутсвия до их применения обоих этих файлов.

\begin{definition}[Конфликтующее объединение]
  \emph{Конфликтующее объединение} двух патчей~--- патч, получающийся
  в результате последовательного применения сначала первого, а затем
  второго патчей. В данной работе будет рассматриваться конфликтующее
  объединение, при котором второй патч может менять \emph{только} то,
  что уже поменял первый.
\end{definition}

Будем решать поставленную задачу с использованием языка
программирования Agda. 

СЮДА НАПИСАТЬ ПРО ДЕРЕВЬЯ
