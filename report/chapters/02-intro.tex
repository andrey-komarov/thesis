% ты этим пользуешься уже в первой главе, потому это надо скорее туда

\section{Определения}

Будем реализовывать на языке программирования Agda \emph{систему
  контроля версий}, поддерживающую следующие операции:
\begin{itemize}
\item неконфликтующее объединение патчей;
\item конфликтующее объединение патчей.
\end{itemize}

\begin{definition}[Неконфликтующее объединение]
  \emph{Неконфликтующее объединение} двух патчей~--- патч, обладающий
  эффектами обоих объединяемых патчей. При этом, объединяемые патчи не
  должны \emph{конфликтовать} друг с другом. Определение
  \emph{конфликтования} патчей будет дано позднее.
\end{definition}

Например, патчи <<добавить пустой файл \texttt{a.txt}>> и <<добавить
пустой файл \texttt{b.txt}>>~--- \emph{неконфликтующие} (при условии
отсутсвия до их применения обоих этих файлов.

\begin{definition}[Конфликтующее объединение]
  \emph{Конфликтующее объединение} двух патчей~--- патч, получающийся
  в результате последовательного применения сначала первого, а затем
  второго патчей. В данной работе будет рассматриваться конфликтующее
  объединение, при котором второй патч может менять \emph{только} то,
  что уже поменял первый.
\end{definition}

Для определения того, \emph{конфликтуют} ли два патча, введём понятие
\emph{формы} патча.

\begin{definition}[Форма патча]
  \emph{Форма} патча~--- нечто, что характеризует места ревизии, в
  которые патч вносит изменения. Для каждой системы контроля версий
  это будет что-то своё. Посмотрев на две формы, можно сказать, будут
  ли конфликтовать два обладающих ими патча.
\end{definition}

Было реализовано две системы контроля версий для разных видов
ревизий~(рисунок \ref{fig:repotypes}):
\begin{itemize}
\item вектора фиксированной длины~(рисунок~\ref{fig:repotypes-vec});
\item двоичные деревья с элементами в
  листьях~(рисунок~\ref{fig:repotypes-tree}).
\end{itemize}

Для этих систем контроля версий и операций над ними были доказаны
некоторые свойства.

\begin{figure}
  \centering
  \begin{subfigure}[b]{0.45\textwidth}
    \centering
    \begin{tikzpicture}
      \matrix [draw=black] 
      {\vecd{a} & \vecd{b} & \vecd{c} & 
        \vecd{d} & \vecd{e} & \vecd{f} \\};
    \end{tikzpicture}
    \caption{Вектор константной длины}
    \label{fig:repotypes-vec}
  \end{subfigure}
  \begin{subfigure}[b]{0.45\textwidth}
    \centering
    \begin{tikzpicture}
      \node[treev] {}
      child[treearr] {node[treev] {}
        child[treearr] {node[treev] {a}
        }
        child[treearr] {node[treev] {b}
        }
      }
      child[treearr] {node[treev] {c}
      };
    \end{tikzpicture}
    \caption{Двоичное дерево с элементами в листьях}
    \label{fig:repotypes-tree}
  \end{subfigure}
  \caption{Виды ревизий}
  \label{fig:repotypes}
\end{figure}

