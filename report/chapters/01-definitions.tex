% надо какой-то обобщяющий текст про то о чём тут в этой главе
% эдакое "заключение наоборот"

% и в остальных главах аналогично

\section{Определения}

Введём некоторые неформальные определения, которыми будем в дальнейшем
руководствоваваться. 

\begin{definition}[Ревизия]
  Назовём \emph{ревизией} состояние, которое может быть сохранено в
  системе контроля версий. Фактически, это просто будет некоторое
  состояние какой-то структуры данных.
\end{definition}

Напрмер, \emph{ревизией} в git-репозитории является список файлов и их
содержимое. 

\begin{definition}[Патч]
  \emph{Патч}~--- нечто, описывающее преобразование одной ревизии в
  другую.
\end{definition}

Например, если ревизия~--- обычный текстовый файл, то в качестве
патча, преобразующего файл \texttt{a.txt} в файл \texttt{b.txt} можно
взять выдачу утилиты UNIX diff: \texttt{diff a.txt b.txt}. 

\begin{definition}[Система контроля версий (VCS)]
  В рамках данной работы \emph{системой контроля версий} будем
  называть некую структуру, хранящую ревизии и патчи вместе с
  интерфейсом доступа к ней.
\end{definition}

\begin{definition}[Коммит]
  Назовём \emph{коммитом} запомненное в репозитории состояние рабочей
  копии (файлов и их содержимого) в какой-либо момент времени.
\end{definition}

В эти определения вписываются все рассмотренные выше системы контроля
версий. 

