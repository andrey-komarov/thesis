\section{Решаемая задача}

Поставим следующую глобальную задачу. Разработка системы контроля
версий, свойства которой можно доказать. Имеющиеся работы можно
разделить на несколько групп:

\begin{enumerate}
\item Большинство систем контроля версий не претендуют ни на какую
  низлежащую теорию. В эту группу можно отнести CVS~\cite{cvs},
  Subversion~\cite{svnbook}, Git~\cite{progit},
  Mercurial~\cite{mercurial} % и прочие перфорсы
  и прочие обширно используемые системы контроля версий. Эти системы
  ориентированы на то, чтобы работать \emph{как можно лучше}, но при
  этом, формально доказать что-либо про их поведение не представляется
  возможным.
\item \emph{Darcs}~\cite{darcs}~--- первая система контроля версий,
  для которой разработана какая-либо теория. Используемая \emph{теория
    патчей} вводит алгебру над патчами, чётко регламентируя поведение
  системы в различных ситуациях.
\item \emph{Camp}~\cite{camp}~--- попытка формализации теории патчей
  Darcs на языке Coq. По состоянию на данный момент проект выглядит
  заброшенным.
\item JetBrains VCS.
TODO Где взять инфу?
\end{enumerate}

Как видно, систем, которые могут дать какие-либо гарантии, на данный
момент нет. В некоторых из имеющихся доказаны некоторые используемые
алгоритмы, но не написанный исходный код. Поэтому, осмысленным
является написание полностью верифицированной системы контроля версий
с использованием современных верификаторов доказательств.
Однако, эта задача слишком сложна и неосуществима в рамках данной
работы.

Целью этой работы является рассмотрение нескольких частных случаев
этой гипотетической системы и изучение их свойств. Реализация и
доказательства будут вестись на языке Agda.
