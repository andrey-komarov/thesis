\section{Решаемая задача}

% "Обобщая свойства существующих решений в отношении поставленной
% задачи имеющиеся VCS можно разделить на следующие классы ..."

\begin{enumerate}
\item Большинство систем контроля версий не претендуют ни на какую
  низлежащую теорию. В эту группу можно отнести CVS~\cite{cvs},
  Subversion~\cite{svnbook}, Git~\cite{progit},
  Mercurial~\cite{mercurial} % и прочие перфорсы
  и прочие обширно используемые системы контроля версий. Эти системы
  ориентированы на то, чтобы работать \emph{как можно лучше}, но при
  этом, формально доказать что-либо про их поведение не представляется
  возможным.
\item \emph{Darcs}~\cite{darcs}~--- первая система контроля версий,
  для которой разработана какая-либо теория. Используемая \emph{теория
    патчей} вводит алгебру над патчами, чётко регламентируя поведение
  системы в различных ситуациях.
\item \emph{Camp}~\cite{camp}~--- попытка формализации теории патчей
  Darcs на языке Coq. По состоянию на данный момент проект выглядит
  заброшенным.
\item JetBrains VCS.
TODO Где взять инфу?
% нигде :(
% просто написать, что мы знаем, что есть что-то но не знаем что
\end{enumerate}

Как видно, систем, которые могут дать какие-либо гарантии, на данный
момент нет. В некоторых из имеющихся доказаны некоторые используемые
алгоритмы, но не написанный исходный код. Поэтому, осмысленным
%                ^^^ wat?
является написание полностью верифицированной системы контроля версий
с использованием современных систем верификации доказательств.
Однако, эта задача слишком сложна и неосуществима в рамках данной
работы.

% повторяем последний кусок введения разворачивая мысль
Целью настоящей работы является разработка нескольких <<каркасов>> для
класса систем контроля версий
% на основе одной общей идеи
% <<логистического>> характера с их формализацией, формулировкой и
% доказательством некоторых их свойст на языке доказательного
% программирования Agda~\cite{agda}}.
%
% вот это нужно развернуть более подробно описав общую идею a-la
% "... заключается в том, чтобы представить историю изменений в системе
% контроля версий структурой данных, индексированной <<формой>> производимых
% ей изменений ..."
% предполагается, что тут читатель
% или a) уже знает достаточно чтобы понять о чём идет речь и дальше
%        читать "с ожиданием"
%     б) читаая дальше она сможет сюда вернуться за общей идеей если
%        мысль рассуждения будет потеряна
