\chapter{Обзор предметной области}
\label{chapter1}

\section{Darcs}

Darcs~--- распределённая система контроля версий, использующая внутри
себя специально разработанную \emph{теорию патчей}. В ней, в отличие
от большинства других распределённых систем контроля версий,
отсутствует понятие <<ветки>>: любое подмножество множетсва всех
имеющихся в данном репозитории патчей можно рассматривать как
\emph{рабочую копию}.

\subsection{Обзор}

В каждый момент времени времени сосуществуют следующие три
элемента: % TODO элемента? не самое удачное слово

\begin{itemize}
\item рабочая директория (Working directory),
\item чистое дерево (Pristine Tree),
\item множество патчей.
\end{itemize}

\begin{definition}[Рабочая директория]
  Единственный из этих трёх элементов, который (при нормальном
  использовании) меняет пользователь. Директория, в которой,
  непосредственно, находятся файлы, за которыми следит Darcs.

  Может содержать изменения, которые ещё не находятся под контролем.
\end{definition}

\begin{definition}[Множество патчей]
  Все патчи, с которыми когда-либо работали в рамках этого репозитория.
\end{definition}

\begin{definition}[Чистое дерево]
  Последнее \emph{записанное} состояние рабочей директории. При
  \emph{записи} рабочая директория будет сравниваться именно с ним.
  Представляет собой список патчей, которые надо применить к пустой
  директории, чтобы получить заданное дерево.
\end{definition}

Стандартный процесс работы с Darcs-репозиторием выглядит так. 

\begin{itemize}
\item Создание пустого репозитория (init);
\item Добавление файлов (add);
\item Запись изменённой рабочей директории (record);
\item Установка метки на текущее записанное состояние (tag);
\item Отправка своей рабочей копии на удалённый сервер (pull);
\end{itemize}

В этой работе не будет детально рассматриваться синтаксис работы с
каждой из них. Вместо этого будет более подробно рассмотрено
внутреннее строение системы.

\subsection{Виды патчей}

Каждый патч состоит из последовательности \emph{элементарных патчей}. 
В Darcs используются следующие виды элементарных патчей:

\begin{definition}[Элементарный патч]
\begin{itemize}
\item Создание пустой директории с именем $d$;
\item Удаление пустой директории с именем $d$;
\item Создание пустого файла с именем $f$;
\item Удаление пустого файла с именем $f$;
\item Добавление в файл $f$ после $n$-й строки строк $s_1, s_2, \ldots s_k$;
\item Удаление из файла $f$ после $n$-й строки строк $s_1, s_2, \ldots s_k$;
\item Замена во всём файле $f$ \emph{слова} $s$ на слово $t$, если
  слова $t$ до этого не встречалось.
\item 
\end{itemize}
\end{definition}

Очевидно, что применение даже элементарного патча возможно не всегда.
Например, нельзя создать файл, если файл с таким именем уже
существует. Или, нельзя удалить пятую строку <<Петя>>, если там
написано <<Вася>>.

\begin{definition}[Контекст]
  Множество всех рабочих директорий, к которым применим патч $p$,
  называется \emph{контекстом патча} $p$.
\end{definition}

\begin{notation}
  Будем обозначать как ${}^aA^b$ патч $A$, который после применения к
  контексту $a$ даёт контекст $b$. Последовательное применение сначала
  патча ${}^aA^b$, а затем, патча ${}^bB^c$ будем обозначать как
  ${}^aA^bB^c$ или, если это не вызывает никаких неоднозначностей,
  просто $AB$. 
\end{notation}

\subsection{Группа патчей}

При проектировании системы, элементарные патчи выбирались так, чтобы
для каждого элементарного патча ${}^aA^b$ существовал обратный
${}^b{A^{-1}}^a$ такой, что ${\cal E}({}^aA^b(A^{-1})^a) = {\cal
  E}({}^aI^a)$, где за ${\cal E}(A)$ обозначается \emph{эффект} от
патча $A$, а $I$~--- патч, который ничего не делает.

Подобно построению свободной группы, построим группу патчей над
элементарными патчами. Напомним, что патч~--- это последовательность
элементарных патчей. Будем считать, что никаких проблем во время
применения патчей в этой последовательности не возникает.

\begin{definition}[Группа патчей]
  В определении группы требуется существование следующего:

  \begin{itemize}
  \item Нулевой элемент~--- пустая последовательность;
  \item Обратный элемент
    $({}^{c_1}P_1^{c_2}P_2^{c_3}\ldots P_n^{c_{n+1}})^{-1} =
    {}^{c_{n+1}}P_n^{c_n}\ldots P_2^{c_2}P_1^{c_1}$;
  \item Групповая операция~--- последовательное применение сначала
    первого патча, а затем, второго.
  \end{itemize}
\end{definition}

\subsection{Коммутация патчей}

Коммутация патчей~--- один из важнейших механизмов в Darcs. С помощью
неё возможна реализация функциональности merge~--- объединения
результатов работы нескольких авторов.

\begin{definition}[Коммутация патчей]
  Патчи ${}^ap^b$ и ${}^bq^c$ \emph{коммутируют}, если существует
  такой контекст $d$ и такие патчи ${}^d{p'}^c$ и ${}^a{q'}^d$, что
  ${\cal E}(p) = {\cal E}(p')$ и ${\cal E}(q) = {\cal E}(q')$.

  Если такие $p'$ и $q'$ существуют, то пишут, что $p, q
  \longleftrightarrow q', p'$.
\end{definition}

Другими словами, патчи коммутируют, если их в последовательном
применении можно поменять местами без ущерба для результата.
Очевидно, что патчи коммутируют не всегда. Например, патчи <<создание
файла $f$>> и <<запись в первую строку файла $f$ строки $s$>> не
коммутируют~--- нельзя сначала записать что-то в файл, а потом его
создать. Патчи же $p = $ <<замена в файле $f$ слова $s$ на слово $t$>>
и $q = $ <<удаление пятой строки файла $f$>> коммутируют, при этом, $p
= p'$, а $q = q'$. 

Рассмотрим более сложный пример. Пусть имеется файл с содержимым 
\begin{tabular}{|r|l|}
  \hline
  1: & super \\
  2: & cool \\
  3: & text \\
  \hline
\end{tabular}. 
К нему сначала применяют патч $p$, вставляющий после
третьей строки строку ``here'', а
затем, патч $q$, вставляющий после второй строки строку ``mega''. 
Этот процесс изображён на рисунке~\ref{fig:commute-before}.

\begin{figure}
  \centering
  \begin{tikzpicture}[>=stealth,thick]
    \node[context] (a) [label=right:{
      \begin{tabular}{|r|l|}
        \hline
        1: & super \\
        2: & cool \\
        3: & text \\
        \hline
      \end{tabular}
    }] {$a$}; 

    \node[context] (b) [below left=of a, label=left:{
      \begin{tabular}{|r|l|}
        \hline
        1: & super \\
        2: & cool \\
        3: & text \\
        \textbf{4}: & \textbf{here} \\
        \hline
      \end{tabular}
    }] {$b$};

    \node[context] (c) [below right=of b, label=right:{
      \begin{tabular}{|r|l|}
        \hline
        1: & super \\
        \textbf{2:} & \textbf{mega} \\
        3: & cool \\
        4: & text \\
        5: & here \\
        \hline
      \end{tabular}
    }] {$c$};
    
    \draw[->] (a) to node[patch]{$p$: +3/``here''} (b);
    \draw[->] (b) to node[patch]{$q$: +1/``mega''} (c);
  \end{tikzpicture}
  \caption{Последовательность двух патчей}
  \label{fig:commute-before}
\end{figure}

Что будет, если поменять порядок применения патчей: сначала применить
$q$ и добавить после первой строки ``mega'', а потом применить $p$ и
добавить ``here'' после третьей строки? Результат применения изображён
на рисунке~\ref{fig:commute-wrong}. Это не совпадает с тем, что
ожидалось увидеть.

\begin{figure}
  \centering
  \begin{tikzpicture}[>=stealth,thick]
    \node[context] (a) [label=left:{
      \begin{tabular}{|r|l|}
        \hline
        1: & super \\
        2: & cool \\
        3: & text \\
        \hline
      \end{tabular}
    }] {$a$}; 

    \node[context] (b) [below right=of a, label=right:{
      \begin{tabular}{|r|l|}
        \hline
        1: & super \\
        \textbf{2:} & \textbf{mega} \\
        3: & cool \\
        4: & text \\
        \hline
      \end{tabular}
    }] {$b'$};

    \node[context] (c) [below left=of b, label=left:{
      \begin{tabular}{|r|l|}
        \hline
        1: & super \\
        2: & mega \\
        3: & cool \\
        \textbf{4:} & \textbf{here} \\
        5: & text \\
        \hline
      \end{tabular}
    }] {$c'$};
    
    \draw[->] (a) to node[patch]{$q$: +1/``mega''} (b);
    \draw[->] (b) to node[patch]{$p$: +3/``here''} (c);
  \end{tikzpicture}
  \caption{Неверная попытка скоммутировать $p$ и $q$}
  \label{fig:commute-wrong}
\end{figure}

Верный ответ изображён на рисунке~\ref{fig:commute-correct}. Как видно
из этого рисунка, не всегда при коммутации $p, q \longleftrightarrow
q', p'$, $p = p'$ и $q = q'$. Однако, всё ещё ${\cal E}(p) = {\cal
  E}(p')$ и ${\cal E}(q) = {\cal E}(q')$.

\begin{figure}
  \centering
  \begin{tikzpicture}[>=stealth,thick]
    \node[context] (a) [label=above:{
      \begin{tabular}{|r|l|}
        \hline
        1: & super \\
        2: & cool \\
        3: & text \\
        \hline
      \end{tabular}
    }] {$a$}; 

    \node[context] (b) [below left=2.5cm of a, label=left:{
      \begin{tabular}{|r|l|}
        \hline
        1: & super \\
        2: & cool \\
        3: & text \\
        \textbf{4}: & \textbf{here} \\
        \hline
      \end{tabular}
    }] {$b$};

    \node[context] (d) [below right=2.5cm of a, label=right:{
      \begin{tabular}{|r|l|}
        \hline
        1: & super \\
        \textbf{2:} & \textbf{mega} \\
        3: & cool \\
        4: & text \\
        \hline
      \end{tabular}
    }] {$d$};

    \node[context] (c) [below right=2.5cm of b, label=below:{
      \begin{tabular}{|r|l|}
        \hline
        1: & super \\
        2: & mega \\
        3: & cool \\
        4: & text \\
        5: & here \\
        \hline
      \end{tabular}
    }] {$c$};
    
    \draw[->] (a) to node[patch]{$p$: +3/``here''} (b);
    \draw[->] (b) to node[patch]{$q$: +1/``mega''} (c);
    \draw[->] (a) to node[patch]{$q'$: +1/``mega''} (d);
    \draw[->] (d) to node[patch]{$p'$: +\textbf{1}/``here''} (c);
  \end{tikzpicture}
  \caption{Верная попытка скоммутировать $p$ и $q$}
  \label{fig:commute-correct}
\end{figure}

\subsection{Merge}

\subsubsection{Merge двух патчей}

Пусть в некоторый момент времени у двух различных пользователей
репозиторий находился в состоянии $a$, и они решили одновременно
внести правки в один и тот же файл. С первого взгляда может
показаться, что для этого может понадобиться вводить в Darcs
какой-нибудь новый примитив, но, на самом деле, можно обойтись уже
имеющейся коммутацией патчей. 

На рисунке~\ref{fig:merge-before} схематически изображена решаемая
задача: имея два патча с общим началом, построить получить результат
их последовательного применения. 

\begin{figure}
  \centering
  \begin{tikzpicture}[>=stealth,thick]
    \node[context] (a) {$a$}; 
    \node[context] (b) [below left=of a] {$b$};
    \node[context] (c) [below right=of a] {$c$};
    
    \draw[->] (a) to node[patch]{$p$} (b);
    \draw[->] (a) to node[patch]{$q$} (c);
  \end{tikzpicture}
  \caption{Merge двух патчей}
  \label{fig:merge-before}
\end{figure}

Посмотрим, что получится, если скоммутировать патчи $p^{-1}$ и $q$. На
рисунке~\ref{fig:merge} видно, что $p^{-1}, q \longleftrightarrow q',
(p^{-1})'$. Тогда возьмём, как показано на
рисунке~\ref{fig:merge-result}, в качестве искомых патчей, $p, q
\longleftrightarrow q', ((p^{-1})')^{-1}$.

\begin{figure}
  \centering
  \begin{subfigure}{0.3\textwidth}
    \centering
    \begin{tikzpicture}[>=stealth,thick]
      \node[context] (a) {$a$}; 
      \node[context] (b) [below left=of a] {$b$};
      \node[context] (c) [below right=of a] {$c$};
      
      \draw[<-] (a) to node[patch]{$p^{-1}$} (b);
      \draw[->] (a) to node[patch]{$q$} (c);
    \end{tikzpicture}
    \caption{До коммутации $p^{-1}$ и $q$}
  \end{subfigure}
  \begin{subfigure}{0.3\textwidth}
    \centering
    \begin{tikzpicture}[>=stealth,thick]
      \node[context] (a) {$a$}; 
      \node[context] (b) [below left=of a] {$b$};
      \node[context] (c) [below right=of a] {$c$};
      \node[context] (d) [below right=of b] {$d$};
      
      \draw[<-] (a) to node[patch]{$p^{-1}$} (b);
      \draw[->] (a) to node[patch]{$q$} (c);
      \draw[->] (b) to node[patch]{$q'$} (d);
      \draw[<-] (c) to node[patch]{$(p^{-1})'$} (d);
    \end{tikzpicture}
    \caption{После коммутации $p^{-1}$ и $q$}
  \end{subfigure}
  \begin{subfigure}{0.3\textwidth}
    \centering
    \begin{tikzpicture}[>=stealth,thick]
      \node[context] (a) {$a$}; 
      \node[context] (b) [below left=of a] {$b$};
      \node[context] (c) [below right=of a] {$c$};
      \node[context] (d) [below right=of b] {$d$};
      
      \draw[->] (a) to node[patch]{$p$} (b);
      \draw[->] (a) to node[patch]{$q$} (c);
      \draw[->] (b) to node[patch]{$q'$} (d);
      \draw[->] (c) to node[patch]{$((p^{-1})')^{-1}$} (d);
    \end{tikzpicture}
    \caption{Коммутация $p$ и $q$}
    \label{fig:merge-result}
  \end{subfigure}
  \caption{Коммутация $p^{-1}$ и $q$}
  \label{fig:merge}
\end{figure}

\subsubsection{Merge множества патчей}

Проведённые выше размышления позволяют сделать из \emph{двух}
параллельных патчей последовательные. Но что делать, если каждый из
авторов сделал более одного патча? Пусть начиная с контекста $a$ были
сделаны две не связанные между собой последовательности патчей
$p_1p_2\ldots p_n$ и $q_1q_2\ldots q_m$. Как найти такие
последовательности $p'_1p'_2\ldots p'_n$ и $q'_1q'_2\ldots q'_m$, что
${\cal E}(p_i) = {\cal E}(p'_i)$, ${\cal E}(q_j) = {\cal E}(q'_j)$, а
также, $p_1\ldots p_n, q_1\ldots q_m \longleftrightarrow q'_1\ldots
q'_m, p'_1\ldots p'_n$? 

Будем считать, что объединяемые последовательности $p_1\ldots p_n$ и
$q_1\ldots q_m$ не имеют общего префикса. Если они имеют, то можно
поступить следующим образом, проиллюстрированным на
рисунке~\ref{fig:manymerge-common-prefix}: отрезав общий префикс,
объединить хвосты, затем приклеить общий префикс в начало результата.

\begin{figure}
  \centering
  \begin{subfigure}{0.4\textwidth}
    \centering
    \begin{tikzpicture}[>=stealth,thick]
      \node[context] (a) {$a$}; 
      \node[context] (b1) [below left=of a] {$b$};
      \node[context] (b2) [below right=of a] {$b'$};
      \node[context] (c) [below=of b1] {$c$};
      \node[context] (d) [below=of b2] {$d$};
      \node[context] (e) [below right=of c] {$e$};
      
      \draw[->] (a) to node[patch]{$c_1\ldots c_k$} (b1);
      \draw[->] (a) to node[patch]{$c_1\ldots c_k$} (b2);
      \draw[->] (b1) to node[patch]{$p_{k+1}\ldots p_n$} (c);
      \draw[->] (b2) to node[patch]{$q_{k+1}\ldots q_m$} (d);
      \draw[->,dashed] (c) to node[patch]{$?$} (e);
      \draw[->,dashed] (d) to node[patch]{$?$} (e);
    \end{tikzpicture}
  \end{subfigure}
  \begin{subfigure}{0.4\textwidth}
    \centering
    \begin{tikzpicture}[>=stealth,thick]
      \node[context] (a) {$a$}; 
      \node[context] (b) [below=of a] {$b$};
      \node[context] (c) [below left=of b] {$c$};
      \node[context] (d) [below right=of b] {$d$};
      \node[context] (e) [below right=of c] {$e$};
      
      \draw[->] (a) to node[patch]{$c_1\ldots c_k$} (b);
      \draw[->] (b) to node[patch]{$p_{k+1}\ldots p_n$} (c);
      \draw[->] (b) to node[patch]{$q_{k+1}\ldots q_m$} (d);
      \draw[->,dashed] (c) to node[patch]{$?$} (e);
      \draw[->,dashed] (d) to node[patch]{$?$} (e);
    \end{tikzpicture}
  \end{subfigure}
  \caption{Избавление от общего префикса}
  \label{fig:manymerge-common-prefix}
\end{figure}


\section{Выводы по главе \ref{chapter1}}
