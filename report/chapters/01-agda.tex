\section{Agda}

Agda~\cite{agda}~--- функциональный язык программирования с поддержкой
\emph{зависимых типов}. В силу \emph{изоморфизма Карри-Ховарда} в
применении к \emph{интуиционистской теории типов Мартина-Лёфа},
используется также как \emph{верификатор доказательств}~(proof
assistant). Разберём некоторые из этих терминов.

\subsection{Интуиционистская логика}

\emph{Интуиционистская}~(или \emph{конструктивная})
логика~\cite{шень2,curryhoward}~--- раздел математики, занимающийся
изучением \emph{конструктивных} объектов, то есть, объектов, для
которых существуюет \emph{программа}, способная данный объект
построить.

Интуиционистская логика отличается от \emph{классической} отсутствием
\emph{закона исключённого третьего}~($\forall A : A \vee \neg A$) или
эквивалентного ему \emph{закона снятия двойного отрицания}~($\forall A
: \neg (\neg A) \to A$). Это отличие даёт возможность оперирования
логическими выражениями также, как программами.

\subsection{Интуиционистская теория типов Мартина-Лёфа}

\emph{Интуиционистская теория типов Мартина-Лёфа}~\cite{mltt}~---
теория типов, расширение простого типизированного
$\lambda$-исчисления. Основным отличием от него является добавление
$\Pi$- и $\Sigma$-типов.

\subsubsection{$\Pi$-типы}

$\Pi$-типы~(\emph{зависимые произведения}) $\Pi_{x:A} B(x)$ являются
обобщением типов-функций, у которых, правда, тип результата может
зависеть от \emph{значения} аргумента: тип функции $X \to Y$ можно
представить как $\Pi_{x:X} Y(x)$, где $x$~--- \emph{свежая}
переменная, не входящая свободно в $Y$. Например, тип функций
$\mathbb{Z} \to \mathbb{N}$ можно представить как $\Pi_{x:\mathbb{Z}}
\mathbb{N}$.

\subsubsection{$\Sigma$-типы}

$\Sigma$-типы~(\emph{зависимые суммы}) $\Sigma_{x:A} B(x)$ являются
обобщением пары, в которой второй аргумент может зависеть от первого.
Например, если $String(n)$~--- тип строк длины $n$, то
$\Sigma_{n:\mathbb{N}} String(n)$ будет парой из длины и строки такой
длины.

\subsubsection{Кумулятивные вселенные}

Чтобы избежать парадокса Girard-а (TODO как его на русский?)(TODO
ссылку на парадокс)(TODO понять этот парадокс) были введены
\emph{вселенные типов} $U_0, U_1, U_2, \ldots$, где $U_0$~---
множество~(TODO вроде, и правда, множество, а не proper class) всех
\emph{простых} типов. Также, $U$ последовательно вложены по отношению
типизации: $U_i : U_{i+1}$. Это даёт возможность иметь переменные типа
<<тип>>. Правда, они будут из вселенной на один выше вселенной
хранимого типа.

\subsection{Изоморфизм Карри\,–\,Ховарда}

\emph{Изоморфизм Карри\,--\,Ховарда}~\cite{curryhoward} позволяет
установить взаимно-однозначное соответствие между логическими
системами и языками программирования. Одним из простых примеров
является изоморфизм между программами на языке \emph{простого
  типизированного $\lambda$-исчисления с суммами и
  произведениями}~\cite{curryhoward} и \emph{высказываниями в
  интуиционистской логике}~\cite{curryhoward}. Его можно увидеть в
таблице на рисунке~\ref{fig:curry-howard-int}. 

Чуть более сложным примером является изоморфизм между
\emph{исчислением высказываний}~\cite{шень2,curryhoward} и
\emph{теорией типов Мартина-Лёфа}.

\begin{figure}
  \centering
  \begin{subfigure}[b]{0.55\textwidth}
    \centering
    \begin{tabular}{|l|l|}
      \hline
      \textbf{Логика} & \textbf{Программирование} \\
      \hline
      Высказывание & Тип \\
      \hline
      Доказуемое высказывание & Населённый тип \\
      \hline
      Ложное высказывание & $\bot$ \\
      \hline
      $A \to B$ & Функция из $A$ в $B$ \\
      \hline
      $A \wedge B$ & $A \times B$ \\
      \hline
      $A \vee B$ & $A + B$ \\
      \hline
    \end{tabular}
    \caption{Исчисление высказываний и просто типизированное
      $\lambda$-исчисление}
    \label{fig:curry-howard-int}
  \end{subfigure}
  \begin{subfigure}[b]{0.4\textwidth}
    \centering
    \begin{tabular}{|l|l|}
      \hline
      \textbf{Логика} & \textbf{Программирование} \\
      \hline
      $\forall x : A(x)$ & $\Pi_{(x:\tau)} A(x)$ \\
      \hline
      $\exists x : A(x)$ & $\Sigma_{(x:\tau)} A(x)$\\
      \hline
    \end{tabular}
    \caption{Исчисление предикатов и теория типов Мартина-Лёфа}
    \label{fig:curry-howard-mltt}
  \end{subfigure}
  \caption{Изоморфизм Карри\,--\,Ховарда}
\end{figure}
