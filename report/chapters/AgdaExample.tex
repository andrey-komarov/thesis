\AgdaHide{
\begin{code}%
\>\AgdaKeyword{module} \AgdaModule{AgdaExample} \AgdaKeyword{where}\<%
\\
%
\\
\>\AgdaKeyword{data} \AgdaDatatype{ℕ} \AgdaSymbol{:} \AgdaPrimitiveType{Set} \AgdaKeyword{where}\<%
\\
\>[0]\AgdaIndent{2}{}\<[2]%
\>[2]\AgdaInductiveConstructor{zero} \AgdaSymbol{:} \AgdaDatatype{ℕ}\<%
\\
\>[0]\AgdaIndent{2}{}\<[2]%
\>[2]\AgdaInductiveConstructor{suc} \AgdaSymbol{:} \AgdaDatatype{ℕ} \AgdaSymbol{→} \AgdaDatatype{ℕ}\<%
\end{code}
}

В языке Agda возможно определение рекурсивных типов без явного
использования $\mu$- или $W$-типов:

\begin{code}%
\>\AgdaKeyword{data} \AgdaDatatype{ℕList} \AgdaSymbol{:} \AgdaPrimitiveType{Set} \AgdaKeyword{where}\<%
\\
\>[0]\AgdaIndent{2}{}\<[2]%
\>[2]\AgdaInductiveConstructor{[]} \AgdaSymbol{:} \AgdaDatatype{ℕList}\<%
\\
\>[0]\AgdaIndent{2}{}\<[2]%
\>[2]\AgdaInductiveConstructor{\_∷\_} \AgdaSymbol{:} \AgdaSymbol{(}\AgdaBound{head} \AgdaSymbol{:} \AgdaDatatype{ℕ}\AgdaSymbol{)} \AgdaSymbol{→} \AgdaSymbol{(}\AgdaBound{tail} \AgdaSymbol{:} \AgdaDatatype{ℕList}\AgdaSymbol{)} \AgdaSymbol{→} \AgdaDatatype{ℕList}\<%
\end{code}

Здесь написано, что список натуральных чисел~--- это или пустой
список~(\AgdaInductiveConstructor{[]}),
либо~(\AgdaInductiveConstructor{\_∷\_}) пара из головы
списка~(натурального числа) и хвоста-списка.

Это определение можно обобщить до списка над произвольными типами:

\begin{code}%
\>\AgdaKeyword{data} \AgdaDatatype{List} \AgdaSymbol{(}\AgdaBound{A} \AgdaSymbol{:} \AgdaPrimitiveType{Set}\AgdaSymbol{)} \AgdaSymbol{:} \AgdaPrimitiveType{Set} \AgdaKeyword{where}\<%
\\
\>[0]\AgdaIndent{2}{}\<[2]%
\>[2]\AgdaInductiveConstructor{[]} \AgdaSymbol{:} \AgdaDatatype{List} \AgdaBound{A}\<%
\\
\>[0]\AgdaIndent{2}{}\<[2]%
\>[2]\AgdaInductiveConstructor{\_∷\_} \AgdaSymbol{:} \AgdaSymbol{(}\AgdaBound{head} \AgdaSymbol{:} \AgdaDatatype{ℕ}\AgdaSymbol{)} \AgdaSymbol{→} \AgdaSymbol{(}\AgdaBound{tail} \AgdaSymbol{:} \AgdaDatatype{List} \AgdaBound{A}\AgdaSymbol{)} \AgdaSymbol{→} \AgdaDatatype{List} \AgdaBound{A}\<%
\end{code}
