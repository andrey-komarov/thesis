\section{Camp}

Camp~--- экспериментальная система контроля версий. Является
теоретическим дальнейшим развитием Darcs и содержит много общей с
Darcs теории. Дадим её обзорное описание.

Также, как в Darcs, в Camp хранятся \emph{патчи}. Изначально
репозиторий пустой. Чтобы добавить туда что-нибудь, нужно создать
патч, превращающий пустой репозиторий во что-то нужное.

\subsection{Неименованные патчи}

\emph{Неименованные патчи} являются основным <<материалом>> для
создания репозитория. Являют собой, собственно, изменения, которые
вносятся в рабочую копию. 

Также, как в Darcs, каждый неименованный патч $\underline{p}$ имеет
патч с обратным эффектом $\underline{p}^{-1}$. Неименованные патчи
можно применять последовательно: $\underline{p}\underline{q}$ значит,
что сначала применяется патч $\underline{p}$, а затем патч
$\underline{q}$.

Над парами патчей, также, вводится частичная функция коммутации: пары
$\langle\underline{p}, \underline{q}\rangle$ и
$\langle\underline{q'}, \underline{p'}\rangle$ \emph{коммутируют}
(записывается как $\langle\underline{p}, \underline{q}\rangle
\longleftrightarrow \langle\underline{q'}, \underline{p'}\rangle$),
если применение $\underline{p}$, а затем $\underline{q}$ эквивалентно
применению сначала $\underline{q'}$, а затем $\underline{p'}$, а
также, $\underline{p}$ имеет то же действие, что и $\underline{p'}$, а
$\underline{q}$~--- то же, что $\underline{q'}$.

Коммутация должна обладать следующими свойствами:

\begin{itemize}
\item Определена для каждой пары $\langle \underline{p}, \underline{q}
  \rangle$ единственным образом: если $\langle \underline{p},
  \underline{q} \rangle \longleftrightarrow \langle \underline{q'},
  \underline{p'}\rangle$ и $\langle \underline{p}, \underline{q}
  \rangle \longleftrightarrow \langle \underline{q''},
  \underline{p''}\rangle$, то $\underline{p'} = \underline{p''}$ и
  $\underline{q'} = \underline{q''}$.
\item Дважды применённая, должна выдавать начальные данные: если
  $\langle \underline{p}, \underline{q} \rangle \longleftrightarrow
  \langle \underline{q'}, \underline{p'}\rangle$, то $\langle
  \underline{q'}, \underline{p'} \rangle \longleftrightarrow \langle
  \underline{p}, \underline{q}\rangle$.
\end{itemize}

\subsection{Именованные патчи}

Введём \emph{имена} патчей~--- нечто, однозначно характеризующее патч.
Имена могут быть \emph{положительными} или \emph{отрицательными}. При
создании патч получает положительное имя. После его обращения имя
меняет знак на противоположный. 

Назовём \emph{патчем} пару из имени и неименованного патча и логичным
образом расширим введённые для неименованных патчей определения для
обращения, композиции и коммутации с тем ограничением, что патч не
коммутирует с патчем с таким же именем или таким же именем с
противоположным знаком. При коммутации имя будет сохраняться: если
$\langle p, q \rangle \longleftrightarrow \langle q', p' \rangle$, то
имя $n(p) = n(p')$, $n(q) = n(q')$.

\subsection{Последовательности патчей}

Обобщим введённые понятия с единичного патча на последовательности
патчей. Единственную сложность представляет обобщения коммутации.
Введём коммутацию последовательносей следующим образом.

\begin{itemize}
\item $\langle \overline{p}, \varepsilon \rangle \longleftrightarrow
  \langle \overline{p}, \varepsilon \rangle$
\item $\langle \varepsilon, \overline{p} \rangle \longleftrightarrow
  \langle \varepsilon, \overline{p} \rangle$
\item $\langle \overline{p}q, r\overline{s} \longleftrightarrow
  \langle r''\overline{s''}, \overline{p''}q''
  \rangle$, если:
  \begin{itemize}
  \item $\langle q, r \rangle \longleftrightarrow \langle r', q' \rangle$
  \item $\langle \overline{p}, r' \rangle \longleftrightarrow \langle
    r'', \overline{p'} \rangle$
  \item $\langle q', \overline{s} \rangle \longleftrightarrow \langle
    \overline{s'}, q'' \rangle$
  \item $\langle \overline{p'}, \overline{s'} \rangle
    \longleftrightarrow \langle \overline{s''}, \overline{p''} \rangle$
  \end{itemize}
\end{itemize}

\subsection{Merge}

Merge в Camp происходит аналогично merge в Darcs: находится
наидлиннейший общий префикс двух сливаемых историй, записывается в
результат, процесс продолжается для двух историй без общего префикса.
Для них запускается процедура, показанная на рисунках~\ref{fig:merge}.

\subsection{Итд}

Catch-и, конфликторы, контекстнутые патчи и прочие ужасные вещи,
в которых я пока примерно ничего не понял.