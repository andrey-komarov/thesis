\startprefacepage

Разнообразные системы контроля версий повседневно используются почти
всеми разработчиками программного обеспечения для совместной работы
% и не только совместной
над проектом. В \emph{распределённых} системах контроля версий могут
% и не только в распределённых, в svn тоже есть бранчи
одновременно разрабатываться множество различных возможностей
программы.
% Имхо, для данной работы предложение ниже не первостепенно.
% Тут следует написать, что системы контроля версий должны
% предоставлять не только интерфейс, но и какие-то гарантии,
% а потому про них, по хорошему, надо что-то доказывать.
% Т.е. я бы переформулировал слегка следующий параграф и его сделал
% выше the предложения.
Поскольку разработка различных возможностей ведётся
независимо друг от друга, возникает вопрос о том, как соединить
воедино внесённые всеми разработчиками изменения.

Подавляющее большинство систем контроля версий реализованы без
какой-либо низлежащей теории. SVN~\cite{svnbook}, git~\cite{git} и
прочие повсеместно используемые системы работают на основе алгоритмов,
для которых нет формального доказательства корректности их работы.
% не забыть и про ad-hoc структуры данных

% Этот параграф надо менее подробно и более обобщённо. Про Coq тут
% точно не надо.
Для системы контроля версий Darcs~\cite{darcs} была разработана
\emph{теория патчей Darcs}, однако, верифицированной реализации нет.
Проект Camp~\cite{camp} является логическим продолжением Darcs. В нём
ставилась цель написать и верифицировать на языке Coq~\cite{coq}
систему контроля версий с лежащей в основе более развитой версией
теории патчей Darcs.

Целью настоящей работы является разработка нескольких <<каркасов>> для
класса систем контроля версий на основе одной общей идеи
<<логистического>> характера с их формализацией, формулировкой и
доказательством некоторых их свойст на языке доказательного
программирования Agda~\cite{agda}.
