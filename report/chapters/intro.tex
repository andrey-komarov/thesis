\startprefacepage

Разнообразные системы контроля версий повседневно используются почти
всеми разработчиками программного обеспечения для работы над проектом.
В них могут одновременно разрабатываться множество различных
возможностей программы. Являясь одним из основных инструментов в
разработке программного обеспечения, система контроля версий должна
предоставлять гарантии своей корректной работы. Способом достижения
этих гарантий является, например, реализация её на каком-либо языке
доказательного программирования. Однако, подавляющее большинство
систем контроля версий реализованы без какой-либо низлежащей теории.
SVN~\cite{svnbook}, git~\cite{git} и прочие повсеместно используемые
системы работают на основе алгоритмов и структур данных, для которых
нет формального доказательства корректности их работы.

% не забыть и про ad-hoc структуры данных
%% например, так?

% Этот параграф надо менее подробно и более обобщённо. Про Coq тут
% точно не надо.

Ранее предпринималось несколько попыток сделать систему контроля
версий, обладающую такими свойствами. Например, для Darcs~\cite{darcs}
была разработана соответствующая \emph{теория патчей}, однако,
доказанной реализации у неё нет. Также существует проект
Camp~\cite{camp}, в котором поставлена цель получить полностью
доказанную Darcs-подобную систему контроля версий, однако этот проект
пока далёк от завершения.

Целью настоящей работы является разработка нескольких <<каркасов>> для
класса систем контроля версий на основе одной общей идеи
<<логицистического>> характера с их формализацией, формулировкой и
доказательством некоторых их свойст на языке доказательного
программирования Agda~\cite{agda}.
