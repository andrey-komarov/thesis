\startprefacepage

Разнообразные системы контроля версий повседневно используются почти
всеми разработчиками программного обеспечения для совместной работы
над проектом. В \emph{распределённых} системах контроля версий могут
одновременно разрабатываться множество различных возможностей
программы. Поскольку разработка различных возможностей ведётся
независимо друг от друга, возникает вопрос о том, как соединить
воедино внесённые всеми разработчиками изменения.

Подавляющее большинство систем контроля версий реализованы без
какой-либо низлежащей теории. SVN~\cite{svnbook}, git~\cite{git} и
прочие повсеместно используемые системы работают на основе алгоритмов,
для которых нет формального доказательства корректности их работы.

Для системы контроля версий Darcs~\cite{darcs} была разработана
\emph{теория патчей Darcs}, однако, верифицированной реализации нет.
Проект Camp~\cite{camp} является логическим продолжением Darcs. В нём
ставилась цель написать и верифицировать на языке Coq~\cite{coq}
систему контроля версий с лежащей в основе более развитой версией
теории патчей Darcs.

В этой работе на языке Agda~\cite{agda} будут построены примеры
<<каркасов>> систем контроля версий: ревизии, патчи, операции над
патчами, а также сформулированы и доказаны некоторые их свойства.


