\section{Git}

Git~--- одна из наиболее распространённых на данный момент
распределённых систем контроля версий. Каждый разработчик хранит у
себя все изменения, вносимые в репозиторий. Репозитории разных
разработчиков могут свободно обмениваться между собой патчами
независимо от какого-либо единственного выделенного сервера,
поддерживая проект в актуальном состоянии.

\subsection{Обзор}

Прежде чем начать пользоваться Git-ом, пользователю необходимо
заполучить себе репозиторий. Это можно сделать двумя способами:

\begin{enumerate}
\item команда \verb!git init! создаёт пустой репозиторий в текущей
  директории;
\item команда \verb!git clone! создаёт репозиторий-копию указанного
  удалённого репозитория.
\end{enumerate}

После создания репозитория с помощью \verb!git init!, у пользователя
появляется пустой репозиторий с единственной \emph{веткой} master.
Если же создание производилось командой \verb!git clone!, то
появляются копии всех веток клонируемого репозитория, а также,
\emph{удалённый репозиторий} origin.

