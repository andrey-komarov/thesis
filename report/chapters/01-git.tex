\section{Git}

Git~\cite{progit,git}~--- одна из наиболее распространённых на данный
момент распределённых систем контроля версий. Каждый разработчик
хранит у себя все изменения, вносимые в репозиторий. Репозитории
разных разработчиков могут свободно обмениваться между собой патчами
независимо от какого-либо единственного выделенного сервера,
поддерживая проект в актуальном состоянии.

\subsection{Внутренние структуры Git}
\label{sec:git-internals}

Внутри репозитория Git хранит \emph{объекты} и \emph{ссылки}.
\emph{Объектом} называются произвольные данные в паре с их 160-битным
SHA-1-хешом. Git хранит отображение SHA-1-хеша в объект, который имеет
заданный хеш. \emph{Ссылкой} на объект называют значение его
SHA-1-хеша. SHA-1-хеша какого-то из имеющихся объектов. Поверх этой
простой структуры сделано всё остальное~--- файлы, ветки, патчи,
коммиты, итд.

У каждого хранимого Git-ом объекта есть свой тип, который определяет,
для какой цели предназначен этот объект. Git использует следующие типы
для объектов: blob, tree, commit, tag.

% TODO картинку для каждого типа?
% не стоит переусердствывовать с картинками. делать их наверняка геморойно
% а в ПЗаписке почти никто их не увиидт.

\subsubsection{blob}

\emph{Blob}~--- самый простой объект. Он просто хранит произвольные
данные и может использоваться, например, для хранения,
непосредственно, файлов, находящихся под контролем версий.

\subsubsection{tree}

Объекты типа \emph{tree} используются для хранения содержимого
директории. В объекте этого типа хранится список ссылок на другие
объекты. Объекты, на которые ссылаются, могут иметь тип tree или blob.
С помощью ссылок на объекты типа tree достигается возможность хранения
директорий в древообразном формате, а не только в корне.

\subsubsection{commit}

В объектах типа commit хранится следующее:

\begin{itemize}
\item информация об авторе коммита: имя, почтовый адрес и время
  совершения;
\item ссылка на объект типа tree, соответствующий состоянию рабочей
  директории на момент совершения коммита;
\item ссылки на \emph{родителей} этого коммита. В случае, если это был
  \emph{merge-коммит}, родителей может быть несколько.
\end{itemize}

\subsubsection{tag}

Tag хранит внутри ссылку на какой-то объект, а также, описание того,
кто создал этот тег. Это может использоваться, например, для того,
чтобы можно было быстро восстановить рабочую директорию на какой-то
опеределённый момент времени, не зная её хеш, а зная лишь имя тега.

\subsubsection{Оптимизация хранения объектов}

Из описания выше можно понять, что для поддержания корректной
репозитория требуется иметь в хранилище объектов все версии всех
файлов. Однако, это, очевидно, невыгодно и будет занимать слишком
много места на диске. Для оптимизации дискового пространства в Git
применяется две оптимизации. 

Во-первых, время от времени, происходит сжатие архивация и сжатие
объектов в один файл. Это позволяет более эффективно использовать
дисковое пространство так как, во-первых, сжатые данные обычно
занимают меньше места, а, во-вторых, нет накладных расходов файловой
системы на хранение большого числа маленьких файлов.

Вторая, более существенная оптимизация заключается в том, что все
версии одного и того же файла не хранятся отдельно. Вместо этого,
хранится только \emph{последняя} версия этого файла, а также, для
каждой из предыдущих хранится разница (diff, delta, дельта) со
следующей за ней. Было решено хранить \emph{последнюю} версию файла
вместо \emph{первой} потому, что обращения к последней версии файла
обычно происходит чаще, чем к первой, а в случае хранения первой, были
бы большие накладные расходы на расчёт последней: для этого бы
потребовалось применить последовательно все разницы для этого файла.
% чё-то мне казалось, что гит таки хранит все блобы просто пакуя всё xdelta
% а написанное тут скорее похоже на Mercurial
%% как я понял, с точки зрения пользователя, действительно, хранятся
%% все версии. а в совсем внутреннем представлении хранится только
%% последняя, а остальные каким-то магическим образом тоже, якобы,
%% хранятся, но сжато.
%% TODO разобраться

\subsection{Внешние структуры Git (TODO: Очень плохое название)}

В разделе~\ref{sec:git-internals} было дано описание внутренних
структур системы контроля версий Git. Однако, пользователю практически
никогда не нужно ни знать ни их устройство, ни то, как ими
манипулировать. Пользователю предоставляется более высокоуровневый
интерфейс управления. 

% большую часть этого обобщить и в общие определения
%Большую часть времени, пользователь \emph{репозитория} будет работать
%с \emph{коммитами} и \emph{ветками}.
%
%\begin{definition}[Коммит]
%  Назовём \emph{коммитом} запомненное в репозитории состояние рабочей
%  копии (файлов и их содержимого) в какой-либо момент времени.
%\end{definition}

\begin{definition}[Git-репозиторий]
  \emph{Git-репозиторий}~--- ориентированный ациклический граф
  коммитов, в котором, для удобства, в некоторых вершинах стоят метки,
  называемые ветками.
  
  Вершины этого графа~--- коммиты, состояния, в которых когда-либо
  находился этот репозиторий. Рёбра~--- патчи, то, как из
  состояния-начала ребра получилось состояние-конец.
\end{definition}

%\begin{definition}[Ветка]
%  \emph{Ветка}~--- именованная метка в одной из вершин графа коммитов
%  в репозитории.
%\end{definition}

Также, в репозитории есть отдельная сущность~--- HEAD. 

\begin{definition}[HEAD]
  \emph{HEAD}~--- указатель на последний сделанный коммит.
\end{definition}

HEAD используется для того, чтобы Git знал, какие рёбра добавлять к
новой созданной вершине при коммите, а также, чтобы знать, с рабочей
директорией, соответствующей какому коммиту, следует сравнивать
текущую изменённую пользователем рабочую директорию.

Рассмотрим пример работы с Git-репозиторием, изображённый на
рисунке~\ref{fig:git-workflow}, поясняющий эту концепцию. Рассмотрим
происходящее на ней по шагам.

\begin{enumerate}
\item Создание нового пустого репозитория~(рис.~\ref{fig:git-init}).
  После этого есть один коммит $A$ и одна ветка master, которая
  указывает на этот коммит. HEAD указывает на master.
\item Добавление файла ``a.txt'' под контроль
  версий~(рис.~\ref{fig:git-add-commit}). После этого добавляется
  новый коммит $B$, в котором есть добавленный файл. Указатель ветки
  master смещается на $B$, HEAD указывает на master.
\item Создание новой ветки feature и переключение на
  неё~(рис.~\ref{fig:git-checkout-feature}). Создаётся ветка,
  являющаяся копией текущей. HEAD указывал на $B$, поэтому,
  вновьсозданная ветка будет указывать туда же. HEAD указывает на
  feature.
\item Добавление нового файла ``b.txt'' под контроль
  версий~(рис.~\ref{fig:git-add-commit-b}). Создаётся новый коммит
  $C$, содержащий оба файла, а также, ветка feature перемещается в
  $C$. HEAD указывает на feature.
\item Переключение на ветку master и добавление файла ``c.txt'' под
  контроль версий~(рис.~\ref{fig:git-checkout-master}). Создаётся
  новый коммит $D$ на основе $B$, содержащий ``c.txt''. Ветка master
  перемещается в $D$. HEAD указывает на master.
\item Объединение веток feature и master~(рис.~\ref{fig:git-merge}).
  Создаётся новый коммит $E$ на основе $C$ и $D$, содержащий изменения
  из обоих. Ветка master перемещается в $E$. HEAD указывает на master.
\end{enumerate}

\begin{figure}
  \centering
  \begin{subfigure}{0.3\textwidth}
    \centering
    \begin{tikzpicture}[>=stealth,thick]
      \node[gitcommit] (a) {$A$\nodepart{second}};
      \node[gitbranch] (master) [above=of a] {master};

      \draw[gitbranchconnect] (master) to (a);
    \end{tikzpicture}
    \caption{init}
    \label{fig:git-init}
  \end{subfigure}
  \begin{subfigure}{0.3\textwidth}
    \centering
    \begin{tikzpicture}[>=stealth,thick]
      \node[gitcommit] (a) {$A$\nodepart{second}};
      \node[gitcommit] (b) [right=of a] {$B$\nodepart{second}a.txt};
      \node[gitbranch] (master) [above=of b] {master};

      \draw[<-] (a) to (b);
      \draw[gitbranchconnect] (master) to (b);
    \end{tikzpicture}
    \caption{add a.txt; commit a.txt}
    \label{fig:git-add-commit}
  \end{subfigure}
  \begin{subfigure}{0.3\textwidth}
    \centering
    \begin{tikzpicture}[>=stealth,thick]
      \node[gitcommit] (a) {$A$\nodepart{second}};
      \node[gitcommit] (b) [right=of a] {$B$\nodepart{second}a.txt};
      \node[gitbranch] (master) [above=of b] {master};
      \node[gitbranch] (feature) [below=of b] {feature};

      \draw[<-] (a) to (b);
      \draw[gitbranchconnect] (master) to (b);
      \draw[gitbranchconnect] (feature) to (b);
    \end{tikzpicture}
    \caption{checkout -b feature}
    \label{fig:git-checkout-feature}
  \end{subfigure}
  \begin{subfigure}{0.45\textwidth}
    \centering
    \begin{tikzpicture}[>=stealth,thick]
      \node[gitcommit] (a) {$A$\nodepart{second}};
      \node[gitcommit] (b) [right=of a] {$B$\nodepart{second}a.txt};
      \node[gitcommit] (c) [below right=of b]
      {$C$\nodepart{second}a.txt, b.txt};
      \node[gitbranch] (master) [above=of b] {master};
      \node[gitbranch] (feature) [below=of c] {feature};

      \draw[<-] (a) to (b);
      \draw[<-] (b) to (c);
      \draw[gitbranchconnect] (master) to (b);
      \draw[gitbranchconnect] (feature) to (c);
    \end{tikzpicture}
    \caption{add b.txt; commit b.txt}
    \label{fig:git-add-commit-b}
  \end{subfigure}
  \begin{subfigure}{0.45\textwidth}
    \centering
    \begin{tikzpicture}[>=stealth,thick]
      \node[gitcommit] (a) {$A$\nodepart{second}};
      \node[gitcommit] (b) [right=of a] {$B$\nodepart{second}a.txt};
      \node[gitcommit] (c) [below right=of b]
      {$C$\nodepart{second}a.txt, b.txt};
      \node[gitcommit] (d) [right=of b]
      {$D$\nodepart{second}a.txt, c.txt};
      \node[gitbranch] (master) [above=of d] {master};
      \node[gitbranch] (feature) [below=of c] {feature};

      \draw[<-] (a) to (b);
      \draw[<-] (b) to (c);
      \draw[<-] (b) to (d);
      \draw[gitbranchconnect] (master) to (d);
      \draw[gitbranchconnect] (feature) to (c);
    \end{tikzpicture}
    \caption{checkout master; add c.txt; commit c.txt}
    \label{fig:git-checkout-master}
  \end{subfigure}
  \begin{subfigure}{\textwidth}
    \centering
    \begin{tikzpicture}[>=stealth,thick]
      \node[gitcommit] (a) {$A$\nodepart{second}};
      \node[gitcommit] (b) [right=of a] {$B$\nodepart{second}a.txt};
      \node[gitcommit] (c) [below right=of b]
      {$C$\nodepart{second}a.txt, b.txt};
      \node[gitcommit] (d) [right=of b]
      {$D$\nodepart{second}a.txt, c.txt};
      \node[gitcommit] (e) [right=of d]
      {$E$\nodepart{second}a.txt, b.txt, c.txt};
      \node[gitbranch] (master) [above=of e] {master};
      \node[gitbranch] (feature) [below=of c] {feature};

      \draw[<-] (a) to (b);
      \draw[<-] (b) to (c);
      \draw[<-] (b) to (d);
      \draw[<-] (c) to (e);
      \draw[<-] (d) to (e);
      \draw[gitbranchconnect] (master) to (e);
      \draw[gitbranchconnect] (feature) to (c);
    \end{tikzpicture}
    \caption{merge feature}
    \label{fig:git-merge}
  \end{subfigure}
  \caption{Пример работы в Git-репозитории}
  \label{fig:git-workflow}
\end{figure}

Как можно видеть из этого описания, большинство операций над
репозиторием можно выразить в виде несложных модификаций низлежащего
графа. Рассмотрим некоторые из этих действий более подробно.

\subsubsection{Создание репозитория}

Существуют два способа создания репозитория.

\begin{enumerate}
\item Создание пустого репозитория. Выполняется командой git init.
  Действие этой команды было продемонстрировано на
  рисунке~\ref{fig:git-init}.
\item Клонирование имеющегося удалённого репозитория. Выполняется
  командой git clone. Получившийся репозиторий содержит почти всё, что
  было в клонируемом. В частности, копируются все прошлые версии
  файлов.
\end{enumerate}

\subsubsection{Виды файлов}

Все файлы в рабочей директории делятся на два типа:
\emph{подконтрольные} (tracked) и \emph{неподконтрольные} (untracked).
Подконтрольные, как можно понять из названия, находятся под контролем
версий и Git о них что-то знает. Неподконтрольные же никогда не
добавлялись и под контролем версий не находятся. Каждый файл относится
строго к одной из этих двух категорий.

Подконтрольные файлы, в свою очередь, делятся ещё на три типа.

\begin{itemize}
\item \emph{Неизменённые} (unchanged)~--- актуальная версия этого
  файла хранится в текущем состоянии репозитория.
\item \emph{Подготовленные к коммиту} (staged)~--- в файле были
  сделаны изменения и эти изменения будут сохранены в ближайшем коммите.
\item \emph{Не подготовленные к коммиту} (unstaged)~--- в файле были
  сделаны изменения, но он не будет записан в ближайший коммит.
\end{itemize}

% это и много ниже — полезные знания и я бы их совсем не выбрасывал,
% ибо пригодиться, но к тому что ты делаешь и бакалаврской оно имеет
% мало непосредственного отношения, потому из ПЗ я бы почти всё это убрал.
%
% надо обращать внимание на структуры данных, но, наверное, не стоит
% уклоняться в идиосинкразии VCSок, если только тебе они зачем-то дальше
% не будут нужны
%% Может показаться, что и эти три типа взаимоисключающие~--- файл может
%% быть либо неизменённым, либо подготовленным к коммиту, либо не
%% подготовленным к коммиту. Однако, это не так, и файл может быть
%% одновременно и подготовленным, и неподготовленным к коммиту. Как этого
%% добиться? Пусть в какой-то файл внесли какие-то изменения и добавили
%% его к коммиту. Потом ещё раз отредактировали, но в этот раз к коммиту
%% не добавляли. Тогда, если в этот момент сделать коммит, то будет
%% записана только первая серия изменений. Файл же, соответственно, будет
%% одновременно частично подготовленным и частично не подготовленным к
%% коммиту.
%% 
%% Операции по добавлению файлов в репозиторий и по переводу добавленного
%% файла из состояния неподготовленного в состояние подготоленного
%% выполняются при помощи команды git add.

%% \subsubsection{Git commit}
%% 
%% Как уже было продемонстрировано, команда commit сохраняет все
%% подготовленные к коммиту файлы в новом коммите, делает предком
%% нового коммита тот, на который ранее указывал HEAD и перемещает HEAD и
%% указатель текущей ветки на созданный коммит. 
%% 
%% \subsubsection{Git branch, git checkout}
%% 
%% Эти команды нужны для управления ветками. Команда git branch позволяет
%% создать ветку-копию текущей, либо удалить любую ветку. Git checkout
%% отвечает за переключение между ветками. Её действие можно увидеть на
%% рисунке~\ref{fig:git-checkout-feature}.

% при всём вышесказанном, rebase и merge нужны, но скорее
% в общих определениях
%
% вообще в полной гипотетической разработке идеи этой работы с
% репозиториями второго и далее уровней этому нужно было бы уделить
% много внимания, но поскольку у тебя тут репозиториев высших порядков
% нет, то и углубляться в мержи и ребезы тут не надо
%
% а вот в заключении надо написать, что почти во всех VCS мержи это
% специальный тип коммита, а rebase не имеет нормального представления
% а гипотетическое развитие этой теории решает эту проблему
%
% написанное тут я бы сохранил на будущее, если ты будешь ещё этим
% заниматься
\subsubsection{Git merge}

Команда для объединения двух или более веток. Пример её работы
изображён на рисунке~\ref{fig:git-merge}. После объединения создаётся
новый коммит, предками которого являются коммиты, на которые указывали
все объединяемые ветки.

\subsubsection{Git rebase}

Действие команды rebase отдалённо похоже на действие команды merge.
Вспомним пример, изображённый на рисунке~\ref{fig:git-rebase-before}.
После применения команды merge получалась история, изображённая на
рисунке~\ref{fig:git-merge}. Однако, по какой-либо причине хочется
иметь линейную историю, по которой можно сказать, что коммит был
раньше или позже другого коммита. Rebase служит именно этой цели. 

Пусть на рисунке~\ref{fig:git-rebase-before} в качестве текущей
выбрана ветка feature. Тогда после выполнения команды ``git rebase
master'' история будет выглядеть как на
рисунке~\ref{fig:git-rebase-after}. Особо стоит обратить внимание, что
$C \ne C'$. Как видно, был достигнут тот же конечный результат, что и
при выполнении merge. Отличие в том, что для выполнения merge был
создан отдельный коммит, а при выполнении rebase \emph{необратимо}
модифицировались имеющиеся. В этом и есть отличие merge от rebase~---
при rebase необратимо меняется репозиторий, стираются прошлые коммиты,
вместо них добавляются новые, похожие. При merge же никаких
деструктивных изменений репозитория не происходит: добавляется один
новый коммит и ничего не удаляется.

\subsubsection{Проблемы с rebase}

Одной из проблем, свойственной как git-у, так и другим системам
контроля версий, является то, что команды, аналогичные команде
\texttt{rebase} вносят необратимые изменения в репозиторий. В
противовес этому, в целях систем контроля версий присутствует задача
сохранения \emph{всех} прошлых состояний, ревизий. Команды,
аналогичные \texttt{rebase} являются преградой на пути к этой цели. 

Одной из целей данной работы является построение системы контроля
версий, лишённой этого недостатка, действительно, сохраняющей всю
историю и не имеющую операций, способных нанести непоправимый вред
репозиторию.

\begin{figure}
  \centering
  \begin{subfigure}{0.45\textwidth}
    \centering
    \begin{tikzpicture}[>=stealth,thick]
      \node[gitcommit] (a) {$A$\nodepart{second}};
      \node[gitcommit] (b) [right=of a] {$B$\nodepart{second}a.txt};
      \node[gitcommit] (c) [below right=of b]
      {$C$\nodepart{second}a.txt, b.txt};
      \node[gitcommit] (d) [right=of b]
      {$D$\nodepart{second}a.txt, c.txt};
      \node[gitbranch] (master) [above=of d] {master};
      \node[gitbranch] (feature) [below=of c] {feature};

      \draw[<-] (a) to (b);
      \draw[<-] (b) to (c);
      \draw[<-] (b) to (d);
      \draw[gitbranchconnect] (master) to (d);
      \draw[gitbranchconnect] (feature) to (c);
    \end{tikzpicture}
    \caption{До rebase}
    \label{fig:git-rebase-before}
  \end{subfigure}
  \begin{subfigure}{0.6\textwidth}
    \centering
    \begin{tikzpicture}[>=stealth,thick]
      \node[gitcommit] (a) {$A$\nodepart{second}};
      \node[gitcommit] (b) [right=of a] {$B$\nodepart{second}a.txt};
      \node[gitcommit] (d) [right=of b]
      {$D$\nodepart{second}a.txt, c.txt};
      \node[gitcommit] (c') [right=of d]
      {$C'$\nodepart{second}a.txt, b.txt, c.txt};
      \node[gitbranch] (master) [above=of d] {master};
      \node[gitbranch] (feature) [below=of c'] {feature};

      \draw[<-] (a) to (b);
      \draw[<-] (b) to (c');
      \draw[<-] (b) to (d);
      \draw[gitbranchconnect] (master) to (d);
      \draw[gitbranchconnect] (feature) to (c');
    \end{tikzpicture}
    \caption{После rebase}
    \label{fig:git-rebase-after}
  \end{subfigure}
  \caption{Демонстрация работы git rebase}
  \label{fig:git-rebase}
\end{figure}
