\startconclusionpage

В данной работе была сформулирована задача создания доказанной системы
контроля версий. К сожалению, эта задача слишком обширна и полностью
решить её в рамках этой работы не представляется возможным. Были
сделаны первые шаги к её решению. На языке Agda было написано две
основы для системы контроля версий. В одной из них ревизии являются
векторами какой-то длины над произвольным типом. Во второй~---
двоичными деревьями. Для этих двух видов ревизий были описаны патчи,
операции над патчами и свойства этих операций. 

Дальнейшим развитием этой работы видится следующее. Замечено, что два
рассмотренных случая получились очень похожими. Хотелось бы выделить
какой-то общий интерфейс, набор типов и операций, имея который, можно
сказать, что нечто является системой контроля версий. 

Другим вариантом дальнейшего развития видится реализация интерфейса
\emph{настоящего} приложения, с которым может работать пользователь.
Небольшой помехой для этого является то, что в текущей реализации для
выполнения операций над патчами требуется иметь доказательство того,
что эту операцию можно сделать. Эту проблему можно решить, если
написать функцию, по двум патчам возвращающую либо доказательство
того, что их можно объединить, либо доказательство того, что нельзя.
Для первого из двух реализованных вариантов такая функция уже
реализована. % однако, в продакшен^W отчёт за ненадобностью не попала

Также, возможны дальнейшие исследования того, какие ещё структуры
могут быть использованы для ревизий. Например, исследование
возможности формализации UNIX-патчей, или, что более, вероятно, патчей
Darcs.