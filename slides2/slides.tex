\documentclass[14pt]{beamer}
\usepackage[utf8]{inputenc}
\usepackage{amsmath}
\usepackage{amssymb}
\usepackage[english,russian]{babel}
\hypersetup{colorlinks=true, urlcolor=blue}
\usepackage{xcolor}
\usetheme{Warsaw}

\setbeamertemplate{caption}[numbered]
\setbeamerfont{page number in head/foot}{size=\large}
\setbeamertemplate{footline}[frame number]

\colorlet{patchform}{blue!40}
\usepackage{tikz}
\usetikzlibrary{shapes.multipart}
\usetikzlibrary{chains}
\usetikzlibrary{fit}
\usetikzlibrary{backgrounds}
\usetikzlibrary{arrows}
\tikzset{vecpatch/.style={
    draw=black, 
    minimum width=0.5cm, 
    minimum height=1cm,
    inner sep=-1cm,
    outer sep=-1cm,
    anchor=center}
}
\tikzset{vecpatch-tiny/.style={vecpatch, minimum height=0.5cm}}
\tikzset{vecpatch-f/.style={vecpatch}}
\tikzset{vecpatch-t/.style={vecpatch,fill=yellow}}
\tikzset{vecpatch-d/.style={vecpatch-tiny,anchor=base}}
\tikzset{vecpatch-op/.style={font=\large}}
\tikzset{vecpatch-form/.style={vecpatch-tiny,fill=patchform}}

\tikzset{treev/.style={draw=black,circle,outer sep=0,minimum
    height=8mm,very thick}}
\tikzset{treearr/.style={draw, -latex, very thick}}

\newcommand{\vecfe}[0]{\node[vecpatch-tiny] {};}
\newcommand{\vecff}[0]{\node[vecpatch-form] {};}
\newcommand{\vecf}[0]{\node[vecpatch-f] {};}
\newcommand{\vecd}[1]
{\node[vecpatch-d, draw=none] {$#1$}; 
 \node[vecpatch-d, text opacity=0] {b};}
\newcommand{\vect}[2]{\node[vecpatch-t]
{\begin{tabular}{l}$#1$\\$#2$\\\end{tabular}};}

\title{Аксиоматизация системы контроля версий в формализме зависимых типов}
\author{Андрей Комаров, гр. 4538}
\institute{НИУ ИТМО}
\date{\today}

\begin{document}

\begin{frame}
  \maketitle

  \begin{flushright}
    Научный руководитель: Малаховски Я. М.
  \end{flushright}
\end{frame}

\begin{frame}{Цель}
  \begin{block}{Глобальная}
  \begin{itemize}
  \item Реализовать формальный <<каркас>> системы контроля версий
  \item на <<логицистическом>> аппарате.
  \item Доказать его свойства.
  \end{itemize}
  \end{block}

  \begin{block}{Этой работы}
  \begin{itemize}
  \item Формализовать несколько частных случаев этого аппарата
    \begin{itemize}
    \item на языке Agda.
    \end{itemize}
  \item Доказать какие-то их свойства.
  \end{itemize}
  \end{block}
\end{frame}

\begin{frame}[fragile]{Что хранится в репозитории?}
  Было рассмотрено два варианта:

  \begin{columns}[b]
    \begin{column}{0.45\textwidth}
      \begin{figure}
        \centering
        \begin{tikzpicture}
          \matrix [draw=black] 
          {\vecd{a} & \vecd{b} & \vecd{c} & 
            \vecd{d} & \vecd{e} & \vecd{f} \\};
        \end{tikzpicture}
        \caption{Вектор константной длины}
      \end{figure}
    \end{column}
    
    \begin{column}{0.45\textwidth}
      \begin{figure}
        \centering
        \begin{tikzpicture}
          \node[treev] {a}
            child[treearr] {node[treev] {b}
              child[treearr] {node[treev] {c}
              }
              child[treearr] {node[treev] {d}
              }
            }
            child[treearr] {node[treev] {e}
            };
        \end{tikzpicture}
        \caption{Двоичное дерево с элементами в вершинах}
      \end{figure}
    \end{column}
  \end{columns}

\end{frame}

\begin{frame}[fragile]{Патч для векторов}
  \begin{itemize}
  \item Патч:
     \begin{tikzpicture}[anchor=center]
       \matrix [draw=red]
       {\vecf & \vect{a}{b} & \vecf & \vecf       & \vect{z}{y} \\};
     \end{tikzpicture}~--- заменить $a$ на второй позиции на $b$, $z$
     на пятой~--- на $y$.
  \item \emph{Форма патча}~--- места вектора, которые он меняет:
     \begin{tikzpicture}
       \matrix
       {\vecfe & \vecff & \vecfe & \vecfe & \vecff \\};
     \end{tikzpicture}
  \end{itemize}
\end{frame}

\begin{frame}[fragile]{Патчи для деревьев и их форма}
  
  \begin{tikzpicture}
    \node[treev,font=\small] (from) {$A$};
    \node[treev,font=\small, right=4mm of from] (to) {$B$};
    \path[draw, -latex, very thick] (from) -- (to); 

    \node[outer sep=0,fit=(from)(to),draw=black] (patch){};
    \begin{pgfonlayer}{background}
      \node [fill=yellow,fit=(from)(to)] {};
    \end{pgfonlayer}
  \end{tikzpicture}~--- заменить дерево $A$ на дерево $B$

  \begin{columns}[b]
    \begin{column}{0.65\textwidth}
      \begin{figure}
        \centering
        \begin{tikzpicture}
          \node[treev,font=\small] (from) {a}
          child {node[treev,font=\small,yshift=7mm] {b}
          }
          child[missing] {}
          ;
          \node[fit=(from)(from-1),inner sep=0,outer sep=0] (whole from) {};

          \node[treev,font=\small, right=4mm of from] (to) {c}
          child[missing] {}
          child {node[treev,font=\small,yshift=7mm] {d}     
          }
          ;
          \node[fit=(to)(to-2),inner sep=0,outer sep=0] (whole to){};
          \path[treearr] (whole from) -- (whole to); 

          \node[outer sep=0,fit=(whole from)(whole to),draw=black] (patch){};
          \begin{pgfonlayer}{background}
            \node [fill=yellow,fit=(whole from)(whole to)] {};
          \end{pgfonlayer}
          
          \node[treev,font=\small,above right=of patch] (tt) {};
          \path[treearr] (tt) -- (patch);
          
          \node[treev,font=\small,below right=of tt] (tr) {};
          \path[treearr] (tt) -- (tr);
          
          \node[treev,font=\small,below left=of tr] (from2) {e};
          \node[right=4mm of from2] (to2) {$\varepsilon$};
          \path[treearr] (from2) -- (to2);

          \node[outer sep=0,fit=(from2)(to2),draw=black] (patch2){};
          \begin{pgfonlayer}{background}
            \node [fill=yellow,fit=(from2)(to2)] {};
          \end{pgfonlayer}
          
          \path[treearr] (tr) -- (patch2);
          
          \node[below right=of tr,xshift=-7mm,yshift=2.5mm] (trr) {};
          \path[treearr] (tr) -- (trr);
        \end{tikzpicture}
        \caption{Патч для дерева}
      \end{figure}
    \end{column}
    \begin{column}{0.35\textwidth}
      \begin{figure}
        \centering
        \begin{tikzpicture}
          \node[treev] {}
          child[treearr] {node[treev,fill=patchform] {}
          }
          child[treearr] {node[treev] {}
            child[treearr] {node[treev,fill=patchform] {}
            }
            child[treearr] {node[treev,fill=green] {}
            }
          }
          ;
        \end{tikzpicture}
        \caption{Его форма}
      \end{figure}
    \end{column}
  \end{columns}
\end{frame}
\begin{frame}[fragile]{Совместность форм}
  \begin{itemize}
  \item Формы совместны $\Rightarrow$ патчи можно применять
    \emph{параллельно}
  \item Совместные формы можно \emph{объединять}
  \item 
    \begin{tikzpicture}\matrix 
      {\vecfe & \vecff & \vecfe & \vecfe & \vecff \\};
    \end{tikzpicture} и 
    \begin{tikzpicture}\matrix 
      {\vecfe & \vecfe & \vecff & \vecff & \vecfe \\};
    \end{tikzpicture} совместны
    \begin{itemize}
    \item Вместе~--- 
      \begin{tikzpicture}\matrix 
        {\vecfe & \vecff & \vecff & \vecff & \vecff \\};
      \end{tikzpicture}
    \end{itemize}
  \item 
    \begin{tikzpicture}\matrix 
      {\vecfe & \vecff & \vecfe & \vecfe & \vecff \\};
    \end{tikzpicture} и 
    \begin{tikzpicture}\matrix 
      {\vecfe & \vecfe & \vecff & \vecff & \vecff \\};
    \end{tikzpicture} несовместны
  \end{itemize}
\end{frame}
\newcommand{\vecf}[0]{\node[vecpatch-f] {};}
\newcommand{\vect}[2]{\node[vecpatch-t] {$#1 \mapsto #2$};}

\begin{frame}[fragile]{Вектора, неконфликтующее объединение}
  \begin{itemize}
  \item Порядок не важен
  \end{itemize}

  \begin{figure}
    \centering
    \begin{tikzpicture}
      \matrix [draw=red] (lhs)
      {\vecf & \vect{a}{b} & \vecf & \vecf       & \vect{z}{b} \\};
      
      \node [vecpatch-op, right=of lhs] {$\wedge$};

      \matrix [draw=red, below=of lhs] (rhs)
      {\vecf & \vecf       & \vecf & \vect{x}{x} & \vecf \\};

      \node [vecpatch-op, left=of rhs] {$\wedge$};
      \node [vecpatch-op, right=of rhs] {$=$};
      
      \matrix [draw=red, below=of rhs] (res)
      {\vecf & \vect{a}{b} & \vecf & \vect{x}{x} & \vect{z}{b} \\};
      
      \node [vecpatch-op, left=of res] {$=$};
    \end{tikzpicture}
  \end{figure}
\end{frame}



\end{document}
