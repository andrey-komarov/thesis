\documentclass[14pt]{beamer}
\usepackage[utf8]{inputenc}
\usepackage{amsmath}
\usepackage{amssymb}
\usepackage[english,russian]{babel}
\hypersetup{colorlinks=true, urlcolor=blue}
\usepackage{xcolor}
\usetheme{Warsaw}

\setbeamertemplate{caption}[numbered]
\setbeamerfont{page number in head/foot}{size=\large}
\setbeamertemplate{footline}[frame number]

\colorlet{patchform}{blue!40}
\usepackage{tikz}
\usetikzlibrary{shapes.multipart}
\usetikzlibrary{chains}
\usetikzlibrary{fit}
\usetikzlibrary{backgrounds}
\usetikzlibrary{arrows}
\tikzset{vecpatch/.style={
    draw=black, 
    minimum width=0.5cm, 
    minimum height=1cm,
    inner sep=-1cm,
    outer sep=-1cm,
    anchor=center}
}
\tikzset{vecpatch-tiny/.style={vecpatch, minimum height=0.5cm}}
\tikzset{vecpatch-f/.style={vecpatch}}
\tikzset{vecpatch-t/.style={vecpatch,fill=yellow}}
\tikzset{vecpatch-d/.style={vecpatch-tiny,anchor=base}}
\tikzset{vecpatch-op/.style={font=\large}}
\tikzset{vecpatch-form/.style={vecpatch-tiny,fill=patchform}}

\tikzset{treev/.style={draw=black,circle,outer sep=0,minimum
    height=8mm,very thick}}
\tikzset{treearr/.style={draw, -latex, very thick}}

\newcommand{\vecfe}[0]{\node[vecpatch-tiny] {};}
\newcommand{\vecff}[0]{\node[vecpatch-form] {};}
\newcommand{\vecf}[0]{\node[vecpatch-f] {};}
\newcommand{\vecd}[1]
{\node[vecpatch-d, draw=none] {$#1$}; 
 \node[vecpatch-d, text opacity=0] {b};}
\newcommand{\vect}[2]{\node[vecpatch-t]
{\begin{tabular}{l}$#1$\\$#2$\\\end{tabular}};}

\title{Аксиоматизация системы контроля версий в формализме зависимых типов}
\author{Андрей Комаров, гр. 4538}
\institute{НИУ ИТМО}
\date{\today}

\begin{document}

\begin{frame}
  \maketitle

  \begin{flushright}
    Научный руководитель: Малаховски Я. М.
  \end{flushright}
\end{frame}

\begin{frame}{Цель}
  \begin{block}{Глобальная}
  \begin{itemize}
  \item Реализовать формальный <<каркас>> системы контроля версий
  \item на <<логицистическом>> аппарате.
  \item Доказать его свойства.
  \end{itemize}
  \end{block}

  \begin{block}{Этой работы}
  \begin{itemize}
  \item Формализовать несколько частных случаев этого аппарата
    \begin{itemize}
    \item на языке Agda.
    \end{itemize}
  \item Доказать какие-то их свойства.
  \end{itemize}
  \end{block}
\end{frame}

\begin{frame}[fragile]{Что хранится в репозитории?}
  Было рассмотрено два варианта:

  \begin{columns}[b]
    \begin{column}{0.45\textwidth}
      \begin{figure}
        \centering
        \begin{tikzpicture}
          \matrix [draw=black] 
          {\vecd{a} & \vecd{b} & \vecd{c} & 
            \vecd{d} & \vecd{e} & \vecd{f} \\};
        \end{tikzpicture}
        \caption{Вектор константной длины}
      \end{figure}
    \end{column}
    
    \begin{column}{0.45\textwidth}
      \begin{figure}
        \centering
        \begin{tikzpicture}
          \node[treev] {a}
            child {node[treev] {b}
              child {node[treev] {c}
              }
              child {node[treev] {d}
              }
            }
            child {node[treev] {e}
            };
        \end{tikzpicture}
        \caption{Двоичное дерево с элементами в вершинах}
      \end{figure}
    \end{column}
  \end{columns}

\end{frame}

\begin{frame}[fragile]{Форма патча}
  \begin{itemize}
  \item Патч~--- преобразование хранимого в репозитории: 
    \begin{itemize}
    \item 
      \begin{tikzpicture}[anchor=center]
        \matrix [draw=red] 
        {\vecf & \vect{a}{b} & \vecf & \vecf       & \vect{z}{y} \\};
      \end{tikzpicture}~--- заменить $a$ на второй позиции на $b$, $z$
      на пятой~--- на $y$.
    \end{itemize}
  \item \emph{Форма патча}~--- что он меняет:
    \begin{itemize}
    \item
      \begin{tikzpicture}
        \matrix 
        {\vecfe & \vecff & \vecfe & \vecfe & \vecff \\};
      \end{tikzpicture}
    \end{itemize}
  \end{itemize}
\end{frame}

\input{src/30-trees.tex}
\begin{frame}[fragile]{Совместность форм}
  \begin{itemize}
  \item Формы совместны $\Rightarrow$ патчи можно применять
    \emph{параллельно}
  \item Совместные формы можно \emph{объединять}
  \item 
    \begin{tikzpicture}\matrix 
      {\vecfe & \vecff & \vecfe & \vecfe & \vecff \\};
    \end{tikzpicture} и 
    \begin{tikzpicture}\matrix 
      {\vecfe & \vecfe & \vecff & \vecff & \vecfe \\};
    \end{tikzpicture} совместны
    \begin{itemize}
    \item Вместе~--- 
      \begin{tikzpicture}\matrix 
        {\vecfe & \vecff & \vecff & \vecff & \vecff \\};
      \end{tikzpicture}
    \end{itemize}
  \item 
    \begin{tikzpicture}\matrix 
      {\vecfe & \vecff & \vecfe & \vecfe & \vecff \\};
    \end{tikzpicture} и 
    \begin{tikzpicture}\matrix 
      {\vecfe & \vecfe & \vecff & \vecff & \vecff \\};
    \end{tikzpicture} несовместны
  \end{itemize}
\end{frame}
\newcommand{\vecf}[0]{\node[vecpatch-f] {};}
\newcommand{\vect}[2]{\node[vecpatch-t]
{\begin{tabular}{l}$#1$\\$#2$\\\end{tabular}};}

\begin{frame}[fragile]{Вектора, объединения}
  \begin{columns}[T]
    \begin{column}{0.45\textwidth}
      \begin{figure}
        \centering
        \begin{tikzpicture}
          \matrix [draw=red] (lhs)
          {\vecf & \vect{a}{b} & \vecf & \vecf       & \vect{z}{b} \\};
          
          \node [vecpatch-op, right=0 of lhs] {$\wedge$};

          \matrix [draw=red, below=3mm of lhs] (rhs)
          {\vecf & \vecf       & \vecf & \vect{x}{x} & \vecf \\};

          \node [vecpatch-op, left=0 of rhs] {$\wedge$};
          \node [vecpatch-op, right=0 of rhs] {$=$};
          
          \matrix [draw=red, below=3mm of rhs] (res)
          {\vecf & \vect{a}{b} & \vecf & \vect{x}{x} & \vect{z}{b} \\};
          
          \node [vecpatch-op, left=0 of res] {$=$};
        \end{tikzpicture}
        \caption{Неконфликтующее объединение}
      \end{figure}
    \end{column}

    \begin{column}{0.45\textwidth}
      \begin{figure}
        \centering
        \begin{tikzpicture}
          \matrix [draw=red] (lhs)
          {\vecf & \vect{a}{b} & \vect{x}{y} & \vecf & \vect{z}{y} \\};
          
          \node [vecpatch-op, right=0 of lhs] {$\ggg$};

          \matrix [draw=red, below=3mm of lhs] (rhs)
          {\vecf & \vect{b}{c} & \vecf       & \vecf & \vect{y}{x} \\};

          \node [vecpatch-op, left=0 of rhs] {$\ggg$};
          \node [vecpatch-op, right=0 of rhs] {$=$};
          
          \matrix [draw=red, below=3mm of rhs] (res)
          {\vecf & \vect{a}{c} & \vect{x}{y} & \vecf & \vect{z}{x} \\};
          
          \node [vecpatch-op, left=0 of res] {$=$};
        \end{tikzpicture}
        \caption{Конфликтующее объединение}
      \end{figure}
    \end{column}
  \end{columns}

\end{frame}



\end{document}
